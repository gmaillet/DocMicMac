\chapter{Using satellite images}

This chapter presents some tools for using satellite images in the {\tt MicMac} workflow.

\section{Pleiades-Spot images}

A tool to convert Pleiades files into localization grids called {\tt Dimap2Grid} allows using Pleiades images with {\tt Malt}.\\*

User has to provide:
\begin{itemize}
\item a RPC\footnote{Rational Polynomial Coefficient}/DIMAP file for each image
\item cartographic step (pixels)
\item projection system
\end{itemize}

As initial localization is not enough to run a good correlation process, a refine step has been added to improve the grid.

\subsection{Image couple}

The way to get grids from a Pleiades image couple is as following:

\begin{itemize}
\item Dimap2Grid for each image (produces a rough grid)
\item Tapioca (generates tie-points)
\item RefineModel (computes affinity coefficients)
\item Dimap2Grid with option refineCoef (produces accurate grid)
\end{itemize}

The command to use for this workflow are:
\begin{verbatim}
mm3d TestLib Dimap2Grid dimapFile imageFile altitudeMin altitudeMax nbLayers
mm3d Tapioca All imagesPattern -1
mm3d TestLib RefineModel image_1.GRI image_2.GRI pts_1_2.dat meanAltitude
mm3d TestLib Dimap2Grid dimapFile imageFile altMin altMax nbLay refineCoef=refine/refineCoef.txt
\end{verbatim}

To know the syntax of {\tt Dimap2Grid}:
\begin{verbatim}
*****************************
*  Help for Elise Arg main  *
*****************************
Mandatory unnamed args :
  * string :: {Dimap file}
  * string :: {Image name}
  * REAL :: {altitude min}
  * REAL :: {altitude max}
  * INT :: {number of layers}
Named args :
  * [Name=targetSyst] string :: {target system Proj4 +init=IGNF:LAMB93+datum}
  * [Name=stepPixel] REAL :: {Step in pixel}
  * [Name=stepCarto] REAL :: {Step in m (carto)}
  * [Name=sampCrop] INT :: {upper left samp - crop}
  * [Name=rowCrop] INT :: {upper left row - crop}
  * [Name=refineCoef] string :: {File of Coef to refine Grid}
\end{verbatim}

{\tt Dimap2Grid} generates a .GRI file whose name is set after image name.
First two mandatory arguments should be quite obvious. Other arguments are:
\begin{itemize}
\item min and max altitude: ground min and max altitude
\item number of layers: number of altitudes layers for the grid
\item targetSyst: target coordinate system, in the proj4 syntax
\item stepPixel: grid step in image coordinates (default = 100 pixels)
\item stepCarto: grid step in cartographic coordinates (default = 50m)
\item sampCrop: line of the upper left corner (for image cropping)
\item rowCrop: row of the upper left corner (for image cropping)
\item refineCoef: file produced by RefineModel command to refine Grid
\end{itemize}

{\tt RefineModel} syntax is:

\begin{verbatim}
*****************************
*  Help for Elise Arg main  *
*****************************
Mandatory unnamed args :
  * string :: {master image GRID}
  * string :: {slave image GRID}
  * string :: {Tie Points}
  * REAL :: {average altitude of the TiePoints}
\end{verbatim}

Tie-points string is the tie-point file (.dat) computed by Tapioca and available in the {\tt Homol} directory.\\*

Once GRI files have been computed, they have to be converted to binary files:
\begin{verbatim}
mm3d Gri2Bin path/file.GRI path/file.GRIBin
\end{verbatim}

Then, correlation can be run in ground geometry, with following command:

\begin{verbatim}
mm3d Malt UrbanMNE ".*JP2" GRIBin MOri=GRID BoxTerrain=[X1,Y1,X2,Y2]
ZoomI=32 ZoomF=1 ZMoy=100 ZInc=500 NbVI=2
\end{verbatim}

\begin{itemize}
 \item MOri= GRID states that we work with grid files
 \item   BoxTerrain are region of interest Lambert93 coordinates
 \item   ZoomI=32: first step is done with images rescaled with factor 32
 \item   ZoomF=1 : last step at full resolution
 \item   ZMoy=100  Z mean value
 \item   ZInc=100 uncertainty value around Z (in meter)
 \item   NbVI=2 minimal number of visible images  (by default 3 for UrbanMNE)
\end{itemize}

\subsection{Set of Images}

The way to get grids from a set of Pleiades images is the same as for an image couple, except for the refine stage, for which the command is (right now) slightly different:

{\tt Refine} syntax is:

\begin{verbatim}
*****************************
*  Help for Elise Arg main  *
*****************************
Mandatory unnamed args :
  * string :: {GRID files pattern}
Named args :
  * [Name=DTM] string :: {DTM file}
  * [Name=ExpRes] bool :: {Export residuals (def=false)}
\end{verbatim}

So the command to use for this workflow looks like:
\begin{verbatim}
mm3d TestLib Refine .*GRI
\end{verbatim}

First mandatory argument is the pattern for GRID files. Optional arguments are:
\begin{itemize}
\item DTM: use a DTM file (xml) to constrain depth (not supported yet)
\item ExpRes: export residuals to refine/residus.txt (image coordinates and image residuals) and refine/residusGlob.txt (ground coordinates and rms error) (default = false)
\end{itemize}

\section{SAKE - Simplified tool for satellite images correlation}

SAKE stands for ``SAtellite Kit for Elevation'' and it is a simplified tool to generate DEMs and ortho-images from sets of satellite images, developed by Ana-Maria Rosu. Like MM2DPosSism and FDSC, it is also part of the collaboration between IPGP and IGN/ENSG, and funded by CNES through the TOSCA program. 
