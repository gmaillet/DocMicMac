\chapter{Advanced matching, practical aspect}


\section{Cost function}


The discrete version of the cost function is:


\begin{equation}
   E(Z) = \sum Corr(i,j,Z(i,j)) + F(|Z(i+1,j)-Z(i,j)|)  + \dots
\end{equation}

The $F$ function allows to control the a priori on the Z:

\begin{itemize}
   \item  if the desired model is smooth, a convex   $F$  can be adequate (it's better to
          climb a given jump by regular step);

   \item  if  the desired model has many discontinuities, a $F$   concave can be adequate
          (it's better to climb a given jump in one single step);

   \item when there is no strong a prior, the  default choice is to have $F$ linear;

\end{itemize}


The basic form of $F$ has two parameters $R$ and $Q$ :

\begin{equation}
   F(\Delta_Z) =  R |\Delta_Z| + Q {\Delta_Z}^2
\end{equation}


This allows to create linear and convex function. To create concavity,
there exists parameters $S$ and $A$ such that:

\begin{equation}
   F(\Delta_Z) =  R |\Delta_Z| + Q {\Delta_Z}^2 ,  |\Delta_Z| < S
\end{equation}

\begin{equation}
   F(\Delta_Z) =  R S +   R A (|\Delta_Z|-S)   + Q {\Delta_Z}^2 ,  |\Delta_Z| \geq  S
\end{equation}


Typically this means that when $Z$ is over the threshold $S$, the slope is multiplied by $A$.

\begin{itemize}
   \item {\bf ZRegul} add a linear term $ZRegul |\Delta_Z| $ to $F$;
\end{itemize}


The MicMac tag for this parameters are:

\begin{itemize}
   \item {\tt <ZRegul>}  for $R$;
   \item {\tt <ZRegul\_Quad>}  for $Q$;
   \item {\tt <SeuilAttenZRegul>}  for $S$;
   \item {\tt <AttenRelatifSeuilZ>}  for $A$;
\end{itemize}






