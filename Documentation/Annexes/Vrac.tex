\chapter{Vrac}

Ce chapitre comporte un certain nombre de notes, qui n'ont
pas forc\'ement grand chose \`a voir avec MicMac, mais
que je met ici en attendant de savoir quoi en faire (bref,
c'est du Vrac \dots).

%-------------------------------------------------------------------
%-------------------------------------------------------------------
%-------------------------------------------------------------------

\section{Notation}

On caract\'erise une pose $P$ par un couple $C,R$ ou $C$ est le centre
perspectif et $R$ la rotation. Soit $p_c$ un point en coordonn\'ees
cam\'era et $p_m$ sont homologues en coordonn\'ees monde, on a :

\begin{equation}
   p_m = C + R p_c
\end{equation}

Le couple $C,R$  caract\'erise une rotation affine; la composition
des application donne une structure de groupe naturelle

\begin{equation}
   (C_1,R_1) * (C_2,R_2) = (C1+R_1C_2,R_1R2)
\end{equation}

Soit $P_1$ et $P_2$ deux poses de cam\'era; on s'int\'eresse \`a l'orientation
relative; on va calculer les poses $P^1_1$ et $P^1_2$  dans le rep\`ere li\'e \`a la cam\'era 
$1$.  On peut  a naturellement  $P^1_1=Id$.

\begin{equation}
   C^1_1 = \begin{pmatrix} 0 \\ 0  \\ 0 \end{pmatrix}
   R^1_1 = \begin{pmatrix} 1 &  0  & 0 \\ 0  & 1 &  0   \\ 0 & 0 & 1 \end{pmatrix}
 \end{equation}


L'orientation relative \'etant d\'efinie \`a un facteur d'\'echelle pr\`es 
on peut  poser arbitrairement

\begin{equation}
   || \overrightarrow{C^1_1 C^1_2} || = || C^1_2 || = 1
 \end{equation}

Si l'on conna\^it par ailleurs la pose absolue $P_1$ et la pose relative
$P^1_2$, alors la pose absolue de la cam\'era 2 est :

\begin{equation}
    (C_1+\lambda R_1C^1_2,R_1R^1_2)
\end{equation}

Le terme $\lambda$ repr\'esente la facteur d'\'echelle, pour le reste il
suffit de remarquer que $p_m = P1 P^2_1 p2$.

%-------------------------------------------------------------------
%-------------------------------------------------------------------

\section{G\'eom\'etrie \'epipolaire}

On voit facilement qu'il
existe une infinit\'e de rotation  dont l'image du  premier vecteur
directeur , ie $^t \begin{pmatrix} 1 &  0  &  0 \end{pmatrix}$,
est l'axe $\overrightarrow{C_1 C_2} $; elles se 
d\'eduisent d'ailleurs les un des autres par une rotation autour
de l'axe $C_1 C_2$.   Soit $R^e$ une de ces rotation,
on  appel\'es poses \'epipolaires les poses de la forme :
$(C_1 R^e)$ et $(C_2 R^e)$.

L'int\'er\^et de ces poses \'epipolaires vient des remarques suivantes :


L'homologue d'une direction $^t \begin{pmatrix} x^e_1 &  y^e_1  &  1 \end{pmatrix}$
         est un direction $^t \begin{pmatrix} x^e_2 &  y^e_2  &  1 \end{pmatrix}$ 
	 \emph{avec} $ y^e_1 =  y^e_2$; la contrainte d'homologie entre points
	 images y est donc particuli\'erement facile a exprimer;

\begin{equation}
   y^e_1 =  y^e_2
\end{equation}

Le passage de $C_1 R_1$ \`a $C_1 R^e$ (vs  $C_2 R_2$ \`a $C_2 R^e$ ) est
    est purement vectoriel puisque les centres sont confondus 
    %$p^e_1 = {R^e }^{-1} R_1 p_1$;

\begin{equation}
     p^e_1 = ({R^e }^{-1} R_1)  p_1
\end{equation}

On pose   :

\begin{equation}
      R^e_1= ({R^e }^{-1} R_1) 
\end{equation}

On a :

\begin{equation}
\label{CEpiCnorm}
     \begin{pmatrix} x^e_1 &  y^e_1  &  1 \end{pmatrix} 
     = R^e_1 \begin{pmatrix} x_1 &  y_1  &  1 \end{pmatrix}
\end{equation}


%-------------------------------------------------------------------

\section{Matrice essentielle}

Avec des poses
epipolaire on a toujours la relation  entre directions
homologues :

\begin{equation}
      \begin{pmatrix} x^e_1 &  y^e_1  &  1 \end{pmatrix}
      \begin{pmatrix} 0 & 0 & 0 \\ 0 & 0 & -1   \\  0 & 1 & 0 \end{pmatrix}
      \begin{pmatrix} x^e_2 \\  y^e_2  \\  1 \end{pmatrix}
      =   y^e_2 - y^e_1 
      = 0
 \end{equation}

Notons :

\begin{equation}
      \Ess_0 =
      \begin{pmatrix} 0 & 0 & 0 \\ 0 & 0 & -1   \\  0 & 1 & 0 \end{pmatrix}
 \end{equation}

En utilisant la relation~\ref{CEpiCnorm}, on a :

\begin{equation}
    ^t p_1 ^t R^e_1  \Ess_0 R^e_2 p_2 = 0
\end{equation}

On voit donc qu'il existe une  matrice,  $\Ess_{1,2}$,  appel\'ee
matrice essentielle telle que  pour tout couple de direction homologue
$p_1,p_2$ on ait :

\begin{equation}
    ^t p_1  \Ess_{1,2}  p_2 = 0
 \end{equation}


%-------------------------------------------------------------------

\section{Calcul de l'orientation relative par matrice essentielle}


%-------------------------------------------------------------------

\section{Cas planaire}


%-------------------------------------------------------------------
\section{Grandes focales, projection axo et points triples}


Il existe une rotation :f

\section{G\'eom\'etrie \'epipolaire}
%-------------------------------------------------------------------
%-------------------------------------------------------------------

\section{Matrice essentielle}



%-------------------------------------------------------------------
\section{Grandes focales, projection axo et points triples}
