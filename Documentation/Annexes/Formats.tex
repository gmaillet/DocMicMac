\chapter{Formats}

%-------------------------------------------------------------------
%-------------------------------------------------------------------
%-------------------------------------------------------------------

\section{Calibration formats}

\label{FORM:CALIB}

\section{Grilles de calibration}
\label{Format:Grille}

Cette section d\'ecrit le format grille utilis\'e par MicMac pour
coder  la calibration interne des cam\'eras 
de mani\`ere g\'en\'erique (c.a.d. ind\'ependemment du mod\'ele
param\'etrique choisi).

%-------------------------------------------------------------------
\subsection{Format de codage des d\'eformations du plan }

\subsubsection{Format}

A bas niveau, le format de grille est essentiellement un format {\tt Xml}  simple
adapt\'e au codage des d\'eformations r\'eguli\`eres du plan;
c'est \`a dire des objets qui correspondent au concept math\'ematique
de bijection  $\mathcal{C}^\infty$ d'une partie de $\RR^2$ dans une
autre partie. Soit $\psi$ une telle bijection, le format est sp\'ecifi\'e
de la mani\`ere suivante :

\begin{itemize}
   \item le tag p\`ere porte le nom de {\tt  <doublegrid>};
   \item ce tag comporte deux fils qui ont rigoureusement la m\^eme structure,
         le fils  {\tt <grid\_directe>} correspond \`a une tabulation 
         de valeurs de $\psi$ et le fils  {\tt <grid\_inverse>} 
         correspond \`a une tabulation de  de valeurs de $\psi^{-1}$ ;
\end{itemize}

Chaque valeur {\tt <grid\_directe>} et {\tt <grid\_inverse>}  correspond
donc \`a la tabulation des valeurs d'une d\'eformation d'un domaine de $\RR^2$.
Consid\'erons par exemple la fonction directe $\psi$ et notons
$\psi(P)= (\psi_x(P),\psi_y(P))$ ses deux composantes; il faut d\'efinir le
domaine sur lequel $\psi$ est tabul\'e, la r\'esolution choisie, et les
valeur $\psi_x$ et $\psi_y$; la tabulation est alors d\'efinie par :

\begin{itemize}
    \item un origine $\mathcal{O}$ (tags {\tt <origine><x>} et {\tt <origine><y>});
    \item un pas $\Delta$ ( tag {\tt <step>}); 
    \item deux  tableaux de valeurs $Data_x$ et $Data_y$  d\'efinis
          par  deux noms de fichier (tags {\tt <filename><x>}  
          et {\tt <filename><y>}) et une taille  $(T_x,T_y)$
          (tags {\tt <size><x>} et {\tt <size><y>});
          chaque fichier est une suite de  $T_x*T_y$ nombres r\'eels cod\'es sur des
          doubles (format "raw");
\end{itemize}

Les valeurs des tableaux $Data_x$ et  $Data_y$, contenus dans les fichiers 
{\tt <filename>}, sont d\'ecrites par les \'equations~\ref{GRID:TABUL:X}
et~\ref{GRID:TABUL:Y}.

\begin{equation}
   Data_x[I+T_x J] = \psi_x(\mathcal{O}_x+\Delta I,\mathcal{O}_y+\Delta J)
   \label{GRID:TABUL:X}
\end{equation}
\begin{equation}
   Data_y[I+T_x J] = \psi_y(\mathcal{O}_x+\Delta I,\mathcal{O}_y+\Delta J)
   \label{GRID:TABUL:Y}
\end{equation}

\subsubsection{Utilisation}

\label{Util:Grid}

En soi, le format  ne sp\'ecifie rien de plus que des conventions
pour tabuler des valeurs
de  $\psi$ sur une portion rectangulaire d'une  grille  r\'eguli\`ere.
Une utilisation courante possible, pour estimer $\psi(P)$ en un point quelconque,
pourrait consister \`a effectuer les op\'erations suivantes~:

\begin{itemize}
   \item calculer les "indices r\'eels" $i=\frac{P_x-\mathcal{O}_x}{\Delta}$
         et $j=\frac{P_y-\mathcal{O}_y}{\Delta}$;
    \item consid\'erer $Data_x$ et $Data_y$ comme des images et utiliser
           un sch\'ema d'interpolation (par exemple bilin\'eaire) pour
           calculer leurs valeurs au point r\'eel $(i,j)$.
\end{itemize}


%-------------------------------------------------------------------
%-------------------------------------------------------------------
%-------------------------------------------------------------------

\subsection{Application \`a la calibration interne}

\subsubsection{Rappels et notation}

Cette section donne un rappel des principaux mod\`eles de distorsion
param\`etrique courament utilis\'es lors du calcul de la calibration
interne. La lecture de cette section n'est pas strictement n\'ecessaire
\`a la compr\'ehension du format.


Soit une cam\'era $C$,
on consid\`ere un point $P=^t(x,y,z)$ du  monde objet (espace $\RR^3$) et son
homologue $Q_c=^t(u,v)$ dans l'image (espace $\RR^2$);
on note $\pi_c$  la fonction qui associe $Q$ \`a $P$ :

\begin{equation}
   Q=\pi_c(P)
\end{equation}

On se limite aux cam\'eras st\'enop\'e; on note 

\begin{itemize}
   \item $\pi_0(^t(x,y,z))=\frac{^t(x,y)}{z}$, projection "canonique";
   \item  la position d'une cam\'era dans l'espace objet est d\'efinie
             par ses param\`etre "extrins\`eques", c'est \`a dire
             centre optique $\mathcal{C}$ et   matrice de 
             rotation $\mathcal{R}$ 
   \item     la cam\'era elle-m\^eme est d\'efinie par ses param\`etres
              intrins\`eques    distance focale $F$ et  point principal $P^p$; 
\end{itemize}

Pour une cam\'era "id\'eale", on a la relation  :

\begin{equation}
   \pi(P) =   P^p+F*\pi_0(\mathcal{R}(P-\mathcal{C}))
   \label{Proj:Cam:Ideale}
\end{equation}


En r\'ealit\'e diff\'erents ph\'enom\`enes sont
succeptible de modifier la relation~\ref{Proj:Cam:Ideale}. Le 
ph\'enom\`ene pr\'epond\'erant est en g\'en\'eral une distorsion
radiale \footnote{parce que le syst\`eme optique est en premi\`ere
approximation \`a sym\'etrie
de r\'evolution}. Toute fonction radiale est caract\'eris\'ee par
son centre de sym\'etrie $C_r$ et une fonction de distorsion
$\phi$ (n\'ecessairement paire si la fonction r\'esultante est 
$C^\infty$) qui ne d\'epend que du rayon.
G\'en\'eralement  $\phi$ est mod\'elis\'ee par un polynome~:

\begin{equation}
   \phi(R) =   1 + a_2 R^2 + a_4 R^4 +\dots
\end{equation}
\begin{equation}
      P-C_r= ^t(x_c,y_c) \; \; R=\sqrt{x_c^2+y_c^2}
\end{equation}
\begin{equation}
   D_{\phi,C_r}(P) =   C_r + (P-C_r) *  \phi(R) 
   \label{MODEL:RADIAL}
\end{equation}

Dans ce mod\`ele on a alors :

\begin{equation}
   \pi(P) =   D_{\phi,C_r}(P^p+F*\pi_0(\mathcal{R}(P-\mathcal{C})))
\end{equation}


Lorsque l'on suppose que  les axes optique des diff\'erentes lentilles
ne sont pas rigoureusement confondus, un d\'eveloppement limit\'es
conduit \`a introduire  une distorsion de
d\'ecentrement d\'efinie par deux param\`etres $\alpha$ et $\beta$ :


\begin{equation}
   D_{\phi,C_r}^{\alpha,\beta}(P) =    D_{\phi,C_r}(P)
       + ^t(\alpha(2 x_c^2 + R^2)+2\beta x_c y_c ,\beta(2y_c^2+R^2) + 2\alpha x_c y_c)
   \label{MODEL:DECENTRE}
\end{equation}

Si l'on suppose que le plan du capteur n'est pas rigoureusement orthogonal
\`a  l'axe optique, on rajoute un terme affine (voir~\ref{DIST:ROT:PRES}
pourquoi il n'y a que deux termes supl\'ementaires ) :

\begin{equation}
   D_{\phi,C_r,A,B}^{\alpha,\beta}(P) =   
   D_{\phi,C_r}^{\alpha,\beta}(P) + ^t(A x_c+B y_c,0)  
   \label{Modele:Fraser}
\end{equation}

L'\'equation~\ref{Modele:Fraser} correspond au mod\`ele dit 
\emph{photogram\'etrique standard}. Ce mod\`ele couvre la plupart
des cas courants; cependant d'autre ph\'enom\`enes non pris
en compte par le mod\`ele photogram\'etrique standard  sont suceptibles d'intervenir :
non plan\'eit\'e du capteur, d\'eformations planim\'etriques,
d\'efauts haute fr\'equence dans les filtres plac\'es
en amont du syt\`eme optique \dots 

Pour les cam\'era d\'evelopp\'ees \`a l'IGN, 
le mod\`ele radial(\ref{MODEL:RADIAL}) pourrait couvrir
un peu plus de la  majorit\'e des cas test\'es et 
le mod\`ele photogram\'etrique standard
(\ref{Modele:Fraser}) donne une description satisfaisante pour plus de $80\%$
des cam\'eras.  Les autres cas (g\'en\'eralement pour les courtes focales)
n\'ec\'essitente une mod\'elisation plus compl\`ete dans la laquelle
la fonction de distortion est \emph{a priori} une fonction quelconque 
(mais "tr\`es" r\'eguli\`ere) de $\RR^2$ dans $\RR^2$. On aboutit
dans ce cas \`a la mod\'elisation suivante :

\begin{equation}
   \mathcal{D} : \RR^2 \rightarrow \RR^2 \; \; C^\infty
\end{equation}
\begin{equation}
   \pi(P) =   \mathcal{D}(\pi_0(\mathcal{R}(P-\mathcal{C})))
   \label{EQ:DIST:GEN}
\end{equation}

Dans l'\'equation \ref{EQ:DIST:GEN} les param\`etres  $F$ et $P^p$ ont
disparu car il seraient redondants avec une application $D$  qui peut
a priori \^etre quelconque.


\subsubsection{Codage des distorsion par des grilles}

Compte tenu de la relative complexit\'e du mod\`ele~\ref{Modele:Fraser},
et du fait que ce mod\`ele  n'est pas toujours suffisant, le format
d'export retenu pour les calibrations  est un format grille.
On part de  l'\'equation~\ref{EQ:DIST:GEN}  , qui correspond
au cas le plus g\'en\'eral (les \'equations 
\ref{MODEL:RADIAL}, \ref{MODEL:DECENTRE} et \ref{Modele:Fraser}
en sont \'evidemment un cas particulier), et on  utilise
le format grille pour coder $\psi=\mathcal{D}^{-1}$  consid\'er\'e
comme une application du plan dans lui m\^eme.

Plus pr\'ecis\'ement cela signifie que, une fois la grille lue,
on peut par exemple utiliser le m\'ecanisme d\'ecrit 
en~\ref{Util:Grid} pour:

\begin{itemize}
   \item \`a partir d'un point image $Q$ calculer la direction
         du rayon incident dans le rep\`ere cam\'era par 
         $^t(\psi_x(Q),\psi_y(Q),1)$

   \item  \`a partir d'un point terrain $P$, et des param\`etres 
          extrins\`eques $R$ et $C$, calculer la projection image par
          $Q= \psi^{-1}(\pi_0(\mathcal{R}(P-\mathcal{C})))$

\end{itemize}

En pratique ce format a \'et\'e utilis\'e \`a l'IGN pour exporter des calibrations 
faites selon les m\'ethodes suivantes~:

\begin{itemize}
   \item  mod\`ele radiale sur polygone de calibration;

   \item  mod\`ele photogram\'etrique standard sur polygone de calibration;

   \item  mod\`ele grille  calcul\'e par auto-calibration ; dans cette m\'ethode
          on n'utilise que  des points homologues calcul\'e de mani\`ere dense
          et c'est directement une grille qui est calcul\'ee;

   \item  combinaison de m\'ethodes pr\'ec\'edentes.

\end{itemize}


Il s'agit donc d'un format d'export qui a \'et\'e choisi
pour sa capacit\'e \`a repr\'esenter toutes les distorsions correspondant
\`a des cam\'eras st\'enop\'es sans pr\'ejuger de la m\'ethode
de calcul ou de l'eventuel mod\`ele param\'etrique sous-jacent.



\subsubsection{Justification des mod\`eles parm\'etriques}

\label{Just:Model:Decentr}


Cette section a essentiellement pour but de mettre au propre mes notes
sur la justification du mod\`ele d\'ecentrique.

On voit facilement qu'une d\'eformation $D$ $\mathcal{C}^\infty$ \`a sym\'etrie radiale
de centre $^t(a,b)$ s'\'ecrit n\'ecessairement sous la forme  :

\begin{equation}
   D_x(X,Y) = a + (X-C_x) f(\sqrt{(X-C_x)^2+(Y-C_y)^2})
\end{equation}
\begin{equation}
   D_y(X,Y) = b + (Y-C_y) f(\sqrt{(X-C_x)^2+(Y-C_y)^2})
\end{equation}

O\`u $f$ est une fonction paire et $\mathcal{C}^\infty$ . Classiquement, on suppose
alors que $f$ est approximable  
\footnote{cf th\'eor\`eme de Stone-Weierstrass} 
par un polynome en $\rho^2$ :

\begin{equation}
   f(\rho) = \sum _{k=0}^N  \alpha_k \rho ^{2k}
\end{equation}

Le terme $\alpha_0$ serait redondant avec les focales, on pose donc $\alpha_0=1$,
ce qui donne :

\begin{equation}
  X_C = X-C_x  \; \; Y_C = Y-C_y
\end{equation}
\begin{equation}
   D_x(X,Y) =  X + X_C \sum _{k=1}^N  \alpha_k (X_C^2+Y_C^2)^k
\end{equation}
\begin{equation}
   D_y(X,Y) =  Y + Y_C \sum _{k=1}^N  \alpha_k (X_C^2+Y_C^2)^k
\end{equation}

Le choix d'un mod\`ele radiale de distorsion, vient du fait que, si les lentilles
sont des objets de r\'evolution, et que leur axes optiques sont confondus, alors le
syst\`emes optique est lui m\^eme \`a sym\'etrie de r\'evolution. Lorsque l'on
remet en cause l'hypoth\`ese selon laquelle les axes optiques sont confondus,
on fait l'hypoth\`ese que la distorsion r\'esultante est une somme de distorsion
radiales de centre "tr\`es l\'eg\`erement" diff\'erent.

Soient $C$ et $C'$ les deux centres optiques quasi confondus, on \'ecrit  $C'=C-^t(a,b)$
ou  $X_{C'}= X_C+a , Y_{C'}= Y_C+b $ avec $a\approx 0$ et $b\approx 0$.


On peut \'ecrire :

\begin{equation}
   D'_x(X,Y) =  X + X_{C'} \sum _{k=1}^N  \alpha'_k (X_{C'}^2+Y_{C'}^2)^k
\end{equation}
\begin{equation}
   D'_x(X,Y) \approx  X + X_{C} \sum _{k=1}^N  \alpha'_k (X_{C}^2+Y_{C}^2)^k
                + a \frac{\partial(X_{C'} \sum _{k=1}^N  \alpha'_k \rho_{C'}^{2k})} {\partial a}
                + b \frac{\partial(X_{C'} \sum _{k=1}^N  \alpha'_k \rho_{C'}^{2k})} {\partial b}
\end{equation}

On voit que le premier terme est un terme radial de centre $C$ qui va, additionn\'e \`a
la distorsion de centre $C$, va donner des termes $\alpha_k+\alpha'_k$. Il nous reste
a \'evaluer le terme proprement "d\'ecentrique" :


\begin{equation}
   \frac{\partial(X_{C'} \sum _{k=1}^N  \alpha'_k \rho_{C'}^{2k})} {\partial a} 
    = \sum _{k=1}^N  \alpha'_k \rho_{C}^{2k-2}(\rho_C^2 + 2k X_C^2)
\end{equation}
\begin{equation}
   \frac{\partial(X_{C'} \sum _{k=1}^N  \alpha'_k \rho_{C'}^{2k})} {\partial b} 
    = \sum _{k=1}^N  \alpha'_k  2k \rho_{C}^{2k-2} X_C Y_C
\end{equation}


Soit une distorsion de d\'ecentrement $D^{C/C'}$ :

\begin{equation}
   D^{C/C'}_x = \sum _{k=1}^N  \rho_{C}^{2k-2} \alpha'_k(a\rho_C^2 +2k aX_C^2+2k b  X_C Y_C )
   \label{Eq:DecGen:X}
\end{equation}
\begin{equation}
   D^{C/C'}_y = \sum _{k=1}^N  \rho_{C}^{2k-2} \alpha'_k(b\rho_C^2 +2k bX_C^2+2k a  X_C Y_C )
   \label{Eq:DecGen:Y}
\end{equation}

Si on ne s'int\'eresse qu'au premier terme, on tombe sur la formule habituelle :


\begin{equation}
   D^{C/C'}_{x_1} =  \alpha'_k(a\rho_C^2 +2aX_C^2+2 b  X_C Y_C ) = A_1 (\rho_C^2+2X_C^2) + 2 B_1 X_C Y_C
\end{equation}
\begin{equation}
   D^{C/C'}_{y_1} =  \alpha'_k(b\rho_C^2 +2bY_C^2+2 a  X_C Y_C ) = B_1 (\rho_C^2+2Y_C^2) + 2 A_1 X_C Y_C
\end{equation}
\begin{equation}
    A_1=\alpha'_1a  \; , \; B_1=\alpha'_1b
\end{equation}

La plupart des auteurs, s'arr\^etent au premier terme. Rien n'emp\'eche \emph{a priori}
de rajouter, par exemple, le terme suivant :

\begin{equation}
   D^{C/C'}_{x_2} =   A_2 (\rho_C^4+4X_C^2\rho_C^2) + 4 B_2 X_C Y_C\rho_C^2
\end{equation}
\begin{equation}
   D^{C/C'}_{y_2} =   B_2 (\rho_C^4+4Y_C^2\rho_C^2) + 4 A_2 X_C Y_C\rho_C^2
\end{equation}
\begin{equation}
    A_2=\alpha'_2a  \; , \; B_2=\alpha'_2b
\end{equation}



%-------------------------------------------------------------------
%-------------------------------------------------------------------
%-------------------------------------------------------------------

\subsection{Application \`a une rotation pr\`es}

\label{Appl:Rot:Pres}

Cette section contient des d\'eveloppements qui sont plut\^ot de
l'ordre de la photogramm\'etrie g\'en\'erale. 
Son contenu n'est donc pas directement li\'e 
au format de grille; mais ces remarques me semblent utiles,
et il se trouve qu'elles sont d'autant plus importantes 
que le mod\`ele chosi pour \emph{calculer}
la calibration est peu contraint.

\subsubsection{Formalisation}

\label{DIST:ROT:PRES}


Soit $\mathcal{R}$ la  rotation associ\'e \`a une cam\'era, on note
ses coefficents ainsi :

\begin{equation}
\mathcal{R} =\begin{pmatrix} a & b & c \\ d & e & f \\ g & h & i \end{pmatrix}
\end{equation}

On remarque que pour $P=^t(x y z)$:

\begin{equation}
   \pi_0(\mathcal{R} P) =
    \begin{pmatrix} \frac{ax+by+cz}{gx+hy+iz} \\ \frac{dx+ey+fz}{gx+hy+iz}  \end{pmatrix}
\end{equation}

\begin{equation}
   \pi_0(\mathcal{R} P) =\begin{pmatrix} 
                 \frac{a\frac{x}{z}+b\frac{y}{z}+cz}{g\frac{x}{z}+h\frac{y}{z}+i} \\
                 \frac{d\frac{x}{z}+e\frac{y}{z}+fz}{g\frac{x}{z}+h\frac{y}{z}+i} 
               \end{pmatrix}
   \label{EQ:PI0RP}
\end{equation}

Notons $\mathcal{H_R}$ l'homographie $\RR^2 \rightarrow \RR^2$ d\'efinie par :


\begin{equation}
   \mathcal{H_R}\begin{pmatrix}u\\v\end{pmatrix}  
               =\begin{pmatrix} \frac{au+bv+c}{gu+hv+i} \\ \frac{du+ev+f}{gu+hv+i}  \end{pmatrix}
\end{equation}

L'\'equation~\ref{EQ:PI0RP} peut alors se r\'eecrire :

\begin{equation}
   \pi_0 \mathcal{R} = \mathcal{H_R} \pi_0
\end{equation}

Consid\'erons maintenant $N$ cam\'eras $C_k$, $ k\in [1\; N]$, et  leur \'equation de projection :

\begin{equation}
   \pi_k(P) =   \mathcal{D}(\pi_0(\mathcal{R}_k(P-\mathcal{C}_k)))
  \label{EQ:PROJ:NON:AMB}
\end{equation}

On voit que pour toute rotation $\mathcal{S}$ on peut \'ecrire :

\begin{equation}
   \pi_k(P) =   \mathcal{D}(\pi_0(\mathcal{S S}^{-1}\mathcal{R}_k(P-\mathcal{C}_k)))
\end{equation}

Soit encore :

\begin{equation}
  \pi_k(P)    =  (\mathcal{D}\mathcal{H_S} ) \pi_0 ((\mathcal{S}^{-1}\mathcal{R}_k)(P)-\mathcal{C}_k)))
  \label{EQ:PROJ:AMB}
\end{equation}

On remarque alors que les configuration correspondant aux  \'equations 
\ref{EQ:PROJ:NON:AMB} et \ref{EQ:PROJ:AMB} sont compl\`etement indiscernables. 
Concr\`etement cela signifie que la calibration d'une cam\'era n'est
jamais d\'efinie qu`\`a une rotation pr\`es puisque une rotation 
quelconque $\mathcal{S}^{-1}$ de l'ensemble des positions de la cam\'era peut
\^etre exactement absorb\'ee par une modification de la distorsion en
rempla\c{c}ant $\mathcal{D}$ par $\mathcal{D H_S}$

%-------------------------------------------------------------------

\subsubsection{Cons\'equence pour la comparaison de calibration}

Il est assez clair  que pour mesurer  la proximit\'e entre deux
calibrations diff\'erentes d'une m\^eme cam\'era, la mesure
d'\'ecarts entre les valeur du mod\`ele param\'etrique est une tr\`es
mauvaise m\'ethode . Les
raisons les plus \'evidentes sont :

\begin{itemize}
   \item  les diff\'erent  param\`etres ne sont pas tous homog\`enes;

   \item  dans certaines configurations, un param\`etre peuvent varier 
          librement sans que cela ait d'influence sur la calibration;
          un cas classique est celui du mod\`ele radiale lorsque la distorsion
          est quasi nulle : le centre de distorsion peut se "ballader" sans que
          cela ait d'influence sur la qualit\'e du r\'esultat final;

   \item  un ph\'enom\`ene similaire, mais plus complexe \`a
          d\'etecter est celui des param\`etres fortement corr\'el\'es pour
          lesquels les variations de l'un peuvent \^etre compens\'es par des variation 
          de l'autre sans affecter le r\'esultat de la calibration (c'est la cas du
          mod\`ele photogram\'etrique standard pour le point
          principal et le centre de distorsion);
          
   \item  enfin , de mani\`ere \'evidente, cette m\'ethode n'est pas utilisable
          pour comparer deux calibration issus de mod\`ele param\'etriques
          diff\'erents.
\end{itemize}

Une moins mauvaise m\'ethode pourrait \^etre de comparer directement les 
les direction des rayons perspectifs dans le rep\`ere cam\'era.
Repartant de l'\'equation~\ref{EQ:DIST:GEN}, cela revient \`a se
donner une distance sur les d\'eformation de $\RR^2$.
Par exemple, \'etant donn\'ee $\mathcal{D}_1$ et $\mathcal{D}_2$
deux calibration, on pourrait d\'efinir en choisissant une norme $\mathcal{L}^2$ :

\begin{equation}
   d_2(\mathcal{D}_1,\mathcal{D}_2) =  
      \iint_{\mathcal{I}} (\mathcal{D}_1(u,v)-\mathcal{D}_2(u,v))^2 d_u d_v
\end{equation}

Cette distance n'est pas bonne non plus ; en effet  soit $\mathcal{R}$
une rotation, la valeur $d_2(\mathcal{D},\mathcal{D} \mathcal{H_R})$ peut
\^etre grande alors que nous avons qu'il s'agit en fait exactement de la m\^eme
calibration.

Sans rentrer dans les d\'etails math\'ematiques, 
lorsque l'on manipule des objets
\`a une op\'eration pr\`es, la bonne distance est la distance quotient
\footnote{on ne rentre pas ici dans les conditions pour que cette
distance soit correctement d\'efinie}. Concr\'etement, soit 
$\mathcal{S}^3$ le groupe des rotations de $\RR^3$, on utilisera
la distance $\DdSt$ d\'efinie par  :


\begin{equation}
   \DdSt(\mathcal{D}_1,\mathcal{D}_2) 
   = Min_{(\mathcal{S} \in \mathcal{S}^3)} d_2(\mathcal{D}_1,\mathcal{D}_2 \mathcal{H_S})
\end{equation}

Dans le cas d'une norme  $\mathcal{L}^2$ le calcul de la 
distance $\DdSt$ est relativement ais\'e ; en effet, en prenant
quelque pr\'ecautions, on assure facilement que  le mininum de 
$d_2(\mathcal{D}_1,\mathcal{D}_2 \mathcal{H_S})$ est r\'ealis\'e
pour un valeur proche de l'identit\'e. En lin\'earisant le probl\`eme,
on trouve alors le minimum en une ou deux it\'erations.

%-------------------------------------------------------------------
%-------------------------------------------------------------------
%-------------------------------------------------------------------

\subsection{Point principal}

%-------------------------------------------------------------------
\subsubsection{Le point principal bouge}

La n\'ecessit\'e d'utiliser $\DdSt$ au lieu de $d_2$ \'etant d'autant plus
grande que  le mod\`ele est peu contraint, on pourrait esp\'erer qu'avec
les mod\`eles les plus simples on puisse faire l'\'economie de cette
complexit\'e. En fait ce n'est pas le cas notament pour les grandes
focales;  consid\'erons par exemple, une "petite" rotation
 $\mathcal{R}^x_\theta$ d'angle $\theta$ autour de l'axe $Ox$, un
d\'eveloppement limit\'e montre que $\mathcal{H}_{\mathcal{R}^x_\theta}$
est \'egale \`a une translation sur les $v$, donc \`a une modification
du point principal(avec une erreur en $\mathcal{O}(\theta uv)$). Donc, m\^eme
avec le mod\`ele le plus simple de la cam\'era id\'eale sans
distorsion (voir~\ref{Proj:Cam:Ideale})  deux calibrations peuvent
\^etre tr\`es proches en r\'ealit\'e (ie selon $\DdSt$) et assez
\'eloign\'ee avec une mesure de $d_2$.



\label{DIST:HORS:GR}
%-------------------------------------------------------------------
\subsubsection{On a perdu le point principal}

Avec les formulation habituelles de la calibration utilis\'ee dans les
\'equations \ref{Modele:Fraser},
\ref{MODEL:DECENTRE}, \ref{Proj:Cam:Ideale} ou \ref{MODEL:RADIAL}, le
point principal semble \^etre un objet math\'ematiquement  
d\'efini puisqu'il est 
intervient de mani\`ere explicite dans le mod\`ele param\'etrique.

Pour d\'efinir un \'equivalent du point principal  avec l'\'equation
\ref{EQ:DIST:GEN}, on pourrait par analogie avec les autre formule,
poser :

\begin{equation}
   P^p = \mathcal{D}^{-1} (^t(0,0))
\end{equation}

Cette m\'ethode serait erron\'ee car la calibration n'\'etant
d\'efinie qu'\`a une rotation pr\`es, cette d\'efinition peut mettre
le point principal n'importe o\`u dans l'image suivant la rotation choisie.

On voit donc que la notion de point principal, est :  

\begin{itemize}
     \item une notion peu stable pour les longues focales;

    \item quelque chose d'ind\'efini pour les mod\`ele de calibration
          quelconques.
\end{itemize}

On peut d'ailleurs pousser le raisonnement un peu plus loin  et voir que
la focale non plus n'est pas une notion compl\`etement d\'efinie.  Par analogie
avec les mod\`ele simple, tel que~\ref{Proj:Cam:Ideale}, o\`u la focale intervient
explicitement, on peut la d\'efinir comme un rapport d'homomth\'etie entre les
coordonn\'ees image et les coordonn\'ees $\frac{(x,y)}{z}$,
    

%-------------------------------------------------------------------

\subsubsection{On a retrouv\'e le point principal ?}

La m\'ethodes de calibration mise en place sur les
cam\'eras num\'eriques de l'IGN pour mod\'eliser des distorsions quelconques
est la suivante :

\begin{enumerate}
     \item calcul d'un mod\`ele param\`etrique approch\'e (par auto-calibration);
     \item calcul d'un mod\`ele queconque consid\'er\'e comme une petite 
           perturbation corrective du mod\`ele param\'etrique;
     \item calcul d'un mod\`ele param\`etrique  sur le polygone, \`a partir
           de mesure \'eventuellement corrig\'ee du mod\`ele pr\'ec\'edent;
\end{enumerate}

Lorsque l'on attend des distorsions faibles (typiquement  avec les focales $\geq 50mm$),
on passe directement \`a l'\'etape de calibration sur polygone. Que l'on commence \`a 
l'etape $1$, ou directement \`a l\'etape $3$, 

%-------------------------------------------------------------------

\subsection{Param\`etres hors grilles}

\subsubsection{Param\`etres photogram\'etrique "traditionnel"}


\subsubsection{Autres param\`etres}

%-------------------------------------------------------------------

%-------------------------------------------------------------------
%-------------------------------------------------------------------
%-------------------------------------------------------------------

\section{Points Homologues}
\label{Format:Homol}

%-------------------------------------------------------------------
%-------------------------------------------------------------------
%-------------------------------------------------------------------
\section{Fichiers MNT}
\label{Format:MNT}


