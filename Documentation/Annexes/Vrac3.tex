\chapter{Vrac -Bis }

Suite du Vrac. Regroupe plut\^ot des  bribes de documentation interne 
eLiSe .

%-------------------------------------------------------------------
%-------------------------------------------------------------------
%-------------------------------------------------------------------

\section{Utilisation des foncteur compil\'es}

On retrace ici le (long \dots) chemin qui conduit de l'utilisation
finale d'un foncteur jusqu'\`a la modification d'un syst\`eme (L2 ou L1).
Il s'agit d'extrait de code avec plein de coupure.

\subsection {Fili\`ere non indexee}

{\small
\begin{verbatim}
void  cEqObsRotVect::AddObservation (...)
{
    mN2.SetEtat(aDir2/euclid(aDir2)); ....
    mSet.AddEqFonctToSys(mFoncEqResidu,aPds,WithD2);

}

REAL cSetEqFormelles::AddEqFonctToSys ( const tContFcteur & aCont ...)
{
   VAddEqFonctToSys(aFonct,aPds,WithDerSec);
}

const std::vector<REAL> & cSetEqFormelles::VAddEqFonctToSys (cElCompiledFonc * aFonct)
{
    aFonct->SetCoordCur(mAlloc.ValsVar()); // Met les val courrantes
    aFonct->SetValDer();   // Fait le calcul
    aFonct->AddDevLimOrd1ToSysSurRes(*mSys,aPds,true); // Transfert a la matrice
}

void cElCompiledFonc::AddDevLimOrd1ToSysSurRes
     (cGenSysSurResol & aSys,REAL aPds,bool EnPtsCur)
{
     AddContrainteEqSSR(false,aPds,aSys,EnPtsCur);
}

// Peut etre appelee par AddDevLimOrd1ToSysSurRe, mais aussi pour mettre
// des contraintes
void cElCompiledFonc::AddContrainteEqSSR
     (bool contr,REAL aPds,cGenSysSurResol & aSys,bool EnPtsCur)
{
   // met les valeur des derivees dans mRealDer qui, apparament 
   // est un tableau ayant la meme taille que le systeme global !
   for ( aD ...)
       aSys.GSSR_AddNewEquation(aPds,&(mRealDer[aD][0]),aB);
}

void cGenSysSurResol::GSSR_AddNewEquation(REAL aPds,REAL * aL,REAL aB)
{
     V_GSSR_AddNewEquation(aPds,aL,aB);
}

  // Ensuite il y a differente variante, en L2 ca donne
void L2SysSurResol::V_GSSR_AddNewEquation(REAL aPds,REAL * aCoeff,REAL aB)
{
    AddEquation(aPds,aCoeff,aB);
}

void L2SysSurResol::AddEquation(REAL aPds,REAL * aCoeff,REAL aB)
{
    // selectione (! enfin ...) les coeffs non nuls dans une table d'indexe
    L2SysSurResol::V_GSSR_AddNewEquation_Indexe(VInd,aPds,&VALS[0],aB);
}

// En L1 ca donne simplement 

void SystLinSurResolu::V_GSSR_AddNewEquation(REAL aPds,REAL * aCoeff,REAL aB)
{
     PushEquation(aCoeff,aB,aPds);
}


\end{verbatim}
}


Cette fili\`ere est tr\`es lourde, cependant elle est indispensable en $L_1$, puisque
chaque \'equation doit \^etre repr\'ent\'ee explicitement et de mani\`ere compl\`ete.


\subsection {Fili\`ere indexee}

Elle est utilisee par les foncteur ayant un tres grand nombre d'inconnue
et (par consequence) des matrices creuses. Par exemple dans
{\tt src/photogram/phgr\_cGridIncImageMnt.cpp}?


{\small
\begin{verbatim}
void cGridIncImageMnt::OneStepRegulD2 (REAL aPds)
{
    ..
     pSetIncs->AddEqIndexeToSys(pRegD2,aPds/4,mIncs);
}

const std::vector<REAL> & cSetEqFormelles::AddEqIndexeToSys
   ( cElCompiledFonc * aFonct, REAL aPds, const std::vector<INT>  & aVInd)
{
    aFonct->SVD_And_AddEqSysSurResol(... ValVar() ...);
}

void cElCompiledFonc::SVD_And_AddEqSysSurResol
     (
           const std::vector<INT> & aVInd,
           REAL aPds,
           REAL *       Pts,
           cGenSysSurResol & aSys,
           bool EnPtsCur
      )
{
   // Si oui le systeme aSys sait utiliser des directement des
   // matrice
      bool UseMat = aSys.GSSR_UseEqMatIndexee();

    ...
      if (UseMat)
      {
           double ** aDM = aMat.data();
	   double  * aDF = aFLin.data();

           aSys.GSSR_EqMatIndexee(aVInd,aPds,aDM,aDF,aCste);
      }
      else
      {
          aSys.GSSR_AddNewEquation_Indexe(aVInd,aPds,&(mCompDer[aD][0]),aB);;
      }
}

  GSSR_UseEqMatIndexee :
     - vrai pour L2SysSurResol (matrice pleine)
     - faux pour cGenSysSurResol (donc pour L1 qui en derive sans redef)
     - vrai pour cFormQuadCreuse  (donc cElMatCreuseMap et cElMatCreuseStrFixe)

Par ailleur GSSR_AddNewEquation_Indexe , plante en L1.
\end{verbatim}
}

Optimisation a faire pour les grandes aeros :

\begin{itemize}
   \item   switcher, sans doute a partir de AddEqFonctToSys, vers le
   mode indexe;
    \item sauf si le system ne le supporte pas ! Donc par ex L1;
\end{itemize}


\section{Communication avec les syst\`emes sur-contraints}

\subsection{Convention d'\'ecritures}

Soit un ensemble d'\'equations d'observations :

\begin{equation}
   F^j(X)- O^j = 0;
\end{equation}

Un foncteur correpond a une lin\'erisation de  ces equations.  On 
regroupe plusieur \'equations en un seul foncteur, lorsqu'il y a une
synergie permettant de mettre en commun les calcul (cas courant, projection
image en $x$ et $y$).  A VOIR : PEUT ON GAGNER DU TEMPS EN FACTORISANT TOUTE
LES OBSERVARION D'UNE PROJECTION SUR UNE MEME CAMERA.

Si on linéarise l'\'equation en $X_0$, on a :

\begin{equation}
   F^j(X_0) + \sum{\frac{\partial F^j}{\partial X_k} \delta_k} - O^j = 0;
\end{equation}

En notant :

\begin{equation}
      D^j_k= \frac{\partial F^j}{\partial X_k} = {\tt mCompDer[j][k]} 
\end{equation}
\begin{equation}
      V^j =  F^j(X_0) - O^j = {\tt mVal[j]}
\end{equation}

On a l'\'equation lin\'eaire  dont les para\`etres sont calcul\'es dans le foncteur:

\begin{equation}
    \sum{ D^j_k \delta_k}  + V^j = 0;
\end{equation}


Dans l'interface des syst\`emes lin\'eaire, le m\'ethod {\tt GSSR\_AddNewEquation(P,A,B)}
correspond \`a l'observation :

\begin{equation}
    (\sum{ A_k X_k} = B ) * P;
\end{equation}


C'est pour cela que dans {\tt AddContrainteEqSSR} on a $B=-V$.

A priori, dans toutes ces \'equations, l'inconnue est directement un delta 
par rapport au point courant (param\`etre {\tt EnPtsCur} vaut {\tt true}).
Si ce n'\'etait pas le cas,  les inconnues seraient $x_k$ et les  \'equation
deviendraient :

\begin{equation}
   F^j(X_0) + \sum{\frac{\partial F^j}{\partial X_k} (x_k-{X_0}_k} - O^j = 0;
\end{equation}

Soit  :

\begin{equation}
    \sum{D^j_k x_k} =   \sum{D^j_k {X_0}_k} -V;
\end{equation}

D'o\`u le code extrait de {\tt cElCompiledFonc::AddContrainteEqSSR} :

\begin{verbatim}
   REAL aB = -mVal[aD];
    ...
         aB +=  mCompDer[aD][aIC] * mCompCoord[aIC]
\end{verbatim}


\subsection{UseEqMatIndexee}


Si le syst\`eme d'\'equation l'autorise, on va \'ecrire directemnt un forme
quadratique . Chaque \'equation :

\begin{equation}
    P (^t L  X =  B)  
\end{equation}

Appporte une contribution :

\begin{equation}
    P (^t L  X -  B) ^2  = P (^t X (^tLL)X  -2B^tL X + B^2)
\end{equation}


