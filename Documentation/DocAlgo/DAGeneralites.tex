\chapter{G\'en\'eralit\'es}

%---------------------------------------------

\section{Notations G\'eom\'etriques}

On est dans la contexte ou l'on dispose de $N$ images 
not\'ees $Im_k(u,v)_{k\in [1,N]}$ .

On adopte les notations suivantes :

\begin{itemize}
   \item \ETer,  avec  $\ETer = \RR ^2$, l'espace "terrain", au sens assez large, 
         c'est l'espace dans lequel est restitu\'e le "MNT"; 

   \item \EPx, avec  $\EPx = \RR$ ou avec  $\EPx = \RR ^2$ suivant les cas, 
         l'espace des parallaxes; on notera \DimPx la dimension de l'espace
         des parallaxes;

   \item \EIm l'espace de la k\EME image;
\end{itemize}

En interne, MICMAC voit la g\'eom\'etrie  \`a travers une seule
fonction $\pi_k$ par image:

\begin{equation}
   \pi_k :   \ETer \otimes \EPx \rightarrow \EIm
\end{equation}
\begin{equation}
              (x,y,p_x)  \stackrel{\pi_k}{\rightarrow}  (u,v)
\end{equation}

Avec $p_x=z$ lorsque $\EPx = \RR$ et   $p_x=(p_{x_1},p_{x_2})$ 
lorsque $\EPx = \RR^2$. A  $p_x$ fix\'ee les fonctions $\pi_k$,
 consid\'er\'ees comme des fonction de \ETer dans  \EIm, 
sont injectives sur leur domaine d'int\'er\^et et on note
abusivement  $\pi_k^{-1}$ la fonction "inverse" d\'efinie par  :

\begin{equation}
   \pi_k^{-1} :   \EIm \otimes \EPx \rightarrow \ETer
\end{equation}

\begin{equation}
   \pi_k^{-1}(\pi_k(x,y,p_x),p_x) = (x,y) 
\end{equation}


Le r\'esultat de la mise en correspondance est un fonction  \FPx de
\ETer dans \EPx:

\begin{equation}
   \FPx :   \ETer  \rightarrow  \EPx 
\end{equation}

On note ${\FPx}^d,d\in [1,\DimPx]$ les diff\'erentes composantes de\FPx.


    % - - - - - - - - - - - - - - - - - - - - - - - - - - - - - - - -

\section{Discr\'etisation et quantification}

\label{Disc:Quant}

Dans la tr\`es grande majorit\'e des cas,
MicMac  fonctionne par discr\'etisation 
des espaces \ETer et \EPx (par opposition par exemple aux
approches variationnelles). A chaque \'etape
de l'approche multi-r\'esolution (voir~\ref{Appr:Multi:Resol}), sont 
d\'efinis des pas de discr\'etisation  $\Delta^{xy}$ pour \ETer , 
et $\Delta^{px}_k, k \in [1,\DimPx]$ pour  \EPx.

A une \'etape donn\'ee, on se donne en plus un pas de
discr\'etisation de l'espace image \DeltaI; dans
la  majorit\'e des cas \DeltaI est choisi
de mani\`ere \`a ce que la r\'esolution image obtenue soit
\'egale \`a  $\Delta^{xy}$,
mais il peut y avoir des exceptions. Pour chaque pas $\DeltaI$
utilis\'e MicMac calcule des images sous-r\'esolues d'un facteur
$\DeltaI$, on note $Im_k(\frac{}{\DeltaI})$ ces images et 
$\EIm*\DeltaI$ leur espace associ\'e.


Pour des pas de discr\'etisation donn\'es,
on d\'efini $p_k$ la version discr\`ete de $\pi_k$ par 
\footnote{avec modification imm\'ediate de~\ref{EQ:Disc:Proj} 
lorsque $\DimPx=1$}:

\begin{equation}
   p_k : \ZZ^2 \otimes\ZZ^{\DimPx} \rightarrow \EIm*\DeltaI
\end{equation}

\begin{equation}
   \breve\pi_k(i,j,u,v) = \pi_k(i*\Delta^{xy},j*\Delta^{xy},u*\Delta^{px}_1,v*\Delta^{px}_2)*\DeltaI
\label{EQ:Disc:Proj}
\end{equation}

Pour des pas   $\Delta^{xy}$ et $\Delta^{px}_k$ donn\'es, le travail
du corr\'elateur est de rechercher un tableau $F_{px}$ de $\ZZ^2$ dans
$\ZZ^{\DimPx}$, version discr\`ete de \FPx:

\begin{equation}
   F_{px}(i,j)^k = \frac{\FPx(i*\Delta^{xy},j*\Delta^{xy})}{\Delta^{px}_k}
\end{equation}

    % - - - - - - - - - - - - - - - - - - - - - - - - - - - - - - - -

