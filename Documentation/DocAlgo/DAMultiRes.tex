%---------------------------------------------

\chapter{Approche multi-r\'esolution}

\label{CHAP:Multi:Resol}

\section{Motivations}



\section{Mod\`ele pr\'edictif}


\label{Appr:Multi:Resol}

Afin de limiter la combinatoire MicMac utilise une approche 
multi-r\'esolution o\`u, \`a chaque \'etape, la solution de
l'\'etape pr\'ec\'edente est utilis\'ee comme un pr\'edicteur
autour duquel l'exploration se fera dans un voisinage limit\'e.

Notons de mani\`ere g\'en\'erique $X'$ ($F_{px}'^k$,$\Delta'^{xy}$ \dots) 
la valeur de $X$ \`a l'\'etape pr\'ec\'edente. 


\begin{equation}
   \rho ^{xy} = \frac{\Delta'^{xy}}{\Delta^{xy}} ;
   \rho ^{px}_k = \frac{\Delta'^{px}_k}{\Delta^{px}_k}
\end{equation}

On d\'efini le pr\'edicteur $Pred_{px}$ par :

\begin{equation}
   Pred_{px}(i,j)^k = F_{px}'^k(\frac{i,j}{\rho^{xy}}) * \rho^{px}_k
\end{equation}

Ensuite on se donne des valeurs $\delta_{A}^k$ et  $\delta_{P}^k$ 
repr\'esentant l'incertitude "altim\'etrique" et planim\'etrique.
Au sens de la morpho-math, notons $\oplus$ la dilatation et $\ominus$ l'\'erosion .
On d\'efinit alors ${F^+_{px}}^k$ et ${F^-_{px}}^k$ 
les "nappes englobantes" qui encadrent l'intervale de
recherche de $ F_{px}^k$ \`a l'\'etape courante par :


\begin{equation}
   {F^-_{px}}^k =  Pred_{px}^k \ominus \delta_{P}^k - \delta_{A}^k
\end{equation}

\begin{equation}
   {F^+_{px}}^k =  Pred_{px}^k \oplus \delta_{P}^k + \delta_{A}^k
\end{equation}

\section{Noyaux utilis\'es pour la sous-r\'esolution}


