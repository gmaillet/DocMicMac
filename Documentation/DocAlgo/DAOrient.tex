\chapter{Algorithm on orientation}

    % - - - - - - - - - - - - - - - - - - - - - - - - - - - - - - - -
    % - - - - - - - - - - - - - - - - - - - - - - - - - - - - - - - -
    % - - - - - - - - - - - - - - - - - - - - - - - - - - - - - - - -

\section{Tomasi-Kanabe}

This section give a brief description of the Tomasi-Kanabe algorithm
that is (will be) used in Martini as as suplementary test of
orientation for long focal camera. See~\cite{TomKan} , the original paper,
for more details.


Let $C_1,C_2, \dots C_N$ be $N$ camera with infinite focal lentgh, classically
the projection fonction are otho centric projection, and there is a strict redundancy
between principal-point/origin and  focal/origin. The projection $\pi_k$ fonction on camera $C_k$
can be given by :

\begin{equation}
   \pi_k(P) =  (A_k,B_k)   + S_k * (i_k . P , j_k.P) \label{TomKanProj}
\end{equation}

Where :

\begin{itemize}
   \item  the translation $(A_k,B_k)$ represent the "principal point/origin on X,Y"
   \item  the scaling factor $S_k$ represent the "focal legnth/origin on Z"
   \item  $i_k$ and $j_k$ are the axes of the camera they are the two first vector
          of an othonorme base $||i_k|| = ||j_k|| = 1$ , $i_k . j_k = 0 $ ,
          we note $k_k = i_k \wedge j_k$.
\end{itemize}


Having $M$ point seen in the $N$ camera  (we know the projection of these point but not their $3$ coordinates) we note :

\begin{itemize}
    \item $P_m , m \in[1,M]$  the unknown $3d$ coordinates of the $m^{th}$ points;
    \item $(u_{n,m},v_{n,m}), m \in[1,M], n \in [1,N] $   the $2d$ coordinates of projection
          of $m^{th}$ points in $n^{th}$ camera;
    \item we have :
\end{itemize}

\begin{equation}
    \pi_n(P_m) = (u_{n,m},v_{n,m})
\end{equation}
As the repair in which we want to express $P_k$ is arbirtrary, we decide to  set the 
origin at the center of mass; we then have   :


\begin{equation}
   \sum_{m=1}^{M} P_m = 0 \label{TomKanCDM}
\end{equation}

As equation~\ref{TomKanProj} is linear : 


\begin{equation}
     \forall n : \sum_{m=1}^{M} \pi_n(P_m)  
     =  (A_n,B_n) * M + S *(i_n. \sum_{m=1}^{M} P_m,j_n. \sum_{m=1}^{M} P_m)
     = (A_n,B_n) * M
     = \sum_{m=1}^{M} (u_{n,m},v_{n,m})
\end{equation}

So we see that $A_n,B_n$ can easily be computed by  :


\begin{equation}
   A_n = \frac{\sum_{m=1}^{M}u_{n,m}}{M} ;
   B_n = \frac{\sum_{m=1}^{M}v_{n,m}}{M} ; \label{TomKanAB}
\end{equation}

To allegeate notation as in~\cite{TomKan}, we suppose we begin normalise $u_{n,m}$ and $v_{n,m}$ 
by substracting the center of mass, and equation~\ref{TomKanAB} resume to $A_n=B_n=0$.


\begin{equation}
   (u_{n,m},v_{n,m}) =  \pi_n(P_m) =   S_n * (i_n . P_m, j_n.P_m) \label{TomKanProj2}
\end{equation}


We note $M^u_v$ the matrix :


\begin{equation}
M^u_v=
\left( \begin{array}{cccc} 
             u_{1,1} & u_{2,1}  & \dots & u_{N,1} \\ 
             u_{1,2} & u_{2,2}  & \dots & u_{N,2} \\
              \dots &  \dots  & \dots &   \dots   \\
             u_{1,M} & u_{2,M}  & \dots & u_{N,M} \\
             v_{1,1} & v_{2,1}  & \dots & v_{N,1} \\ 
              \dots &  \dots  & \dots &   \dots   \\
             v_{1,M} & v_{2,M}  & \dots & v_{N,M} 
        \end{array} 
\right)
\end{equation}


\begin{equation}
   M^u_v= \left( \begin{array}{ccc} 
             S_1*i^x_1 & S_1*i^y_1 & S_1*i^z_1 \\
             S_2*i^x_2 & S_2*i^y_2 & S_2*i^z_2 \\
             \dots & \dots & \dots \\
             S_M*i^x_M & S_M*i^y_M & S_M*i^z_M \\
             S_1*j^x_1 & S_1* j^y_1 & S_1* j^z_1 \\
             \dots & \dots & \dots \\
             S_M*j^x_M & S_M*j^y_M & S_M*j^z_M 
        \end{array} 
\right)
       *
        \left( \begin{array}{cccc} 
             P^x_1 & P^x_2 &  \dots & P^x_N \\
             P^y_1 & P^y_2 &  \dots & P^y_N \\
             P^z_1 & P^z_2 &  \dots & P^z_N \
        \end{array} 
        \right)
     = M^i_j * M^P
    \label{TomKanFactMUV}
\end{equation}



We can still
suppose that $M^i_j$ is a square Matrix by padding is with $0$ column of lines.
Padding it with $0$ column correspond to add the projection of a point of coordinate
 $(0,0,0)$, padding with $0$ line correspond to add a camera with scaling factor $0$.
Using a singular value decomposition $M^u_v$ can be written :


\begin{equation}
   M^u_v=   ^t R_1 \Delta R_2
\end{equation}

With  :

\begin{itemize}
   \item  $R_1 ^t R1 = Id$  
   \item  $R_2 ^t R2 = Id$  
   \item  $\Delta$ diagonal matrix.
\end{itemize}

As $M^i_j$ (or $M^p$) is of rank $3$, $M^u_v$ is also of rank $3$.  So  theoretically, $\Delta$  has only $3$
non zero value, due to numerical erroe in $u_{n,m},v_{n,m}$ and in computation, it may be different; however,
 $\Delta$  can be best approximated by the $3*3$ matrix corresponing to the highest eigen value. Suppressing
the corresponding row of $R_1$ and column of $R_2$ we obtain a
decomposition of $M^u_v$ in the form :

\begin{equation}
   M^u_v=   r_1  \delta r_2
\end{equation}

Witj $\delta$ a $3*3$ matrix, $r_1$ a $N*3$ matrix and $r_2$ a $3*N$ matrix, seting arbirtarily 
$r'_2 =  \delta r_2$, we can write  :

\begin{equation}
   M^u_v=   r_1   r'_2
\end{equation}


Also this formula is close to~\ref{TomKanFactMUV}, we cannot identify $M^i_j=r_1$ and $ M^P=r'_2$
because the decomposition is far from unique, in fact for any $3*3$ matrix $Q$, we can write :

\begin{equation}
   M^u_v=   (r_1 Q) ( Q^{-1}  r'_2)
\end{equation}

To determine $Q$ we are now using the fact that  the $(i_m,j_m)$ are orthonormal, we set 


\begin{equation}
   M^u_v=   ^t \left( \begin{array}{ccccccc} I_1, I_2 \dots & I_M & J_1 & \dots & J_M 
        \end{array} 
\right)
\end{equation}


And we can write the metric constraints :

\begin{itemize}
   \item  $  ^t I_1 Q ^t Q I_1 = 1 $
   \item  $  ^t J_1 Q ^t Q J_1 = 1 $
   \item  $\forall n :  ^t I_n Q ^t Q J_n = 0 $
   \item  $\forall n \neq 1 :  ^t I_n Q ^t Q I_n =  ^t J_n Q ^t Q J_n   $
\end{itemize}

As $Q ^t Q$ is a symetric matrix, we can estimate it  by least mean square using the
above linear equation. Knowing $W= Q ^t Q$, it's easy to recover $Q$, we do a singular
value decomposition of $W$ :

\begin{equation}
    W =  ^t R D R
\end{equation}

We can compute a possible value of $Q$ by

\begin{equation}
    Q  =  ^t R  \sqrt{D}
\end{equation}


This value is not unique, because for any rotation $r$, $Q' = Qr$ will be also a solution. This
non unicity simply reflect the fact that the orientation of the camera can only be computed
up to a global rotation.




