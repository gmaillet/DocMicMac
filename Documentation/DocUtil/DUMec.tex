
\chapter{Mise en correspondance}



%---------------------------------------------
\section{G\'en\'eralit\'e}

\emph{N.B.} : MEC= Mise En Correspondance.

\subsection{Organisation}

\label{ORG:MEC}

La section {\tt <Section\_MEC>} est constitu\'ee de quelques
tags globaux \`a la mise en correspondance puis d'une liste
de descripteurs d'\'etapes {\tt <EtapeMEC>}.

Afin de conserver un maximum de souplesse,
MicMac \'etant aussi un outil de R\&D , la grande
majorit\'e des tags se  retrouvent en fait dans
les descripteurs d'\'etapes ce qui permet, au cas o\`u,
d'avoir un contr\^ole \'etape par \'etape sur les valeur.
La section~\ref{TAG:MEC:GLOB} est consacr\'ee aux quelques
tags globaux; les sections suivantes sont consacr\'ees  aux
descendants de {\tt <EtapeMEC>}.

\subsection{Mode diff\'erentiel, valeurs par d\'efaut}

La liste {\tt <EtapeMEC  Nb="+" DeltaPrec="1">} est en mode
diff\'erentiel (voir~\ref{DUG:Mode:Dif}), ce qui permet
de concilier un fonctionnement simple dans le cas g\'en\'eral
avec un r\'eglage fin pour des utilisations plus pointues.
Le premier \'el\'ement de cette liste d'\'etapes de mise
en correspondance ne sera pas ex\'ecut\'e, il permet de
r\'egler les valeur par d\'efaut qui seront commune \`a
la majorit\'e des \'etapes.  Pous s'assurer qu'il n'y a
pas d'ambiguit\'e sur ce point, MicMac impose que la valeur
de r\'esolution (tag {\tt DeZoom})  de ce premier \'el\'ement
soit $-1$ (valeur impossible).

Le mode diff\'erentiel complique un peu la gestion des valeur
par d\'efaut. Il existe un seul fils de {\tt <EtapeMEC>} obligatoire
\`a toutes le \'etapes : {\tt DeZoom}. Les autres sont d'arit\'e {\tt ?},
ce qui peut en fait correspondre \`a plusieurs r\'ealit\'e diff\'erentes :

\begin{itemize}
    \item les tags qui sont logiquement obligatoires mais qui, du fait
          du mode diff\'erentiel, peuvent n'\^etre sp\'ecifi\'es qu`\`a
          la premi\`ere \'etape;

    \item les tags qui sont logiquement facultatifs, pour des raisons
          internes de programmation il n'est pas possible d'utiliser
          le m\'ecanisme de sp\'ecification par l'attribut {\tt Def}
          (voir~\ref{Val:Def}) ; la valeur par d\'efaut est g\'er\'ee
          dans le code MicMac; sauf oubli, elle est sp\'ecifi\'ee dans
          cette doc et en commentaire.

    % \item les tags qui sont contextuellement obligatoires (

\end{itemize}



\subsection{Equivalence de noms}

Comme en~\ref{SEC:Interv:Px}, les m\^emes param\`etres ont
des noms diff\'erents suivant la dimension de  parallaxe.
Le tableau~\ref{TAB:EQUIV:Mec} r\'esume les \'equivalences.

\begin{figure}
\begin{tabular} {  l c c c c } \\  \hline
               & Z & Px1 & Px2 & Nom Math\\ \hline
       Pas & ZPas & Px1Pas & Px2Pas &  $r_k$\\ \hline
       R\'egularisation & ZRegul & Px1Regul & Px2Regul & $\alpha_k$ \\ \hline
       Dilatation-Px & ZDilatAlti & Px1DilatAlti & Px2DilatAlti &\\ \hline
       Dilatation-XY & ZDilatPlani & Px1DilatPlani & Px2DilatPlani &\\ \hline
       Redressement & ZRedrPx & Px1RedrPx & Px2RedrPx &\\ \hline
       Dequantif Redr & ZDeqRedr & Px1DeqRedr & Px2DeqRedr &\\ \hline
\end{tabular}
\caption {Equivalence de noms}
\label{TAB:EQUIV:Mec}
\end{figure}



%---------------------------------------------
%---------------------------------------------
%---------------------------------------------

\section{Param\`etres globaux}

\label{TAG:MEC:GLOB}

\begin{verbatim}
     <Section_MEC Nb="1">
        <ProportionClipMEC/>
        <NbMinImagesVisibles/>
        <DefCorrelation/>
        <EpsilonCorrelation/>
        ...
     </Section_MEC>
\end{verbatim}


    %---------------------------------------------

\subsection{Clip de la zone de MEC}

\begin{verbatim}
       <ProportionClipMEC Type="Box2dr" Nb="?"> </ProportionClipMEC>
\end{verbatim}

Cette valeur permet de restreindre la zone sur laquelle est
r\'eellement effectu\'ee la mise en correspondance . Cette
restriction ne porte que sur le calcul et ne
modifie pas l'emprise terrain du chantier; par exemple, si on lance
plusieurs fois MicMac  avec  le m\^eme fichier de
param\`etres, sauf cette valeur {\tt <ProportionClipMEC>},
on ne verra pas de probl\`eme de raccord entre les
diff\'erents morceaux calcul\'es.

Comme son nom l'indique, il s'agit d'une proportion; la bo\^ite
{\tt [0.0 0.0 1.0 1.0]} correspond \`a l'ensemble du chantier;
lorsque  {\tt <ProportionClipMEC>} est omis (arit\'e={\tt ?}),
la valeur par d\'efaut est naturellement {\tt [0.0 0.0 1.0 1.0]}.

{\tt <ProportionClipMEC>} est utile en phase mise au point
lorsque l'on veut tester rapidement un jeu de param\`etres
sans cr\'eer un chantier \emph{ad hoc}.


    %---------------------------------------------

\subsection{Calcul du masque de MEC}

\label{MEC:Masq}

La section planim\'etrie(\ref{Autresec:Planimetrie}) permet
de sp\'ecifier l'emprise (bo\^ite englobante) terrain  et un
\'eventuel masque utilisateur (~\ref{Masq:Ter:Util}).
Les tags de cette section permettent de param\'etrer le
masque calcul\'e automatiquement par MicMac.


\subsubsection{Nombre minimal d'images}

\begin{verbatim}
     <NbMinImagesVisibles Nb="?" Type="int" Def="2"> </NbMinImagesVisibles>
\end{verbatim}



MicMac calcul un masque qui se superpose \`a l'\'eventuel
masque terrain sp\'ecifi\'e par l'utilisateur.
Soit $Nb^{im}$ la valeur de ce param\`etre;
ce masque est d\'efini comme l'ensemble des points
terrain qui sont vus d'au moins $Nb^{im}$ avec l'hypoth\`ese
de parallaxe moyenne. La formule utilis\'ee par MicMac
est donc :


\begin{equation}
        \{p\in\ETer / card\{k\in[1 \, N]/ \pi_{k}(p,\PxMoy) \in \EIm\} \geq Nb^{im}\}
\end{equation}



    %---------------------------------------------
\subsection{Divers}
    % - - - - - - - - - - - - - - - - -
\subsubsection{Valeur par d\'efaut de l'attache aux donn\'ees}

\begin{verbatim}
       <DefCorrelation Nb="?" Type="double" Def="0.0"> </DefCorrelation>
\end{verbatim}

Lors du calcul d'un coefficient de ressemblance inter-image,
diff\'erentes raisons peuvent emp\^echer MicMac de calculer un
coefficient de corr\'elation (par exemple  parce que, compte-tenu
des masques image et terrain, il n'y a pas au moins deux images visibles).
Dans ces cas MicMac renvoye la valeur de {\tt <DefCorrelation>}.

    % - - - - - - - - - - - - - - - - -

\subsubsection{Corr\'elation d\'eg\'en\'er\'ees}

\begin{verbatim}
       <EpsilonCorrelation Nb="?" Type="double" Def="1e-10"> </EpsilonCorrelation>
\end{verbatim}

L'estimation du coefficient de corr\'elation  est d'autant
plus bruit\'ee  que l'ecart-type est faible (et ind\'efini
quand il est nul).  Pour \'eviter tout probl\`eme d'exception
arithm\'etique, MicMac utilise  {\tt Max(Variance,<EpsilonCorrelation>)}
dans le calcul du coefficient de corr\'elation.

%---------------------------------------------
%---------------------------------------------
%---------------------------------------------

\section{G\'eom\'etrie et  nappes englobantes}


     %---------------------------------------------
\subsection{Gestion des r\'esolutions}

    % - - - - - - - - - - - - - - - - -
\subsubsection{R\'esolution terrain}
\begin{verbatim}
    <DeZoom  Nb="1" Type="int">                    </DeZoom>
\end{verbatim}

Cette valeur fixe le ratio de r\'esolution terrain de l'\'etape;
il s'agit du seul param\`etre obligatoire \`a chaque \'etape .
Le pas de r\'esolution terrain de l'\'etape, le $\Delta^{xy}$
 d\'efini en~\ref{Disc:Quant:fr},
sera \'egal \`a {\tt DeZoom * ResolutionTerrain} (o\`u
{\tt ResolutionTerrain} est la valeur d\'efinie en \ref{Autresec:Resol:Terr}).
 Les contraintes sont les suivantes :


\begin{itemize}
   \item  la valeur de la premi\`ere \'etape, qui n'est pas ex\'ecut\'ee,
          doit valoir conventionnellement $-1$ (voir~\ref{ORG:MEC});

   \item  les valeurs suivantes doivent \^etre des puissances de $2$
          et $\geq1$ ;

   \item  ces valeurs doivent \^etre d\'ecroissantes (de mani\`ere non
          stricte)  avec un ratio au maximum de $2$ entre valeurs
          cons\'ecutives;
          autrement  dit, si \`a une \'etape la valeur est  $2^n, n\in \NN$ ,
          \`a l'\'etape suivante cette valeur est  soit $2^n$, soit $2^{n-1}$.
\end{itemize}

    % - - - - - - - - - - - - - - - - -
\subsubsection{R\'esolution image}
\emph{Valeur par d\'efaut $1.0$}
\begin{verbatim}
            <RatioDeZoomImage Nb="?" Type="double"> </RatioDeZoomImage>
\end{verbatim}

En g\'en\'eral, pour des questions de coh\'erence d'\'echantillonage,
on souhaite avoir des r\'esolutions comparables pour
le terrain et l'image.
C'est ainsi que pour les g\'eom\'etries
terrain, la r\'esolution terrain propos\'ee par d\'efaut par
MicMac est \'egale \`a la r\'esolution estim\'ee
 de la prise de vue (voir~\ref{Autresec:Resol:Terr}).
En cons\'equence,  pour maintenir cette coh\'erence \`a chaque \'etape,
le ratio de r\'esolution image  de l'\'etape
(le \DeltaI de~\ref{Disc:Quant:fr} ) est  lui aussi, en g\'en\'eral,
\'egal  au ratio Terrain.

Cependant cette \'egalit\'e des ratio, n'est pas impos\'ee et
MicMac permet de le contr\^oler avec le param\'etre
{\tt RatioDeZoomImage} de la mani\`ere suivante :


\begin{itemize}
     \item le \DeltaI vaut {\tt RatioDeZoomImage * DeZoom};
     \item  par d\'efaut {\tt RatioDeZoomImage} vaut $1$
            (donc assure l'\'egalit\'e des ratios);
\end{itemize}

Le seul cas coh\'erent r\'epertori\'e d'utilisation de  {\tt RatioDeZoomImage}
correspond au contexte suivant :

\begin{itemize}
   \item  on dispose d'un prise de vue \`a une r\'esolution donn\'ee,
          disons $20 cm$ pour fixer les choses;

    \item on esp\`ere (peut \^etre \`a cause du multi-vues) pouvoir
          corr\'eler les images avec un pas terrain de $10 cm$;

     \item si on laissait les options par d\'efaut de MicMac on
           aurait  :
     \begin{itemize}
           \item  \`a {\tt DeZoom=1}, des images \`a $20 cm$
              pour $10 cm$ terrain (\c{c}a c'est in\'evitable);

           \item  \`a {\tt DeZoom=2}, des images \`a $40 cm$
              pour $20 cm$ terrain ;

           \item  \`a {\tt DeZoom=4}, des images \`a $80 cm$
              pour $40 cm$ terrain ;
           \item  \dots
     \end{itemize}

\end{itemize}

Or, tant que  {\tt DeZoom$\geq2$}, on dispose d'images ayant une r\'esolution
adapt\'ee au terrain, et on n'a pas int\'er\^et \`a utiliser des images
moins bien r\'esolues.  Pour utiliser les images \`a la bonne
r\'esolution, on pourra fixer {\tt <RatioDeZoomImage>} \`a $0.5$
dans la prem\`iere \'etape .Cette valeur est ensuite transmise
aux suivantes via le mode diff\'erentiel, on aura simplement soin
de r\'etablir {\tt <RatioDeZoomImage>}  \`a $1$  lorsque {\tt DeZoom=1}.



     %---------------------------------------------
\subsection{Pas de quantification}

\label{Pas:Quantif}

\emph{Pas de valeur par d\'efaut}
\begin{verbatim}
      <ZPas  Nb="?"  Type="double">     </ZPas>
      <Px1Pas  Nb="?"  Type="double">     </Px1Pas>
      <Px2Pas  Nb="?"  Type="double">     </Px2Pas>
\end{verbatim}

Ces param\`etres  permettent de fixer les $\Delta^{px}_k$
d\'ecrits en~\ref{Disc:Quant:fr}. Ils expriment un ratio;
soit $r_k$ leur valeur, la formule utilis\'ee pour calculer
$\Delta^{px}_k$ est la suivante :


\begin{equation}
   \Delta^{px}_k = r_k * \Delta^{xy} * \rho^G_k
\end{equation}


Commentons :

\begin{itemize}
   \item $\rho^G_k$ est un coefficient , calcul\'e par MicMac,
         qui ne d\'epend que de la g\'eom\'etrie et
         sur lequel nous allons revenir (il vaut souvent $1$);

   \item  couramment, les valeurs fix\'ees \`a la premi\`ere \'etape
          peuvent \^etre conserv\'ees tout au long du calcul
          (puisque $\Delta^{px}_k$ est proportionnel \`a $\Delta^{xy}$
           et donc \`a {\tt DeZoom});

   \item le coefficient $\rho^G_k$ assure que, en g\'en\'eral $0.5 \leq r_k \leq 1.0$,
         est une plage de valeur "raisonnable" pour d\'ebuter un essai par
         tatonnement.
\end{itemize}

Le coefficient $\rho^G_k$ est calcul\'e
en fonction de la g\'eom\'etrie de restistution
de la mani\`ere suivante :



\begin{itemize}
   \item  $\rho^G_k=1$  pour {\tt eGeomMNTCarto, eGeomMNTEuclid, eGeomPxBiDim};

   \item  pour {\tt eGeomMNTFaisceauIm1ZTerrain\_Px?D}, $\rho^G_1$ est la r\'esolution
          terrain  de la g\'eom\'etrie intrins\`eque ; $\rho^G_2=1$ (si utile);

   \item  pour {\tt eGeomMNTFaisceauIm1PrCh\_Px?D}, s'il y a deux images $\rho^G_1$
          est calcul\'e \`a partir du $\frac{B}{H}$ pour qu'un
          \'ecart de $1$ en parallaxe corresponde \`a peu pr\`es \`a un \'ecart de
           $1$ pixel sur les images (la correspondance \'etant exacte, en cas de vues
           parall\`eles, gr\^ace \`a la dynamique en $\frac{1}{Z}$);
           avec $N$ image, c'est une moyenne de cette formule qui est utilis\'ee;
           $\rho^G_2=1$ (si utile);

\end{itemize}


     %---------------------------------------------
\subsection{Calcul des nappes englobantes}

\emph{Pas de valeur par d\'efaut}
\begin{verbatim}
            <ZDilatAlti   Nb="?" Type="int">   </ZDilatAlti>
            <ZDilatPlani  Nb="?" Type="int">   </ZDilatPlani>

            <Px1DilatAlti   Nb="?" Type="int">   </Px1DilatAlti>
            <Px1DilatPlani  Nb="?" Type="int">   </Px1DilatPlani>

            <Px2DilatAlti   Nb="?" Type="int">   </Px2DilatAlti>
            <Px2DilatPlani  Nb="?" Type="int">   </Px2DilatPlani>
\end{verbatim}


Ces valeurs fixent le $\delta_{P}^k$ et le $\delta_{A}^k$ d\'efinis
en~\ref{Appr:Multi:Resol}. Ce sont des valeurs enti\`eres qui seront
appliqu\'ees \`a la fonction de parallaxe quantifi\'ee. \emph{Si
l'on change  les pas de quantification d\'efinis en~\ref{Pas:Quantif}
et que l'on veut conserver la m\^eme signification physique \`a la
dilatation, il faut adapter le  $\delta_{A}^k$}.



     %---------------------------------------------
\subsection{Redressement des images}

     %---------------------------------------------
\subsection{Divers}

\subsubsection{Diff\'erentiabilit\'e de la g\'eom\'etrie}

\emph{Valeur par d\'efaut 2}
\begin{verbatim}
    <SzGeomDerivable Nb="?" Type="int"> </SzGeomDerivable>
\end{verbatim}

Pour limiter le volume de calcul, fait l'hypoth\`ese que les
fonctions de projections sont d\'erivables et localement
approximable par leur diff\'erentielle (d\'eriv\'ees secondes
"faible" \`a l'echelle des variations consid\'er\'ees). Typiquement, \'etant
donn\'ee une parallaxe quantifi\'ee $u_0,v_0$ et un point
terrain quantifi\'e $i_0,j_0$ pour estimer les  projection
sur un voisinage de taille $N^b$ de $i_0,j_0$, posons :

\begin{equation}
    P_0(i,j) = p_k(i_0+i,j_0+j,u_0,v_0) \, (i,j) \in [-N^b,+N^b]^2
\end{equation}

MicMac utiliser un sch\'ema classique de diff\'erences finies, pour
fait les calculs \`a partir des estimations de $\pi_k$ sur les
"4-voisins" de $P_0(0,0)$ soit :

\begin{equation}
      V_0 = \frac{P_0(1,0)+P_0(-1,0)+P_0(0,1)+P_0(0,-1)}{4}
\end{equation}
\begin{equation}
      \Delta_x = \frac{P_0(1,0)-P_0(-1,0)}{2} \;
      \Delta_y = \frac{P_0(0,1)-P_0(0,-1)}{2}
\end{equation}

\begin{equation}
        P_0(i,j) \approx  V_0 + i *\Delta_x + j*\Delta_y
\end{equation}

En toute rigueur, il est possible que  ce mode de calcul conduise  \`a
des approximations  inacceptables. {\tt SzGeomDerivable} permet d'indiquer
\`a MicMac la taille du voisinage $N^b$ sur lequel l'approximation
est licite.

%---------------------------------------------
%---------------------------------------------
%---------------------------------------------

%\emph{Valeur par d\'efaut $1.0$}
%\emph{Pas de valeur par d\'efaut}

\section{Autres param\`etres d'entr\'ees}

\subsection{S\'election des images}

\emph{Valeur par d\'efaut : pas de s\'election}

\begin{verbatim}
     <ImageSelecteur Nb="?">
         <ModeExclusion Nb="1" Type="bool"> </ModeExclusion>
         <PatternSel Nb="*" Type="std::string"> </PatternSel>
     </ImageSelecteur>
\end{verbatim}

Parfois, on ne souhaite pas que toutes les images du chantier
soient utilis\'ees \`a toutes les \'etapes. Lorsque {\tt ImageSelecteur}
est sp\'ecifi\'e, il permet de filtrer les images qui seront
utilis\'ees \`a l'\'etape courante. Si {\tt  ModeExclusion}
vaut {\tt true} (resp {\tt false}) on exclut  (resp inclut)
les images dont le nom est appariable sur au moins une des
expressions r\'eguli\`eres (voir~\ref{DU:Regex})  de la liste  {\tt PatternSel}.


\subsection{Interpolation}

\emph{Valeur par d\'efaut {\tt eInterpolBiLin}}

\begin{verbatim}
   <ModeInterpolation Nb="?" Type="eModeInterpolation"> </ModeInterpolation>
   <CoefInterpolationBicubique Nb="?" Type="double" Def="-0.5"> </CoefInterpolationBicubique>
   <TailleFenetreSinusCardinal Nb="?" Type="int" Def="3"> </TailleFenetreSinusCardinal>
\end{verbatim}

Les $\pi_k$ des  points quantifi\'es sont \`a coordonn\'ees r\'eelles;
pour acc\'eder aux radiom\'etries des images en ces points, avec
un minimum de perte d'information, il convient de se donner un
sch\'ema d'interpolation. Le param\`etre {\tt ModeInterpolation}
permet de sp\'ecifier l'interpolateur \`a utiliser ; il est
\`a valeur dans l'\'enum\'eration {\tt eModeInterpolation}:

\begin{itemize}
    \item{\tt eInterpolPPV}  : plus proche voisin;
    \item{\tt eInterpolBiLin}  : bi-lin\'eaire;
    \item{\tt eInterpolBiCub}  : bi-cubique , il existe une famille
         d'interpolateur bicubique d\'efinie par un param\`etre
         (d\'eriv\'ee en $1$ ?), ce param\`etre est control\'e par
         le tag {\tt CoefInterpolationBicubique} (celui utilis\'e par d\'efaut
         est celui qui est restore correctement les signaux  affines);
    \item{\tt eInterpolSinCard}  : sinus cardinal,  l'interpolateur
         \'etant en th\'eorie \`a support infini , le tag
         {\tt TailleFenetreSinusCardinal} permet de choisir sur
         quelle taille de fen\^etre on "l'appodise";
         par TailleFenetreSinusCardinal (par d\'efaut 3);
\end{itemize}

%---------------------------------------------
%---------------------------------------------
%---------------------------------------------

\section{Approche \'energ\'etique}

\subsection{Attache aux donn\'ees}


    % - -  - - - - - - - - - - - - - -
\subsubsection{fen\^etres de corr\'elation}

\emph{Par d\'efaut TypeWCorr=eWInCorrelFixe}

\emph{Pas de valeur par d\'efaut pour SzW.}

\emph{Par d\'efaut SzWInt=SzW.}

\emph{Par d\'efaut SzWy=SzW.}


\begin{verbatim}
    <TypeWCorr Nb="?" Type="eTypeWinCorrel"> </TypeWCorr>
    <SzW    Nb="?"  Type="double">      </SzW>
    <SzWy  Nb="?"  Type="double">      </SzWy>
    <SzWInt Nb="?"  Type="int">      </SzWInt>
\end{verbatim}

On a vu (\ref{MecaGen:FEN:EXPO}) que les fen\^etre de corr\'elation
ne sont pas forc\'ement rectangulaire avec un poids constant. Le
param\`etre {\tt TypeWCorr} permet de r\'egler le choix du type
de fen\^etre, il peut prendre les valeurs suivantes :


\begin{itemize}
   \item  {\tt eWInCorrelFixe} fen\^etre "classique";
   \item  {\tt eWInCorrelExp}  fen\^etre exponentielle;
\end{itemize}


Le param\`etre {\tt SzW} fixe la taille de la vignette de corr\'elation
telle que d\'efinie en~\ref{Vign:Correl}.  Lorsque la fen\^etre choisie
n'est pas un fen\^etre rectangulaire mais, par exemple, une fen\^etre
exponentielle \`a support infini,
ce param\`etre sert en fait \`a fixer par analogie le param\`etre
d'\'echelle : MicMac choisit le param\`etre qui donne \`a la largeur
moyenne du filtre la m\^eme valeur que la fen\^etre classique de taille {\tt SzW}.
Pour les fen\^etre exponentielles, il est tout \`a fait coh\'erent
de lui assigner des valeurs r\'eelles et est donc,  pour  rester g\'en\'eral
de type  {\tt double}.

Les tailles en x et en y de la fen\^etre n'ont pas de raison d'\^etre \'egales:
le  tag {\tt SzWy} permet  d'assigner une valeur diff\'erente \`a la taille en y
(attention, pas encore accessible avec les fen\^etres classiques).

Comme d\'efini en~\ref{Vign:Taille:Un}, pour les fen\^etres classiques,
le param\`etre {\tt SzWInt}, permet \'eventuellement d'avoir une taille
diff\'erente pour la  mesure d'\'ecart (fen\^etre de taille
{\tt SzWInt}) et la normalisation en homoth\'etie-translation
des radiom\'etries (fen\^etre de taille {\tt SzW}).

    % - - - - - - - - - - - - - - - - -
\subsubsection{Multi-corr\'elation}

\label{Multi:Correl:Coef}

\begin{verbatim}
    <AggregCorr Nb="?" Type="eModeAggregCorr"> </AggregCorr>
\end{verbatim}

Ce tag permet de sp\'ecifier comment aggr\'eger les coefficients
de corr\'elation lorsqu'il y a plus de deux images
(voir~\ref{MecaGen:Multi:Correl}). Les valeurs possibles sont :

\begin{itemize}
   \item {\tt eAggregSymetrique} coefficient o\`u toutes les images
         jouent le m\^eme r\^ole (\'equation~\ref{Cor:Im:Sym});

   \item {\tt eAggregIm1Maitre}  coefficient o\`u  la premi\`ere
         image est consid\'er\'ee comme ma\^itresse
         (\'equation~\ref{Cor:Im:Maitre}).
\end{itemize}



    % - - - - - - - - - - - - - - - - -
\subsubsection{Dynamique de corr\'elation}

\begin{verbatim}
    <DynamiqueCorrel Nb="?" Type="eModeDynamiqueCorrel"> </DynamiqueCorrel>
\end{verbatim}

Comme vu avec l'\'equation~\ref{Corr:As:Norme}, le coefficient
de corr\'elation est directement fonction de la distance au carr\'e
entre des vignettes normalis\'ees. Ce tag permet de sp\'ecifier
comment convertir le coefficient de corr\'elation en un co\^ut
utilis\'e pour l'attache aux donn\'ees. Soit $Cost$ le co\^ut
et $Corr$ le coefficient de corr\'elation, les valeurs possibles
sont~:


\begin{itemize}
   \item {\tt eCoeffCorrelStd} le co\^ut est  donn\'e par
         $Cost=1-Corr$, ce co\^ut est donc un \'ecart quadratique
         entre les vignettes normalis\'es;


   \item {\tt eCoeffAngle} le co\^ut est  donn\'e par
         $Cost=acos(Corr)$; lorsque l'on a deux images,
         en revenant \`a l'interpr\'etation du coefficient
         de corr\'elation comme le cosinus de l'angle entre
         $\bar{u}$ et $\bar{v}$ (d\'eriv\'e de la formule~\ref{Cor:Prod:Scal}),
         ce co\^ut peut \^etre interpr\'et\'e comme l'angle entre
          $\bar{u}$ et $\bar{v}$; de mani\`ere plus g\'en\'erale,
         ce co\^ut s'interpr\`ete comme un p\'enalisation $L^1$
         alors que {\tt eCoeffCorrelStd} est plut\^ot une p\'enalisation
         $L^2$.

\end{itemize}

{\bf MODIF:}  Pour que l'utilisateur puisse sp\'ecifier n'importe
quelle fonction rajouter la possibilit\'e d'avoir une dynamique "tabul\'ee".

    %---------------------------------------------
\subsection{A priori}

\subsubsection{R\'egularisation}

\emph{Pas de valeur par d\'efaut pour les 3 premiers}

\emph{Nuls pas d\'efauts pour les couts quadratiques}

\begin{verbatim}
     <ZRegul Nb="?"  Type="double">     </ZRegul>
     <Px1Regul Nb="?"  Type="double">     </Px1Regul>
     <Px2Regul Nb="?"  Type="double">     </Px2Regul>

      <ZRegul_Quad Nb="?"  Type="double">     </ZRegul_Quad>
      <Px1Regul_Quad Nb="?"  Type="double">     </Px1Regul_Quad>
      <Px2Regul_Quad Nb="?"  Type="double">     </Px2Regul_Quad>


\end{verbatim}


Ces valeurs correspondent aux termes de r\'egularisation
$\alpha_k$  d\'efinis en~\ref{DUMG:Appr:Energ};  il permettent
de pond\'erer l'attache aux donn\'ees par rapport \`a l'\emph{a priori}
de r\'egularit\'e (plus ils sont fort, plus on impose un r\'esultat
r\'egulier).
Ces param\`etres posent parfois probl\`eme aux utilisateurs
MicMac car il est difficile de donner  des  consignes rationnelles
sur les moyen de r\`egler leur valeur. Une  remarque
qui peut \^etre faite, est qu'il n'est en g\'en\'eral pas
n\'ecessaire de modifer leur valeur en fonction du niveau
de {\tt DeZoom}; d'un part, en premi\`ere approximation il
peuvent \^etre consid\'er\'es comme invariant \`a l'\'echelle et,
d'autre part, leur r\'eglage pour les basses r\'esolution n'est
pas critique. A d\'efaut de r\`egles rationnelles, on peut
donner quelques plages de valeurs empiriques sur les principaux
cas d'utilisation de MicMac :


\begin{itemize}
    \item pour les MNE urbain, en multi-vues \`a haute r\'esolution,
          $\alpha_1$ est typiquement dans l'intervalle $[0.01 \; 0.05]$;
          la valeur est faible car, d'une part le multi-vues rend plus
          fiable l'attache aux donn\'ees et d'autre part l'{\tt a priori}
          de r\'egularit\'e est faible sur
          le relief urbain et  ses fortes discontinuit\'es;

    \item pour les MNT spot-5, en st\'er\'eo classique,
           $\alpha_1$ est typiquement dans l'intervalle $[0.1 \; 0.2]$;

    \item pour la superposition des cannaux de la cam\'era num\'erique,
           $\alpha_1$ et  $\alpha_2$ sont typiquement dans l'intervalle
           $[0.5 \; 2.0]$; ici le champ recherch\'e est tr\`es
           r\'egulier (fonction basse fr\'equence, faible amplitude);

    \item pour du calcul de points de liaisons denses en a\'ero-triangulation,
          on peut partir de  $\alpha_1$ dans $[0.05 \; 0.2]$ et $\alpha_2$
          dans $[0.25 \; 2.0]$; $\alpha_1$ et  $\alpha_2$ sont a priori
          tr\`es diff\'erents puisque  $\alpha_1$ est d\'etermin\'e par
          l'incertitude sur le relief alors que $\alpha_2$ est d\'etermin\'e
          par l'incertitude sur la mise en place initiale; le ratio
          de variation possible de $\alpha_2$ est tr\`es grand car
          l'\emph{a priori} sur les orientation peut \^etre tr\`es
          variable (typiquement quelques pixels \`a quelques centaines
          suivant les contextes);

\end{itemize}

Ces r\`egles peuvent servir de valeur initiale dans une approche par
tatonnement, il ne faut surtout pas h\'esiter \`a les remettre en cause
\`a la premi\`ere occasion.

Les trois derniers param\`etres permettent de rajouter un terme
quadratique dans le crit\`ere de co\^ut (comme par
exemple~\ref{CostPrgDyn:Quad}). Ces param\`etres ne sont accessibles
qu'avec un optimisation de type programmation dynamique.
Ils peuvent \^etre utilis\'es par exemple pour de la production
de mod\`eles num\'eriques de terrain.

\subsubsection{Post-filtrage}

    %---------------------------------------------
\subsection{Minimisation}

\subsubsection{Choix d'un algorithmes}

\begin{verbatim}
    <AlgoRegul Nb="?" Type="eAlgoRegul"> </AlgoRegul>
\end{verbatim}


    %---------------------------------------------
\subsubsection{Param\`etre sp\'ecifiques \`a Cox-Roy}

\emph{Par d\'efaut CoxRoy8Cnx vaut false, CoxRoyUChar vaut true}

\begin{verbatim}
   <CoxRoy8Cnx Nb="?" Type="bool"> </CoxRoy8Cnx>
    <CoxRoyUChar Nb="?" Type="bool"> </CoxRoyUChar>
\end{verbatim}


Si {\tt CoxRoy8Cnx} vaut {\tt true} le voisinage d\'efini
dans~\ref{CostGrad:MaxFlot} est le $8$-voisinage, sinon c'est
le $4$-voisinage.  L'utilisation du  $8$-voisinage permet
de diminuer sensiblement les artefacts directionnels qui
apparaissent avec le  $4$-voisinage au prix d'un doublement,
en moyenne, du temps de calcul.

Les algorithme de flots sont assez consommateurs en m\'emoire;
pour chaque pixel et chaque parallaxe et chaque, il faut m\'emoriser
autant de valeur num\'erique qu'il y a de voisins ($4$ ou $8$).
Le tag {\tt CoxRoyUChar} permet de contr\^oler le compromis "encombrement
m\'emoire / dynamique"~:

\begin{itemize}
   \item si {\tt CoxRoyUChar} vaut {\tt true}, les valeurs num\'erique sont
         stock\'ees sur un octet; cela a pour cons\'equence que
         les coefficient de corr\'elation et les terme de r\'egularisation
         sont arrondies au $\frac{1}{100}$; cette option est donc
         d\'econseill\'ee avec des r\'egularisations  tr\`es faibles;

   \item si {\tt CoxRoyUChar} vaut false, les valeurs num\'erique sont
         stock\'ees sur deux octets; les valeurs
         sont arrondies au $\frac{1}{10000}$;
\end{itemize}

    %---------------------------------------------
\subsubsection{Param\`etre sp\'ecifiques \`a la programmation dynamique}

\emph{ModulationProgDyn est obligatoire si le mode
 algorithmique choisi est eAlgo2PrgDyn}

\begin{verbatim}
 <ModulationProgDyn Nb="?">
    <EtapeProgDyn  Nb="+">
       <Px1MultRegul Nb="?" Type="std::vector<double>"> </Px1MultRegul>
       <Px2MultRegul Nb="?" Type="std::vector<double>"> </Px2MultRegul>
       <NbDir Nb="?" Type="int" Def="2">             </NbDir>
       <ModeAgreg Type="eModeAggregProgDyn" Nb="1">  </ModeAgreg>
       <Teta0 Nb="?" Type="double" Def="0.0">         </Teta0>
    </EtapeProgDyn>
    <Px1PenteMax Nb="?" Type="double" Def="10.0"> </Px1PenteMax>
    <Px2PenteMax Nb="?" Type="double" Def="10.0"> </Px2PenteMax>
  </ModulationProgDyn>
\end{verbatim}

MicMac utilise le m\'ecanisme de
"programmation dynamique multi-directionnelle"
d\'ecrit en~\ref{ProgDyn:MulDir} auquel on pourra se r\'eferrer.
L'image est balay\'ee selon les
angles $Teta0 + k \frac{\pi}{NbDir}$. Les mesures obtenues pour
chaque direction sont
agr\'eg\'es en fonction de {\tt ModeAgreg} qui peut prendre
une des valeurs \'enum\'er\'ees :

\begin{itemize}
      \item {\tt ePrgDAgrSomme}, on aggr\`ege en calculant la  moyenne;
      \item {\tt ePrgDAgrMax},    on aggr\`ege en retenant le maximum
      \item {\tt ePrgDAgrReinject}, on utilise un mode r\'eentrant (le r\'esultat
            de chaque balayage est utilis\'e comme entr\'ee pour le prochain;
\end{itemize}

{\tt Px1PenteMax} et {\tt Px2PenteMax} permettent d 'imposer une contrainte
de pente maximale. On a tout int\'eret \`a imposer cette contrainte lorsque
cela est pertinent pour le probl\`eme pos\'e. En plus de l'avantage
\'evident d'ajouter une connaissance \emph{a priori}, cela permet
de limiter la combinatoire (en \'evitant d'explorer tous les couples
de paralaxe de pixels voisins).

Lorsqu'il y a plusieurs \'etape {\tt EtapeProgDyn} elle sont
encha\^in\'ees selon un mode r\'eentrant.


{\tt Px1MultRegul} et {\tt Px2MultRegul} sont des param\`etres \'esot\'eriques.

    %---------------------------------------------


    %---------------------------------------------
\subsection{Sous r\'esolution des algorithmes}

\subsection{Option non implant\'ees}


%---------------------------------------------
%---------------------------------------------
%---------------------------------------------

\section{Gestion m\'emoire}


