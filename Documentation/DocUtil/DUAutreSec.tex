\chapter{Sections hors mise en correspondance}

\label{UTIL:AUTRESEC}


%-------------------------------------------------------------------
%-------------------------------------------------------------------
%-------------------------------------------------------------------

\section{Section Terrain}

\begin{verbatim}
<Section_Terrain Nb="1">
       <IntervAltimetrie Nb="?">  ... </IntervAltimetrie>
       <IntervParalaxe Nb="?">    ... </IntervParalaxe>
       <Planimetrie Nb="?">       ... </Planimetrie>
       <RugositeMNT Nb="?">       ... <RugositeMNT>
</Section_Terrain>
\end{verbatim}

Cette section contient les informations li\'ees au terrain lui 
m\^eme, ind\'epdendemment de la fa\c{c}on dont il est imag\'e.

            % - - - - - - - - - - - - - - - - - - - -

\subsection{ {\tt <IntervParalaxe>} et {\tt <IntervAltimetrie>} }

\label{SEC:Interv:Px}

\begin{verbatim}
...
      <IntervAltimetrie Nb="?">
             <ZMoyen Nb="?" Type="double">       </ZMoyen>
             <ZIncCalc Nb="1" Type="double">     </ZIncCalc>
             <ZIncZonage Nb="?" Type="double">   </ZIncZonage>
             <MNT_Init Nb="?"> ... </MNT_Init>
       </IntervAltimetrie>

       <IntervParalaxe Nb="?">
             <Px1Moy Nb="?" Type="double">      </Px1Moy>
             <Px2Moy Nb="?" Type="double">      </Px2Moy>
             <Px1IncCalc Nb="1" Type="double">      </Px1IncCalc>
             <Px2IncCalc Nb="1" Type="double">      </Px2IncCalc>
             <Px1IncZonage Nb="?" Type="double">    </Px1IncZonage>
             <Px2IncZonage Nb="?" Type="double">    </Px2IncZonage>
       </IntervParalaxe>
...
\end{verbatim}



Ces deux sections jouent le m\^eme r\^ole, {\tt <IntervParalaxe>}
est utilis\'e lorsque la parallaxe est de dimension $2$ et 
{\tt <IntervAltimetrie>} lorsqu'elle est de dimension $1$ .
Elle ne diff\`erent que par {\tt <MNT\_Init>}, sp\'ecifique \`a  
{\tt <IntervAltimetrie>} et d\'ecrit en~\ref{MNT:INIT}.


\begin{figure}
\begin{tabular} {  c c c c  } \\  \hline
               & Z & Px1 & Px2\\ \hline
       Moyenne & ZMoyen & Px1Moy & Px2Moy\\ \hline
       Incert-Calc & ZIncCalc & Px1IncCalc & Px2IncCalc\\ \hline
       Incert-Zone & ZIncZonage & Px1IncZonage & Px2IncZonage\\ \hline
\end{tabular}
\caption {Equivalence de noms sur les intervalles de parallaxe}
\label{TAB:EQUIV:IntervPx}
\end{figure}


L'unit\'e  des valeurs exprim\'ees dans cette section 
est celle donn\'ee par le tableau~\ref{TAB:DIM:PX} sauf lorsque 
cette unit\'es est du  M\`etre$^{-1}$ auquel cas les valeurs  sont exprim\'ee
en m\`etre.

     %    -   -   -   -   -   -


\subsubsection{Valeurs moyennes de parallaxe}
Les tags sont  {\tt   <ZMoyen>} ($d=1$) ou {\tt <Px1Moy>} et
{\tt <Px2Moy>} ($d=2$).

Ces tags fixent la valeur moyenne de la parallaxe; elles ont donc
une influence directe sur la zone de recherche explor\'ee lors
du premier niveau de la pyramide de r\'esolution. Accessoirement,
elles ont une influence sur le formatage du r\'esultat (le contenu
des fichiers est exprim\'e en relatif par rapport \`a cette valeur moyenne).

Dans le fichier de sp\'ecifications, on peut voir que
l'arit\'e de ces valeurs est "?"; ces valeurs sont optionnelle lorsque
la g\'eom\'etrie intrins\`eque  fournit  une valeur moyenne par d\'efaut.
Plus pr\'ecis\'ement :

\begin{itemize}
    \item optionnelle pour les g\'eom\'etries purement image, la valeur
          par d\'efaut est $0$  les param\`etres annexes (par ex : distorsion
          et homographie) \'etant suppos\'es
          fournir une bonne pr\'ediction;

    \item optionnelle pour {\tt eGeomImageOri}, le format {\tt Ori} contenant
          une information d'altitude moyenne;

    \item pour {\tt <eGeomImageModule>} ca d\'epend de ce qui est impl\'ement\'e
          dans la librairie dynamique \dots  ; obligatoire pour l'impl\'ementation
          actuelle du format grille d'IGN-espace;
\end{itemize}

On notera \PxMoy cette valeur.


     %    -   -   -   -   -   -

\subsubsection{Incertitude de calcul}

Les tags sont  {\tt   <ZIncCalc>} ($d=1$) ou {\tt <Px1IncCalc>} et
{\tt <Px2IncCalc>} ($d=2$) et  ils sont obligatoires. Ces valeur
permettent de d\'efinir les deux nappes encadrantes utilis\'ees
pour d\'efinir la zone de recherche au premier niveau de la pyramide.

Par exemple, pour $d=1$, la nappe initiale est d\'efinie par 
l'intervalle {\tt [ZMoyen-ZIncCalc , ZMoyen+ZIncCalc]}.

     %    -   -   -   -   -   -

\subsubsection{Incertitude pour le calcul d'emprise}

Les tags sont  {\tt   <ZIncZonage>} ($d=1$) ou {\tt <Px1IncZonage>} et
{\tt <Px2IncZonage>} ($d=2$). Il sont facultatifs et lorsqu'ils sont
omis, ils prennent la m\^eme valeur que {\tt   <ZIncCalc>} \dots .

Ces valeurs servent \`a contr\^oler l'emprise du chantier lorsqu'elle celle
ci est calcul\'ee automatiquement par MicMac voir~\ref{Plani:Def:EmpriseTerrain}

On notera \PxZone cette valeur.

     %    -   -   -   -   -   -

\subsubsection{MNT initial}

\label{MNT:INIT}

\begin{verbatim}
<MNT_Init Nb="?">
     <MNT_Init_Image Nb="1" Type="std::string"> </MNT_Init_Image>
     <MNT_Init_Xml Nb="1" Type="std::string">   </MNT_Init_Xml>
     <MNT_Offset Nb="?"  Type="double" Def="0.0"> </MNT_Offset>
</MNT_Init>
\end{verbatim}

Cette structure permet d'utiliser un MNT pour initialiser le
calcul. La structure est fille de {\tt <IntervAltimetrie>} et
n'est accesible qu'en dimension $1$. Les trois champs qui
la composent sont :


\begin{itemize}
   \item {\tt <MNT\_Init\_Image>} le nom du fichier image contenant le MNT;

   \item {\tt <MNT\_Init\_Xml>} le nom du fichier {\tt xml} contenant
         le g\'eor\'ef\'erencement du MNT au format {\tt <FileOriMnt>} 
         (d\'ecrit en ~\ref{Format:MNT}); les contraintes sont les 
         m\^emes qu'en~\ref{Masq:Ter:Util};

   \item {\tt <MNT\_Offset>} un \'eventuel offset rajout\'e au MNT
          avant de l'utiliser comme valeur initiale;
         cette valeur est utile lorsque , un fois
         le MNT connu, les  intervalle d'incertitude
         sur le relief sont asym\'etriques; typiquement, les altitudes
         attendues au dessus du MNT peuvent \^etre \'elev\'ee car elle
         sont dues \`a tous le sursol, alors que celle en dessous
         restent faible car elles ne
         refl\'etenet que  l'erreur sur le MNT lui m\^eme ;
         si par exemple, l'intervalle d'incertitude est $[-20,80]$,
         on pourra fixer  {\tt <MNT\_Offset>}  \`a $30$ et {\tt <ZIncCalc>}
         \`a $50$;
\end{itemize}

            % - - - - - - - - - - - - - - - - - - - -

\subsection{Planim\'etrie}

\label{Autresec:Planimetrie}

\begin{verbatim}
   <Planimetrie Nb="?">
         <BoxTerrain Nb="?" Type="Box2dr"> </BoxTerrain>
         <ListePointsInclus Nb="*">
              <Pt Nb="+" Type="Pt2dr"> </Pt>
              <Im Nb="1" Type="std::string"> </Im>
         </ListePointsInclus>
         <ResolutionTerrain Nb="?" Type="double"> </ResolutionTerrain>
         <MasqueTerrain Nb="?">  ... </MasqueTerrain>
         <RecouvrementMinimal Nb="?" Type="double">  </RecouvrementMinimal>
   </Planimetrie>
\end{verbatim}

La section {\tt <Planimetrie>} ainsi que tous ses sous-sections
sont optionnelles.

\subsubsection{Calcul de l'emprise sp\'ecifi\'ee}

L'emprise est sp\'ecifi\'ee par l'utilisateur lorsque le
champs {\tt <BoxTerrain>} a une valeur ou que la liste
{\tt <ListePointsInclus>} est non vide.

Chaque \'el\'ement $l$ de la liste $L$ de {\tt <ListePointsInclus>},
permet de sp\'ecifier  une image $I_l$, et un point $P_l$ 
de cette image dont on souhaite que l'homologue terrain soit pr\'esent
dans l'emprise terrain.

On note $B^T$ la bo\^ite englobante qui vaut {\tt <BoxTerrain>} 
quand elle est d\'efinie et $\emptyset$ sinon.

Lorsque l'emprise est sp\'ecifi\'ee elle vaut:

\begin{equation}
   B^T + \sum_{l \in L} B^x(\pi_{I_l}^{-1}(P_l,\PxMoy))
\end{equation}

Conventionnellement, si {\tt Im} vaut \emph{Terrain} il s'agit
d'un point terrain (au sens de la g\'eom\'etrie intrins\`eque).

            % - - - - - - - - - - - - - - - - - - - -

\subsubsection{Calcul de l'emprise par d\'efaut}

Lorsqu'aucune emprise terrain n'est donn\'ee MicMac
en calcule une selon les sp\'ecifications suivantes: la 
bo\^ite englobante de l'ensemble des points terrains
qui, compte-tenu des incertitudes sur la parallaxes,
soient succeptibles d'\^etre vus dans au moins $2$ images.

\label{Plani:Def:EmpriseTerrain}

Formellement on calcule cette bo\^ite par la formule suivante :

\begin{equation}
   \DeuxExtre_{k\in[1,N]}(B^x(\pi_k^{-1}(\EIm ,\PxZone)) +B^x(\pi_k^{-1}(\EIm ,-\PxZone)))
\end{equation}

            % - - - - - - - - - - - - - - - - - - - -

\subsubsection{R\'esolution terrain}

\label{Autresec:Resol:Terr}
Le tag {\tt <ResolutionTerrain>} permet de sp\'ecifier la
r\'esolution \`a laquelle l'espace terrain est discr\'etis\'e.
Cette valeur est exprim\'ee dans l'unit\'e terrain associ\'ee
\`a la g\'eom\'etrie de restitution telle qu'exprim\'ee 
en~\ref{TAB:DIM:PX}.

Lorsque cette valeur  est omise, une valeur est calcul\'ee en
faisant la moyenne des r\'esolution qui sont indiqu\'ees
par les prises de vues. Ces valeurs par d\'efaut sont les suivantes:

\begin{itemize}
   \item pour les g\'eom\'etrie  purement images, la valeur
         par d\'efaut est $1.0$;

   \item pour  la g\'eom\'etrie  {\tt eGeomImageOri} une 
         valeur est calcul\'ee en tenant compte de la focale, de la
         hauteur de vol et de l'altitude sol;

    \item pour {\tt <eGeomImageModule>} ca d\'epend de ce qui est impl\'ement\'e
          dans la librairie dynamique \dots  ; dans l'impl\'ementation
         actuelle des g\'eom\'etries grilles, une valeur est 
         calcul\'ee en utilisant la grille au centre de l'image;
\end{itemize}


            % - - - - - - - - - - - - - - - - - - - -

\subsubsection{Masque terrain}

\begin{verbatim}
    <MasqueTerrain Nb="?">
         <MT_Image Nb="1" Type="std::string"> </MT_Image>
         <MT_Xml Nb="1" Type="std::string"> </MT_Xml>
    </MasqueTerrain>
\end{verbatim}


\label{Masq:Ter:Util}

La structure {\tt <ImMasqImported>} permet de sp\'ecifier un masque 
utilisateur qui va se superposer au masque calcul\'e automatiquement
par MicMac (voir~\ref{MEC:Masq}). Ce masque doit utiliser le m\^eme
syst\`eme de coordonn\'ees que la g\'eom\'etrie utilis\'ee sur
le chantier (par exemple m\^eme zone lambert \dots), par
contre il n'a pas n\'ec\'essairement la m\^eme origine ou
la m\^eme \'echelle que le chantier.

Le tag {\tt <MT\_Image>} sp\'ecifie un nom de fichier image
et le tag {\tt <MT\_Xml>} un nom de fichier {\tt xml} permettant
de g\'eo-r\'ef\'erencer le fichier image. Le fichier  {\tt xml} 
doit \^etre au format {\tt <FileOriMnt>} (voir~\ref{Format:MNT},
ceci implique qu'un r\'esolution et un origine alti doivent \^etre
sp\'ecifi\'ees m\^eme si elles ne seront pas utilis\'ees).

            % - - - - - - - - - - - - - - - - - - - -

\subsubsection{Recouvrement minimal}

Ce tag permet de sp\'ecifier une valeur minimale de l'emprise
terrain. Cette valeur est sp\'ecifi\'ee en proportion de la
taille de la premi\`ere image (par exemple, $0.01$ correspond
\`a $160 Ko$ avec des images $4000*4000$).

Ce tag a \'et\'e rajout\'e lors de l'utilisation de MicMac pour
le calcul automatique de points homologues, il permet d'arr\^eter MicMac
au plus t\^ot lorsque la zone de recouvrement est trop petite
et risque de cr\'eer des d\'eg\'enrescences ult\'erieure.

D\`es que la taille de zone a \'et\'e calcul\'ee, si elle est inf\'erieure
\`a la valeur seuil, MicMac s'arr\^etre en g\'en\'erant l'erreur
catalogu\'ee {\tt eErrRecouvrInsuffisant} (voir~\ref{Err:Catal}).


            % - - - - - - - - - - - - - - - - - - - -
            % - - - - - - - - - - - - - - - - - - - -

\subsection{Param\`etres li\'es \`a la "rugosit\'e"}

\begin{verbatim}
<RugositeMNT Nb="?">
  <EnergieExpCorrel Nb="?" Type="double" Def="-2.0">      </EnergieExpCorrel>
  <EnergieExpRegulPlani Nb="?" Type="double" Def="-1.0">  </EnergieExpRegulPlani>
  <EnergieExpRegulAlti Nb="?" Type="double" Def="-1.0">   </EnergieExpRegulAlti>
</RugositeMNT>
\end{verbatim}


Trois param\`etres li\'es \`a la rugosit\'e du terrain
qui permettent de sp\'ecifier comment les param\`etre
de  r\'egularisation doivent \'evoluer en fonction du facteur
d'\'echelle. 

Non d\'etaill\'es pour l'instant, car je ne suis pas convaincu
qu'il y ait le moindre int\'er\^et \`a utiliser autre chose
que les valeurs par d\'efaut.

%-------------------------------------------------------------------
%-------------------------------------------------------------------
%-------------------------------------------------------------------

\section{Section Prise de Vue}

\begin{verbatim}
 <Section_PriseDeVue Nb="1">
        <Images Nb="1"> ...  </Images>

        <ValSpecNotImage/ Nb="?" Type="int">
        <PrefixMasqImRes Nb="?" Type="std::string" Def="MasqIm">  
        <DirMasqueImages Nb="?" Type="std::string" Def=""> 
        <MasqImageIn Nb="*">
               <OneMasqueImage Nb="*"> ... </OneMasqueImage>
        </MasqImageIn>

        <GeomImages/ Nb="1">
        <NomsGeometrieImage Nb="+">...  </NomsGeometrieImage>
        <ModuleGeomImage Nb="?">... </ModuleGeomImage>
        <NomsHomomologues Nb="?"> ...  </NomsHomomologues>
 </Section_PriseDeVue>
\end{verbatim}

Cette section d\'ecrit l'ensemble des caract\'eristiques du chantier
li\'ees \`a la prise de vue, c'est \`a dire \`a l'ensembles des images et
de leur g\'eom\'etrie.

\label{UTIL:AUTRES:PDV:GI}
            % - - - - - - - - - - - - - - - - - - - -

\subsection{Images}
            %    -   -   -   -   -   -
\subsubsection{Ensemble des images}

\begin{verbatim}
        <Images Nb="1">
            <Im1 Nb="?" Type="std::string">   </Im1>     
            <Im2 Nb="?" Type="std::string">   </Im2>
            <ImPat Nb="*" Type="std::string">   </ImPat>
        </Images>
\end{verbatim}

Cette section permet de sp\'ecifier l'ensemble des images qui
constituent le chantier. Les noms et patron de noms sont toujours
indiqu\'es en relatif par rapport \`a la directory  d'instalation
des images (voir~\ref{DU:Dir:Images}).


{\tt <Im1>} et {\tt <Im2>} (facultatifs) sont des noms d'images tandis que {\tt <ImPat>}
correspond \`a une liste de pattern de s\'election.  C'est l'ensemble
des images sp\'ecifi\'ees qui constituera la s\'election retenu pour
le chantier.  Les noms sp\'ecifi\'es peuvent	 se retrouver en plusieurs endroit,
MicMac supprimera les duplicatas. La seule contrainte r\'eelle est qu'il
y ait en tout au moins $2$ noms d'images diff\'erents.

{\tt <Im1>} et {\tt <Im2>}  peuvent para\^itre redondants compte-tenu du m\'ecanisme
assez souple offert par les patrons {\tt <ImPat>}. Les raison qui justifient
la pr\'esence de ces tags sont les suivantes :

\begin{itemize}
   \item Si la g\'eom\'etrie s\'electionn\'ee contient la notion d'images ma\^itresse 
         la pr\'esence de {\tt <Im1>} permet de sp\'ecifier laquelle des images sera
         ma\^itresse;
 
   \item lorsque MicMac est rappel\'e  par un programme \emph{ma\^itre} pour g\'en\'erer
         des mises en correspondances sur un grand nombre de couples (a\'erotriangulation,
         superposition multi-spectrale), la pr\'esence de {\tt <Im1>} et {\tt <Im2>}
         permet de sp\'ecifier chaque couple d'image par ligne de commande (ce qui ne
         serait pas possible aujourd'hui avec les {\tt <ImPat>} \`a cause de son
         arit\'e, voir~\ref{DU:Modif:LC});

   \item  en g\'eom\'etrie image, il est plus naturel de sp\'ecifier 
         explicitement quelles sont {\tt <Im1>} et {\tt <Im2>}  plut\^ot que
         de reposer sur l'ordre des noms  \dots
\end{itemize}

Supposons par exemple que la directory images contienne des fichiers
{\tt Im000.tif Im001.tif \dots Im999.tif}, le code ci dessous s\'electionnera
les images de num\'eros compris entre $200$ et $299$ ou $400$ et $499$, et
l'image ma\^itresse sera $293$ (si cela  a un sens pour la g\'eom\'etrie).

\begin{verbatim}
        <Images>
            <Im1> Im293.tif  </Im1>     
            <ImPat> Im2..\.tif  </ImPat>
            <ImPat> Im4..\.tif  </ImPat>
        </Images>
\end{verbatim}

            %    -   -   -   -   -   -
\subsubsection{Masque images}

{\small
\begin{verbatim}
<PrefixMasqImRes Nb="?" Type="std::string" Def="MasqIm">    </PrefixMasqImRes>
<ValSpecNotImage Nb="?" Type="int"> </ValSpecNotImage>

<DirMasqueImages Nb="?" Type="std::string" Def=""> </DirMasqueImages>
<MasqImageIn Nb="*">
      <!-- En cas de match multiple, c'est le dernier qui compte -->
      <OneMasqueImage Nb="*">
          <PatternSel Nb="1" Type="cElRegex_Ptr">    </PatternSel>
                   <!-- Si NomMasq=PasDeMasqImage -> Valeur Tag
                        pour ne pas aller chercher de masque -->
          <NomMasq Nb="1"  Type="std::string">      </NomMasq>
      </OneMasqueImage>
</MasqImageIn>
\end{verbatim}
}

MicMac maintient pour chaque image
un masque binaire indiquant la zone valide de l'image.
Lors du calcul d'un score de corr\'elation, dans 
la phase de  mise en correspondance, pour chaque point et chaque parallaxe
seules seront conserv\'ees les images telles que tous les points
de la vignette se projettent en des points valides.
Le masque est repr\'esent\'e et m\'emoris\'e explicitement 
par une pyramide d'image.  Les pyramide cr\'e\'ees sont stock\'ees
dans le r\'epertoire {\tt <TmpPyr>} et les 
noms des fichiers de masque  sont construits
en rajoutant au nom du fichier image le pr\'efixe  {\tt <PrefixMasqImRes>}
(qui par d\'efaut vaut {\tt MasqIm}).

On peut sp\'ecifier plusieurs masques qui se superposent.
Soit par exemple $Im$ une image, l'ensemble des  
masques se superposant est construit de la mani\`ere suivante :

\begin{itemize}

   \item  si on ne sp\'ecifie rien, toute la zone de l'image est valide.

   \item  si l'on sp\'ecifie une valeur $Val$ dans {\tt <ValSpecNotImage>} ,
          un premier masque est construit correspondant aux 
          seuls points ayant une radiom\'etrie diff\'erente de $Val$ dans l'image
          $Im$ seront valides (m\'ethode un peu archa\"ique, maintenue surtout
          par compatibilit\'e);

    \item \`a cet \'eventuel masque, on peut superposer autant de masque que 
          l'on veut via la liste  {\tt  <MasqImageIn >}; le fonctionnement de
           {\tt  <MasqImageIn >} est le suivant :

    \begin{itemize}
          \item  chaque \'element de la liste  {\tt  <MasqImageIn >} doit permettre
                 de calculer  le masque image associ\'e \`a $Im$;

          \item  on parcourt la liste {\tt <OneMasqueImage>} ; soit 
                 le dernier \'el\'ement pour lequel l'expression r\'eguli\`ere 
                 {\tt <PatternSel>} est  appariable sur le nom $Im$, on
                 utilise alors {\tt <NomMasq>} pour calculer le masque
                 associ\'e selon le m\'ecanisme habituel (voir~\ref{DU:Regex});
                 ces images sont recherch\'ees dans {\tt   <DirMasqueImages>};

          \item  si aucune expression r\'eguli\`ere n'est appariable, une
                 erreur est g\'en\'er\'ee; il est cependant possible
                 de sp\'ecifier qu'aucun masque n'est associ\'e en faisant
                 que {\tt <NomMasq>} renvoie la valeur cl\'e {\tt "PasDeMasqImage"}

     \end{itemize}
\end{itemize}

L'extrait ci-dessous est un exemple d'utilisation r\'eel des masques:

\begin{itemize}
  \item pour toutes les images, les points valant $0$ sont hors masques;

  \item pour l'image {\tt 299.tif}, c'est le seul masque utilis\'e; 

  \item pour les autres images {\tt Im.tif}, on va superposer en plus l'image
        de nom {\tt Im.masque.tif} dans le r\'epertoire {\tt masques/};
        
\end{itemize}


\begin{verbatim}
     <ValSpecNotImage> 0 </ValSpecNotImage>
     <DirMasqueImages> masques/ </DirMasqueImages>

     <MasqImageIn>
        <OneMasqueImage>
              <PatternSel> (.*)\.tif</PatternSel>
              <NomMasq>   $1.masque.tif </NomMasq>
        </OneMasqueImage>
        <OneMasqueImage>
              <PatternSel> 299.tif </PatternSel>
              <NomMasq>  PasDeMasqImage  </NomMasq>
        </OneMasqueImage>
     </MasqImageIn>
\end{verbatim}


\subsubsection{Gestionnaire de pyramide d'image}

\begin{verbatim}
<NomsGeometrieImage Nb="+">
      ...
      <ModuleImageLoader Nb="?">
           <NomModule Nb="1" Type="std::string"> </NomModule>
           <NomLoader Nb="1" Type="std::string"> </NomLoader>
      </ModuleImageLoader>
      ...
</NomsGeometrieImage>
\end{verbatim}

Le tag {\tt <ModuleImageLoader>} permet de sp\'ecifier  un gestionnaire
de pyramide d'image. Ce tag a \'et\'e mis comme fils du tag
{\tt <NomsGeometrieImage>} pour factoriser le m\'ecanisme de s\'election
par le nom de l'image (voir~\ref{Assoc:Image:Geom}) et pouvoir, si n\'ecessaire, associer
diff\'erents gestionnaires en fonction du nom de l'image.

Si rien n'est sp\'ecifi\'e, c'est le gestionnaire par d\'efaut de MicMac
qui est utilis\'e;  ce gestionnaire s'attend \`a ce que le nom de l'image
soit un fichier {\tt tif} (ou thom) contenant l'image \`a r\'esolution
$1$, il g\'en\'era sa propre pyramide au fur et a mesure des besoins.

S'il est sp\'ecifi\'e,  alors {\tt <NomModule>} est le nom d'une
librairie dynamique et {\tt <NomLoader>} est une entr\'ee dans cette
librairie (selon le m\'ecanisme d\'ecrit en ~\ref{Lib:Dyn}).
Gr\'egoire Maillet a commenc\'e \`a \'ecrire un module 
utilisant une librairie  JPEG-2000 pour en faire un gestionnaire
de pyramide utilisable dans MicMac. Ce gestionnaire s'attend 
\`a ce que le nom de l'image soit un fichier JPEG-2000 et
utilise ce seul fichier pour retrouver tous les niveaux de
la pyramide.

Il serait possible d'\'ecrire une gestionnaire pour utiliser
le format {\tt DMR} utilis\'e par IGN-espace.


            % - - - - - - - - - - - - - - - - - - - -

\subsection{G\'eom\'etrie (intrins\`eque)}


\subsubsection{Type de G\'eom\'etrie}

\begin{verbatim}
        <GeomImages Nb="1" Type="eModeGeomImage"> </GeomImages>
        <ModuleGeomImage Nb="?">            
           <NomModule Nb="1" Type="std::string"> </NomModule>
           <NomGeometrie Nb="1" Type="std::string"> </NomGeometrie>    
        </ModuleGeomImage>
\end{verbatim}

Ce tag obligatoire sp\'ecifie le format de g\'eom\'etrie intrins\`eque
utilis\'ee. Il doit avoir une valeur dans l'\'enumeration {\tt eModeGeomImage}
(voir~\ref{DuMeca:Geom:IntrinGene}).

Si le {\tt GeomImages} a la valeur {\tt eGeomImageModule}, alors
{\tt ModuleGeomImage} doit avoir une valeur pour sp\'ecifier
une librairie dynamique (fichier {\tt <NomModule>} , 
entr\'ee {\tt NomGeometrie}) permettant de charger le code
utilisateur d\'ecrivant la g\'eom\'etrie (voir~\ref{Lib:Dyn}
et~\ref{DU:Meca:Geo:Mod}).

{\bf MODIF:} Aujourd'hui il n'est pas possible de m\'elanger plusieurs
formats de g\'eom\'etrie dans un m\^eme chantier. Ajouter cette 
possibilit\'e ne devrait pas poser de vrai probl\`eme  il est
pr\'evu de le faire \`a terme (priorit\'e \emph{a priori} assez
basse).


            % - - - - - - - - - - - - - - - - - - - -
\subsubsection{Association images/g\'eom\'etries, cas standard}

\label{Assoc:Image:Geom}

\begin{verbatim}
       <NomsGeometrieImage Nb="+">
        ...
            <PatternSel Nb="1" Type="std::string">    </PatternSel>
            <PatNameGeom Nb="1" Type="std::string">   </PatNameGeom>
         ...
        </NomsGeometrieImage>
\end{verbatim}

La liste non vide (arit\'e ="+") {\tt NomsGeometrieImage} contient
les informations pour calculer automatiquement le nom du fichier
de  g\'eom\'etrie \`a partir du nom de fichier image. Le m\'ecanisme
a \'et\'e mise en place avec l'objectif d'\^etre suffisamment souple
pour qu'il ne soit jamais n\'ecessaire
de renommer les fichier en entr\'ee  de MicMac.

Soit une image de nom de fichier {\tt UnNomImage}, pour calculer
le nom de la g\'eom\'etrie, MicMac proc\`ede ainsi :

\begin{itemize}
   \item la liste {\tt NomsGeometrieImage} est parcourue dans l'ordre;
   \item MicMac s'arr\^ete au premier \'el\'ement tel que {\tt UnNomImage}
         soit appariable sur l'expression r\'eguli\`ere {\tt <PatternSel>} ;
   \item  \`a partir de cet appariement, le pattern {\tt PatNameGeom} 
          est utilis\'e pour calculer le nom de la g\'eom\'trie selon
           (voir~\ref{DU:Regex});
   \item  si aucun appariement n'est trouv\'e, une erreur se produit;
\end{itemize}

Dans la majorit\'e des cas, {\tt <NomsGeometrieImage>} ne contiendra
qu'un seul \'el\'ement parce que l'association "image/g\'eom\'etrie"
est compl\`etement syst\'ematique.


            % - - - - - - - - - - - - - - - - - - - -

\subsubsection{Nom calcul\'e sur Im1-Im2}
\PourLecteurAverti


\begin{verbatim}
       <NomsGeometrieImage Nb="+">
        ...
            <PatternNameIm1Im2 Nb="?" Type="std::string"> </PatternNameIm1Im2>
            <AddNumToNameGeom Type="bool" Nb="?" Def="false"> </AddNumToNameGeom>
        ...
        </NomsGeometrieImage>
\end{verbatim}

La fonctionnalit\'e d\'ecrite ici est apparue en 
utilisant MicMac lors du calcul automatique de points 
homologues pour l'a\'erotriangulation. Dans ce contexte,
on veut utiliser MicMac de la mani\`ere suivante :

\begin{itemize}
   \item MicMac est appel\'e avec un grand nombre de couples;
         la m\^eme image se retrouvant en g\'en\'eral dans 
          plusieurs couples;

   \item pour chaque couple, MicMac est appel\'e en plusieurs \'etapes; 
         une premi\`ere fois en utilisant en entr\'ee des orientations 
         \emph{a priori}, \'eventuellement assez "grossi\`ere"; 

   \item ensuite MicMac est rappel\'e dans diff\'erentes \'etapes de 
         "raffinement" successif o\`u chaque \'etape g\'en\`ere une nouvelle
          orientation qui servira d'entr\'ees \`a l'\'etape suivante
          (le principe \'etant que, sauf bug, la pr\'ecision des orientations
          relatives s'am\'eliorant \`a chaque \'etape, on peut relancer le
          calcul avec des contraintes de plus en plus fortes sur les parallaxes
          transverses);
\end{itemize}

Pour \'eviter tout conflit de nom entre les nombreux chantiers qui vont
\^etre cr\'e\'es (surtout dans le cas o\`u le calcul est parall\'elis\'e
par un outil de calcul distribu\'e), il importe que chaque calcul
puisse sp\'ecifier ses propres noms de fichiers d'orientation.
Supposons  que les images mises en correspondance
soit {\tt Im1=ImAA.tif} et {\tt Im2=ImBB.tif} et qu'apr\`es l'\'etape $3$,
soit g\'en\'er\'es deux fichiers d'orientation relative, selon le
m\'ecanisme d\'ecrit en~\RefFantome, qui s'appellent 
{\tt Ori\_Etap3\_Im1\_AABB.ori} et {\tt  Ori\_Etap3\_Im2\_AABB.ori};
toujours pour \'eviter les
les conflits de noms, ceux-ci sont choisis de mani\`ere \`a \^etre
diff\'erent du cas "sym\'etrique" {\tt Im2=ImAA.tif}  et {\tt Im1=ImBB.tif}.

Ensuite \`a l'\'etape $4$ il faut pouvoir calculer le nom
des fichier d'orientation en entr\'ee d'une part en prenant en compte 
le nom de {\tt Im1} et {\tt Im2} et d'autre part en pouvant sp\'ecifier
une construction qui soit diff\'erente pour chacune des deux images (la
s\'election \'etant bas\'ee ici sur leur rang et non sur leur nom).
Ces possibilit\'es sont  offertes par  les tags {\tt <PatternNameIm1Im2>}
et {\tt <AddNumToNameGeom>} :

\begin{itemize}
   \item  si le tag {\tt <PatternNameIm1Im2>} est sp\'ecifi\'e , alors
          pour c'est la concatenation {\tt Im1@Im2}  (voir~ \ref{DU:Regex:CatNames})
          qui sera appari\'ee
          sur sa valeur consid\'er\'ee comme une expression r\'eguliere
          qui sera utilis\'ee pour \'etendre {\tt <PatNameGeom>} (mais
          c'est toujours sur l'expression {\tt PatternSel} et sur le nom
          d'une seule image que sera faite la s\'election;

   \item  si le tag {\tt <AddNumToNameGeom>} est  sp\'ecifi\'e \`a vrai,
          alors {\tt <PatNameGeom>} n'est pas appari\'e sur le nom
          tout seul, mais sur une concat\'enation du nom suivi de {\tt @k}
          o\`u {\tt k} est le rang de l'image (en commen\c{c}ant \`a $0$);
\end{itemize}

Dans notre exemple, les bon nom de fichiers pourraient alors
\^etre calcul\'es par quelque chose comme :

\begin{verbatim}
  </Section_PriseDeVue>
      ...
      <NomsGeometrieImage>
            <AddNumToNameGeom> true          </AddNumToNameGeom>
            <PatternSel>    (.*)\.tif@0  </PatternSel>
            <PatternNameIm1Im2>
                    Im(.*)\.tif@Im(.*)\.tif
            </PatternNameIm1Im2>
            <PatNameGeom>  Ori_Etap3_Im1_$1$2.ori  </PatNameGeom>
       </NomsGeometrieImage>

       <NomsGeometrieImage>
            <AddNumToNameGeom> true          </AddNumToNameGeom>
            <PatternSel>    (.*)\.tif@1  </PatternSel>
            <PatternNameIm1Im2>
                    Im(.*)\.tif@Im(.*)\.tif
            </PatternNameIm1Im2>
            <PatNameGeom>  Ori_Etap3_Im2_$1$2.ori  </PatNameGeom>
       </NomsGeometrieImage>
       ...
  </Section_PriseDeVue>
\end{verbatim}

            % - - - - - - - - - - - - - - - - - - - -


\subsubsection{Points homologues}

\begin{verbatim}
       <NomsHomomologues Nb="?">
              <PatternSel Nb="1" Type="std::string">    </PatternSel>
              <PatNameGeom Nb="1" Type="std::string">   </PatNameGeom>
              <SeparateurHom Nb="?" Type="std::string" Def="">   </SeparateurHom>
        </NomsHomomologues>

\end{verbatim}


Ce tag permet de calculer le fichier de points homologues utilis\'e
pour calculer, le cas \'ech\'eant, l'homographie de passage de 
{\tt Im1} \`a {\tt Im2} (voir~\ref{Lec:Param:Geom}). Ce nom
de fichier est cacul\'e par appariement sur la concat\'enation
des noms (voir~\ref{DU:Regex:CatNames}) :

\begin{itemize}
    \item {\tt <PatternSel>} est l'expression sur laquelle est 
         appari\'ee la concat\'enation;

    \item {\tt <PatNameGeom>} est le patron qui g\'en\`ere le nom
          defichier;
    \item {\tt <SeparateurHom>} est un \'eventuel s\'eparateur;

\end{itemize}

Voici un exemple pris d'un param\`etrage pour de la superposition
multi-canal:

\begin{verbatim}
     <Images >
          <Im1 > FD0027_r.047_3113 </Im1>
          <Im2 > FD0027_v.047_3113 </Im2>
     </Images>
  ....
     <NomsHomomologues >
              <PatternSel > FD0027_(.)\.(.*)_(.*)@FD0027_(.)\.(.*)_(.*) </PatternSel>
              <PatNameGeom > Liaison_$1$4.xml  </PatNameGeom>
              <SeparateurHom > @  </SeparateurHom>
      </NomsHomomologues>
\end{verbatim}

Dans cet exemple, le nom de fichier de points homologues 
calcul\'e sera {\tt Liaison\_rv.xml}; l'objectif \'etait
bien d'avoir un nom qui d\'epend de la natures des canaux
 (ici rouge et vert) mais pas du num\'ero des images.


%-------------------------------------------------------------------
%-------------------------------------------------------------------
%-------------------------------------------------------------------

\section{G\'en\'eration de R\'esultat}

Cette section, ainsi que~\ref{Model:Analytik},
d\'ecrit des tags permettant de g\'en\'erer des
r\'esultat interm\'ediaires. Ils sont dans la descendance
de {\tt EtapeMEC} afin de pouvoir, \'eventuellement,
g\'en\'erer ces r\'esultat \`a chaque \'etape dans une
phase de mise au point. Ils sont d\'ecrit ici car ils ne
sont en g\'en\'eral pas logiquement actif sur le processus
de mise en correspondance.


\subsection{Image 8 Bits}

\subsection{Image de corr\'elation}

\subsection{Basculement dans une autre g\'eom\'etrie}

\begin{verbatim}
   <BasculeRes Nb="*">                   
        <Ori Nb="1" RefType="FileOriMnt"> </Ori>
   </BasculeRes>
\end{verbatim}

Ce tag permet de g\'en\'er \`a chaque \'etape un (ou plusieurs) basculement
dans une g\'eom\'etrie diff\'erente de l'acquisition. Le tag {\tt Ori}
est du type d\'efini en~\ref{Def:FOM}.


\subsection{Paralaxe relative ??}

%-------------------------------------------------------------------
%-------------------------------------------------------------------
%-------------------------------------------------------------------

\section{Mod\`eles analytiques}
\label{Model:Analytik}


%-------------------------------------------------------------------
%-------------------------------------------------------------------
%-------------------------------------------------------------------

\section{Section Espace de travail}

\subsection{Directory Image}
\label{DU:Dir:Images}

%-------------------------------------------------------------------
%-------------------------------------------------------------------
%-------------------------------------------------------------------
\section{Section dite "Vrac"}
