\chapter{Sp\'ecification du param\'etrage}

Ce chapitre d\'ecrit quels sont les param\'etrages de MICMAC
qui sont \emph{formellement} valides (ind\'ependemment  de
toute consid\'eration sur la s\'emantique des param\`etres
vis \`a vis de la mise en correspondance).

%-------------------------------------------------------------------
%-------------------------------------------------------------------
%-------------------------------------------------------------------

\section{La ligne de commande}

\label{UTIL:SCPECIF:LIGCOM}

Pour appeler le programme MICMAC l'utilisateur doit sp\'ecifier un
nom de fichier {\tt XML}  (obligatoire) ainsi qu'un nombre
quelconque (\'eventuellement $0$) de param\`etres additionnels.
La syntaxe g\'en\'erale  est :

\vspace{0.2cm}
{\tt MICMAC Fichier.xml Tag$_0$=Val$_0$ 
    Tag$_1$=Val$_1$  \dots   Tag$_N$=Val$_N$ }
\vspace{0.2cm}


Avec :

\begin{itemize}
    \item  {\tt Fichier.xml} nom d'un fichier qui doit contenir
      un fichier {\tt XML} valide;
    \item $N$ couples {\tt (Tag$_k$,Val$_k$)} o\`u {\tt Tag$_k$} est
         le nom d'un {\tt Tag} Xml et {\tt Val$_k$} la valeur
          qui lui est associ\'ee; 
     \item les couples de valeur  permettent, sans toucher au fichier
           {\tt Fichier.xml},  de modifier 
          par la ligne de commande le contenu de la  structure qui
          lui est associ\'ee au lancement du programme;
      \item ce m\'ecanisme de modifications par la ligne de commande
            est utile pour des essai interactifs mais aussi quand MICMAC 
            est appel\'e plusieurs fois par un m\^eme programme  
            \emph{ma\^itre} (c'est le cas lors de la parall\'elisation
            de MICMAC et/ou lorsque MICMAC se rapelle lui-m\^eme);
     
\end{itemize}




%-------------------------------------------------------------------
%-------------------------------------------------------------------
%-------------------------------------------------------------------

\section{La Ressource {\tt ParamMICMAC.xml}}

\subsection{R\^ole de la ressource}

Le fichier {\tt ParamMICMAC.xml} est une ressource qui joue un r\^ole 
important pour MICMAC. Il s'agit d'un fichier {\tt XML} dont le
r\^ole est :

\begin{itemize}
    \item  de g\'en\'erer automatiquement le code de la classe \CPP
           permettant de repr\'esenter et charger en m\'emoire 
           un param\'etrage donn\'e (\emph{binding} en jargon);
           cet aspect de l'utilisation de la ressource  {\tt ParamMICMAC.xml}
           est d\'ecrit \`a partir du chap\^itre~\ref{PROG:GENEAUTO};

    \item  de faire l'essentiel  des contr\^oles de validit\'e sur le
           fichier de param\`etre effectif {\tt Fichier.xml};
           cet aspect int\'eresse le programmeur (pour conna\^itre la limite
           du contr\^ole automatique) mais aussi l'utilisateur car ce
           fichier peut \^etre vu comme une description formelle de
           ce qui fait un fichier de param\`etres syntaxiquement correct;

     \item d'\'etendre l'utilisation du m\'ecanisme de modification par 
           arguments optionnels;
\end{itemize}

             % - - - - - - - - - - - - - - - - - - - -

\subsection{Localisation de la Ressource {\tt ParamMICMAC.xml}}

\label{UTIL:SCPECIF:RES:LOC}

Diff\'erents m\'ecanismes sont propos\'es pour indiquer la
localisation de cette ressource. Par ordre de priorit\'e,
MICMAC utilise les m\'ecanismes suivants pour d\'eterminer
la localisation de la  ressource {\tt ParamMICMAC.xml} :

\begin{itemize}
   \item Si la ligne de commande contient un tag NameFileParamMICMAC, c'est
         sa valeur qui est utilis\'ee pour  d\'eterminer la localisation;

   \item Sinon, si le fichier de param\`etres effectifs {\tt Fichier.xml}
         contient un tag {\tt NameFileParamMICMAC}, c'est sa valeur 
         qui est utilis\'ee pour  d\'eterminer la localisation;

    \item Sinon, si dans l'environnement de l'utilisateur la variable
         {\tt NameFileParamMICMAC} est d\'efinie, c'est sa valeur qui
         est utilis\'ee pour  d\'eterminer la localisation;

    \item Sinon,  MICMAC va rechercher la ressource dans 
          {\tt applis/MICMAC/ParamMICMAC.xml};
\end{itemize}


Pour une
utilisation simple de MICMAC, le dernier m\'ecanisme est suffisant
et ne n\'ecessite aucune intervention de l'utilisateur;
c'est le cas lorsque MICMAC est lanc\'e "\`a
la main" depuis le r\'epertoire  o\`u il a \'et\'e install\'e.

Les trois premi\`eres alternatives correspondent \`a des utilisation
plus \'evolu\'ees.  Par exemple, cela correspond aux cas o\`u MICMAC est
une brique d'un syst\`eme complet et doit pouvoir \^etre rappel\'e par
une application ma\^itresse situ\'ee dans un r\'epertoire diff\'erent.
Ces m\'ecanismes peuvent \^etre aussi utilis\'es lorsque l'on souhaite 
parall\'eliser  MICMAC sur un r\'eseau d'ordinateur avec des outils
de calcul distribu\'e.


%-------------------------------------------------------------------
%-------------------------------------------------------------------
%-------------------------------------------------------------------

\section{Appariement d'arbres}

\label{FR:Mic:Tree:Match}

MICMAC utilise une notion d'appariement d'arbres
pour  sp\'ecifier quels sont les fichiers de param\`etres valides :
l'arbre contenu dans {\tt Fichier.xml} doit \^etre appariable sur 
l'arbre contenu dans la descendance du  tag {\tt <ParamMICMAC>} contenu
dans la ressource {\tt ParamMICMAC.xml}.

Quelques d\'efinitions sont n\'ecessaires pour pouvoir pr\'eciser 
ce que MICMAC consid\`ere comme un appariement d'arbre valide :

\begin{itemize}
   \item les notions d'abres, de fils et de n\oe{}uds terminaux sont suppos\'ees
         connues;
   \item  on appelle arbre de sp\'ecification les arbres de 
          {\tt ParamMICMAC.xml}  et arbre effectif les arbres de  
          {\tt Fichier.xml};
   \item les arbres de sp\'ecification ont tous un attibut dit attribut d'arit\'e;
         cet attribut, d\'esign\'e par le mot-clef $Nb$ peut valoir $1$,
         $?$, $*$ ou $+$; ce mot cl\'e repr\'esente un intervalle d'entier
         selon les conventions classique : $1$=exactement un, $?$= un ou z\'ero,
         $+$=un au moins, $*$= un nombre quelconque;
    
\end{itemize}

Un arbre effectif est appariable sur un arbre de sp\'ecification si et
seulement si :

\begin{itemize}
   \item ils ont le m\^eme nom de tag;
   \item chaque fils, de l'arbre effectif est appariable sur un  fils de l'arbre
         de sp\'ecification;
   \item chaque fils de l'arbre de sp\'ecification est appariable
         sur un ensemble de $N$ fils de l'arbre
         effectif,  o\`u $N$ est compatible avec l'arit\'e de l'arbre de sp\'ecification
         (cette d\'efinition implique que, par exemple, si $Nb=?$, l'ensemble des
           n\oe{}uds  appariables de l'arbre effectif peut \^etre vide);
   \item  lorsque le  n\oe{}ud est terminal, le  n\oe{}ud de l'arbre de sp\'ecification
          porte un attribut {\tt Type} qui impose \'eventuellement des contraintes sur
          la valeur du n\oe{}ud  de l'arbre effectif;
\end{itemize}

On notera que, pour l'instant du moins, l'ordre des fils n'importe
pas pour d\'efinir si deux arbres sont appariables.  Il est 
d\'econseill\'e d'utiliser cette propri\'et\'e (des versions 
ult\'erieure pourront imposer une conservation de l'ordre).

Soit par exemple l'arbre de sp\'ecification suivant :

\begin{verbatim}
<UnTestParam>
     <ListTestCpleHomol Nb="*">
                <PtIm1 Type="Pt2di" Nb="1" > </PtIm1>
                <PtIm2 Type="Pt2di" Nb="1" > </PtIm2>
     </ListTestCpleHomol>
     <NomModule Nb="1" Type="std::string"> </NomModule>
     <NomAux Nb="?" Type="std::string">    </NomAux>
</UnTestParam>
\end{verbatim}



Les deux arbres effectifs ci dessous sont appariables sur l'exemple
pr\'ec\'edent :

\begin{verbatim}

<UnTestParam>
     <ListTestCpleHomol>
           <PtIm1> 1 2 </PtIm1>
           <PtIm2> 2 3  </PtIm2>
     </ListTestCpleHomol>
     <NomModule > UnNom  </NomModule>
     <ListTestCpleHomol>
           <PtIm1> 0 0 </PtIm1>
           <PtIm2> 0 0  </PtIm2>
     </ListTestCpleHomol>
</UnTestParam>

<UnTestParam>
     <NomModule > UnNom  </NomModule>
     <NomAux >   UnAutreNom  </NomAux>
</UnTestParam>

\end{verbatim}

 

L'arbre effectif ci dessous n'est pas  appariable
sur l'arbre de sp\'ecification pour (au moins) les
raisons suivantes :

\begin{itemize}
   \item dans l'arbre de sp\'ecification {\tt <UnTestParam>} 
         n'a pas de fil nomm\'e {\tt  <Toto>} ;
   \item dans l'arbre effectif, le contenu de {\tt <PtIm1>} et
         {\tt <PtIm2>}  ne correspondent pas \`a ce qui est attendu pour
         le type {\tt Pt2di} (ce qui signifie \emph{Point 2D entier},
         et attend de lire deux valeurs enti\`ere);
    \item dans l'arbre de sp\'ecification {\tt <UnTestParam>}
          a un fils nomm\'e  {\tt  <NomModule>}, d'arit\'e $1$ qui
          ne se retouve pas dans l'arbre effectif;
    \item  {\tt <NomAux>} n'a pas la bonne arit\'e ($2$ pour 
           $0$ ou $1$ autoris\'e);
\end{itemize}

\begin{verbatim}

<UnTestParam>
     <Toto> n'a pas d'homologue </Toto>
     <ListTestCpleHomol>
           <PtIm1> 1 2 3 </PtIm1>
           <PtIm2> 2   </PtIm2>
     </ListTestCpleHomol>
     <NomAux >   UnAutreNom  </NomAux>
     <NomAux >   UnAutreNom  </NomAux>
</UnTestParam>

\end{verbatim}

%-------------------------------------------------------------------
%-------------------------------------------------------------------
%-------------------------------------------------------------------

\section{Types des n\oe{}uds terminaux}

\subsection{Types g\'en\'eraux}

Dans l'arbre de sp\'ecification chaque noeud terminal comporte un attribut
de type nomm\'e {\tt Type}. Cet attribut {\tt Type} est utilis\'e, lors de la g\'en\'eration
automatique de code (voir chapitre~\ref{PROG:GENEAUTO}), pour
d\'eterminer le type de la classe  \CPP qui doit \^etre utilis\'ee
pour repr\'esenter la valeur contenue dans le n\oe{}ud appari\'e de 
l'arbre effectif.


Cet attribut {\tt Type}   d\'etermine aussi si la valeur (=cha\^ine de 
caract\`eres) contenue dans un champs est valide. Les type existant et
leur pr\'e-requis sur les valeurs sont :


\begin{itemize}
   \item {\bf \tt std::string} aucune restriction sur la valeur;
   \item {\bf \tt bool} la valeur doit \^etre dans $\{0,1,true,false\}$
         ($0$ \'etant \'equivalent \`a $false$ et $1$ \`a $true$);
   \item {\bf \tt int} un seul entier;
   \item {\bf \tt double} un seul r\'eel;
   \item {\bf \tt std::vector<double>} un nombre quelconque de r\'eel
   \item {\bf \tt Pt2di} deux entiers ($x$ et $y$);
   \item {\bf \tt Pt2dr}  deux r\'eels ($x$ et $y$);
   \item {\bf \tt Pt3dr}  trois r\'eels ($x$, $y$ et $z$);
   \item {\bf \tt Box2dr} quatre r\'eels ($x_0 \ y_0\ x_1\ y_1$);
   \item les types \'enum\'er\'es, qui sont d\'ecrits \`a la section
         suivante, et pour lesquels la valeur doit appartenir \`a 
         l'ensemble des valeurs \'enum\'er\'ees par le type;
\end{itemize}

\subsection{Types \'enum\'er\'es}

Dans le fichier de sp\'ecification {\tt ParamMICMAC.xml},
avant la d\'efinition de l'arbre de sp\'ecification {\tt <ParamMICMAC>},
figurent la d\'efinition des types  \'enum\'er\'es.

La d\'efinition d'un type \'enum\'er\'e, de nom {\tt UnType}, et
pouvant prendre les valeur  {\tt Val-1} {\tt Val-2} \dots {\tt Val-N}
se fait selon la syntaxe :

\begin{verbatim}
  <enum Name="UnType">
         <Val-1> </Val-1>
         <Val-2> </Val-2>
               ...
         <Val-N> </Val-N>
   </enum>

\end{verbatim}

%-------------------------------------------------------------------
%-------------------------------------------------------------------
%-------------------------------------------------------------------

\section{D\'efinition de types d'arbres}

\label{Def:Type:Arbre}

Lorsque un m\^eme type d'arbre est utilis\'e plusieurs fois,
MicMac utilise en g\'en\'eral un m\'ecanisme de d\'efinition
de type qui peut \^etre ensuite r\'ef\'erenc\'e.

Lorsque de la   d\'efinition du type d'arbre , l'attribut {\tt ToReference}
est \`a {\tt true}. Ensuite, lors de l'utilisation ,
l'attribut {\tt RefType} a pour valeur le nom du
type \`a r\'ef\'erencer.

Par exemple on pourrait avoir :

\begin{verbatim}
   <Personne Nb="1" Class="true" ToReference="true">
        <Age Nb="1" Type="double">   </Age>
        <Nom Nb="1" Type="std::string">   </Nom>
   </Personne>
...
   <Club Nb="1">
       <Membre Nb="*" RefType="Personne"> </Membre>
       <NomClub Nb="1" Type="std::string">  </NomClub>
   </Club>
...
\end{verbatim}

Et un utilisation correcte :

\begin{verbatim}
  <Club>
      <NomClub> Les amis de MicMac </NomClub>
      <Membres> <Age> 45 </Age> <Nom> MPD </Nom> </Membres>
      <Membres> <Age> 35 </Age> <Nom> GMaillet </Nom> </Membres>
      <Membres> <Age> 32 </Age> <Nom> DBoldo </Nom> </Membres>
  </Club>
\end{verbatim}
%-------------------------------------------------------------------
%-------------------------------------------------------------------
%-------------------------------------------------------------------

\section{Mode "diff\'erentiel"}

\label{DUG:Mode:Dif}

Un n\oe{}ud de l'arbre de sp\'ecification, d'arit\'e $+$,
peut avoir un attribut {\tt DeltaPrec} \`a $1$. Ce n'est le cas aujourd'hui
que du n\oe{}ud {\tt <EtapeMEC>} correspondant aux \'etapes de la
mise en correspondance, mais celui-ci joue un 
r\^ole essentiel dans le param\'etrage algorithmique.
Lorsque {\tt DeltaPrec=1} (par d\'efaut il vaut $0$), l`\'evaluation
de la liste  \footnote{une liste de valeur car l'arit\'e $+$, 
indique un nombre quelconque $\geq 1$}
des valeurs de l'arbre effectif se fait en mode dit diff\'erentiel.

Le mode diff\'erentiel a \'et\'e ajout\'e  pour tenir compte des situtations
o\`u, en g\'en\'eral, la liste des valeurs est "grande"
avec une forte corr\'elation entre les
valeurs successives. C'est le cas par exemple de 
{\tt <EtapeMEC>} o\`u souvent , d'une \'etape \`a l'autre,
tous les param\`etres algorithmiques sont conserv\'es et 
seule la r\'esolution change. L'objectif est alors d'offrir
un syntaxe, qui sans perdre de g\'en\'eralit\'e, permette dans
les cas les plus courants de ne sp\'ecifier que les attributs qui
diff\`erent de l'\'etape pr\'ec\'edente. Formellement, le fonctionnement
est le suivant :

\begin{itemize}
   \item le premier \'element de la liste effective ne fera pas partie
         du r\'esultat utilis\'e par MICMAC, il constitue l'initialisation
         de la \emph{valeur courante};
    \item pour les \'el\'ements suivant, la valeur rajout\'ee
          est constitu\'ee de la valeur courante, modifi\'ee des 
          fils qui sont explicit\'es dans la liste effective;
     \item les fils explicit\'es dans la liste effective ne
            modifie la valeur courant que si ils ont l'attribut
           {\tt Portee} qui a la valeur {\tt ="Globale"};
\end{itemize}

Soit par exemple l'arbre de sp\'ecification suivant :

\begin{verbatim}
<ListeNMP>
    <NMP  Nb="+" DeltaPrec="1">
        <Num  Nb="1" Type="int">  </Num>
        <Mes  Nb="?" Type="std::string"> </Mes>
        <Pi   Nb="?" Type="double">     </Pi>
    </NMP>
</ListeNMP>
\end{verbatim}

Avec l'arbre effectif de la figure~\ref{DUG:ModDiff:AEff}
c'est la liste  de la figure~\ref{DUG:ModDiff:ResulMem}
qui sera utilis\'ee par MICMAC.


\begin{figure}
\begin{verbatim}
 <ListeNMP>
    <NMP> <Num>-1</Num>  <Pi >3.0</Pi> </NMP>

    <NMP> <Num>0</Num> <Mes>Bonjour</Mes> </NMP>
    <NMP> <Num>1</Num> <Pi Portee="Globale">3.14</Pi> </NMP>
    <NMP> <Num>2</Num> <Mes>Au Revoir</Mes> </NMP>
</ListeNMP>
\end{verbatim}
\caption {Exemple d'utilisation du mode diff\'erentiel }
\label{DUG:ModDiff:AEff}

\end{figure}


\begin{figure}
\begin{verbatim}
     {Num=0 ; Pi=3.0  ; Mes="Bonjour";}
     {Num=1 ; Pi=3.14  ;}
     {Num=2 ; Pi=3.14  ; Mes="Au Revoir";}
\end{verbatim}
\caption {R\'esultat de l'utilisation du mode diff\'erentiel }
\label{DUG:ModDiff:ResulMem}
\end{figure}

On remarque que :

\begin{itemize}

   \item la  premi\`ere valeur n'est pas ajout\'ee dans la liste mais
         a pour effet de donner une valeur initiale \`a {\tt <Pi>};

   \item  la modification de {\tt <Pi>} sur la valeur {\tt Num=1}
          se propage \`a la  valeur {\tt Num=2} car la modification 
          se fait avec l'attribut {\tt Portee="Globale"};

   \item  la modification de {\tt <Mes>} sur la valeur {\tt Num=0}
          ne se propage pas ({\tt <Mes>} est absent de {\tt Num=1});
\end{itemize}


%-------------------------------------------------------------------
%-------------------------------------------------------------------
%-------------------------------------------------------------------
\section{Autres caract\'eristiques}

             % - - - - - - - - - - - - - - - - - - - -

\subsection{Modification par ligne de commande}

\label{DU:Modif:LC}

Le section~\ref{UTIL:SCPECIF:LIGCOM} a indiqu\'e qu'il
\'etait possible dans certain cas de modifier la valeur
des tags par la ligne de commande.

Un arbre du fichier de sp\'ecification, de nom {\tt UnTag}  
est modifiable dans le fichier effectif par la ligne
de commande ssi :

\begin{itemize}
   \item c'est un noeud terminal;
   \item sont attribut d'arit\'e vaut $1$ ou $?$;
   \item tous ses ascendants ont un arit\'e de $1$;
\end{itemize}

Ces r\'egles, un peu restrictives, permettent d'\'eviter 
que la modification soit ambig\"ue ou qu'elle conduise \`a
d\'efinir une syntaxe trop compliqu\'ee (par exemple pour
saisir des structures imbriqu\'ee).
La valeur pass\'ee en ligne de commande \'ecrase celle
qui existait \'eventuellement dans le fichier effectif.

             % - - - - - - - - - - - - - - - - - - - -

\subsection{Valeurs par d\'efaut}

\label{Val:Def}

Lorsqu'un  n\oe{}ud de l'arbre de sp\'ecification 
est terminal, et que son symbole d'arit\'e est $?$,
alors il peut contenir un attribut de nom {\tt Def}
indiquant la valeur par d\'efaut qui sera donn\'ee
\`a ce tag s'il n'est pas sp\'ecifi\'e par l'utilisateur.


\subsection{Fichiers de param\`etres g\'en\'er\'es}


Dans le r\'epertoire  de mise en correspondance, MICMAC
g\'en\`ere deux fichiers xml :

\begin{itemize}
   \item un fichier {\tt Fichier\_Ori.xml}  qui est une simple
         recopie, caract\`ere par caract\`ere, de {\tt Fichier.xml};
         cette copie est faites pour garder une trace des param\`etres
         avec lesquels a \'et\'e lanc\'e MICMAC;

   \item un fichier {\tt Fichier\_Compl.xml}, qui est un {\tt dump}
         de la structure m\'emoire qui a \'et\'e associ\'ee 
         \`a {\tt Fichier.xml};  {\tt Fichier\_Compl.xml} diff\`ere
         de  {\tt Fichier.xml}, notamment \`a cause des valeurs
         par d\'efaut et du mode diff\'erentiel;
\end{itemize}

Ansi avec l'exemple de la figure~\ref{DUG:ModDiff:AEff},
le fichier {\tt Fichier\_Compl.xml} contiendra le texte
de la figure~\ref{DUG:ModDiff:Compl}:

\begin{figure}
\begin{verbatim}
 <ListeNMP>
    <NMP> <Num>-1</Num> <Pi >3.0</Pi> </NMP>
    <NMP> <Num>0</Num>  <Pi>3.0</Pi>  <Mes>Bonjour</Mes> </NMP>
    <NMP> <Num>1</Num>  <Pi>3.14</Pi> </NMP>
    <NMP> <Num>2</Num>  <Pi>3.14</Pi> <Mes>Au Revoir</Mes> </NMP>
</ListeNMP>
\end{verbatim}
\caption {Exemple d'utilisation du mode diff\'erentiel }
\label{DUG:ModDiff:Compl}

\end{figure}


%-------------------------------------------------------------------
%-------------------------------------------------------------------
%-------------------------------------------------------------------




