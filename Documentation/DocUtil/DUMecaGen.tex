\chapter{M\'ecanismes G\'en\'eraux}

Ce chap\^itre d\'ecrit diff\'erents m\'ecanismes g\'en\'eraux \`a
MICMAC qu'il est n\'ec\'essaire de ma\^itriser avant de pouvoir
aborder la sp\'ecification d'un certain nombre de tags.

%---------------------------------------------

\section{G\'en\'eralit\'es et notations}

Cette section, ou un sous-ensemble, sera sans doute transf\'er\'ee 
ult\'erieurement dans la partie algorithmique.

    % - - - - - - - - - - - - - - - - - - - - - - - - - - - - - - - -
    % - - - - - - - - - - - - - - - - - - - - - - - - - - - - - - - -


    % - - - - - - - - - - - - - - - - - - - - - - - - - - - - - - - -

\subsection{Bo\^ites englobantes}

Une bo\^ite  est caract\'eris\'ee  par $P^-=(x^-,y^-)$ et 
$P^+=(x^+,y^+)$, elle d\'efinit la zone du plan  :

\begin{equation}
   [x^- , x^+] \otimes [y^- , y^+]
\end{equation}


Si $E$ est un
ensemble de point, on note $B^x(E)$ la bo\^ite englobante
de $E$.  Si $P_t$ est un point,
on note $B^x(P_t)$ la bo\^ite contenant le singleton  .
On note $\emptyset$ la bo\^ite vide. 

La bo\^ite qui englobe deux boites $B_1$ et $B_2$ est not\'ee
 $B_1+B_2 =B^x(B_1 \cup  B_2)$. On pose $\emptyset+B=B$.

L'op\'eration est \'evidemment associative et commutative et
idempotente. Si $B_k,k \in [1\, N]$ est une collection de
bo\^ites, on pose  :

\begin{equation}
     \sum_{k \in [1\, N]}B_k= B_1+B_2+ \dots +B_N
\end{equation}

Si $B_k,k \in [1\, N]$ est une collection de
bo\^ites, avec $N\geq 2$ on d\'efinit  $\DeuxExtre B_k$
la bo\^ite obtenue en prenant pour valeurs limites
les min et max \`a l'exclusion des valeurs extr\^emes.
Par exemple, le  $x^-$ de cette bo\^ite est la deuxi\`eme plus petite
valeur des $x^-$ des $B_k$.

%---------------------------------------------

\section{G\'eom\'etries}
    % - - - - - - - - - - - - - - - - - - - - - - - - - - - - - - - -

\subsection{G\'eom\'etries intrins\`eque  et de  restitution}

\label{Geom:Intr:Rest}

De mani\`ere g\'en\'erale, la sp\'ecification de  la g\'eom\'etrie 
$\pi_k$ d'une image, se fait sous MICMAC par la description de
deux transformations : 

\begin{itemize}
   \item une transformation  intrins\`eque $\PiI_k$ de $\ETer \otimes \EPx $ 
         dans \EIm;
         cette transformation correspond aux caract\'eristiques propres
         du "chantier" (par exemple poses et calibrations internes des cam\'eras,
         syst\`eme de coordonn\'ees g\'eod\'esique dans lequel sont exprim\'ees
         les poses);
         
          
   \item une transformation de restitution $\pi_T$ de  $\ETer \otimes \EPx $
         dans $\ETer \otimes \EPx $ qui permet d'effectuer les calculs dans
         un espace diff\'erent de celui du chantier;
\end{itemize}

On a la relation :

\begin{equation}
   \pi_k = \PiI_k \circ \pi_T
\end{equation}

Les motivations pour effectuer le calcul de mise en correspondance dans
une g\'eometrie diff\'erente de la g\'eom\'etrie intrins\`eque
peuvent \^etre de deux natures :

\begin{itemize}
   \item  {\bf formatage}; dans le cas d'une prise de vue  a\'erienne,
          par exemple, il est courant que les fichiers d'orientations
          soient donn\'es dans un rep\`ere euclidien local alors que l'on
          souhaite exploiter le MNT r\'esultat dans un ref\'erentiel 
          g\'eod\'esique (lambert ou autre); plut\^ot que de faire un 
          basculement \emph{a posteriori}, il est plus simple et plus
          pr\'ecis de demander \`a MICMAC d'effectuer directement 
          tous les calculs dans le syst\`eme dans lequel seront 
          exploit\'ees les donn\'ees;


   \item  {\bf algorithmiques}; par exemple, lors d'une mise en correspondance
          avec peu d'images (typiquement 2 ou 3), l'exp\'erience et 
          la th\'eorie montrent que l'on obtient des r\'esultats de mise
          en correspondance plus fiables lorsqu'il existe une image
          "ma\^itresse" \footnote{image pour laquelle,
           $\pi_k$ ne d\'ependant pas de la parallaxe}; dans ce contexte il
           peut \^etre int\'eressant de faire les calcul dans une g\'eom\'etrie 
           dans laquelle les "verticales" sont les faisceaux issus d'une
           image \`a sp\'ecifier (l'image "ma\^itresse").

\end{itemize}


Les termes utilis\'es par MICMAC pour d\'esigner ces g\'eom\'etries sont
un h\'eritage des prises de vue a\'erienne standard et sont un peu
impropres dans le cadre g\'eneral :


\begin{itemize}
   \item \emph{g\'eom\'etrie image} pour la  g\'eom\'etrie intrins\`eque;
   \item \emph{g\'eom\'etrie terrain} pour la  g\'eom\'etrie de 
          restitution;
\end{itemize}

    % - - - - - - - - - - - - - - - - - - - - - - - - - - - - - - - -

\subsection{G\'eom\'etrie intrins\`eque (image)}


     %    -   -   -   -   -   -


\subsubsection{Description g\'en\'erale}

\label{DuMeca:Geom:IntrinGene}

Les g\'eom\'etries intrins\`eques actuellement connues par MICMAC sont :

\begin{itemize}
   \item     {\tt   <eGeomImageOri>        } correspondant \`a une g\'eom\'etrie
             st\'enop\'ee stock\'ee selon le format \emph{historique}  {\tt Ori}
             du MATIS;

   \item     {\tt   <eGeomImageModule>     } correspondant \`a une 
             g\'eom\'etrie  sp\'ecifi\'ee par l'utilisateur sous forme de 
             librairie dynamique (permettant de repr\'esenter de mani\`ere 
             g\'en\'erique toutes les   g\'eom\'etries d\'eriv\'ees d'un capteur
             physique par un projection $\RR^3 \rightarrow \RR^2$);

   \item     {\tt   <eGeomImage\_Epip>     } correspondant \`a une 
             g\'eom\'etrie \'epipolaire classique;

   \item     {\tt   <eGeomImageDHD\_Px>    }, {\tt   <eGeomImage\_Hom\_Px>  }
             {\tt   <eGeomImageDH\_Px\_HD> } correspondant \`a diff\'erentes
             variante de g\'eom\'etries qui ne sont pas directement li\'ees 
             \`a une position dans l'espace du capteur;

   \item     {\tt   <eNoGeomIm>            } valeur sp\'eciale, permettant 
             d'utiliser certaine fonctionnalit\'es de MICMAC, non li\'ees
             \`a la mise en correspondance,  ne n\'ecessitant aucune 
             connaissance de la g\'eom\'etrie;

\end{itemize}



\subsubsection{G\'eom\'etrie {\tt   <eGeomImageOri>}}

\label{GEOM:ORI}

Il s'agit de la g\'eom\'etrie st\'enop\'ee classique, soit :


\begin{equation}
   \PiI_k(P) =   Dist^{-1}_k(Pp_k+F_k*\pi_0(\mathcal{R}_k(\mathcal{C}_k -P)))
    \label{Eq:Geom:Ori}
\end{equation}

Avec pour l'image $k$:

\begin{itemize}
   \item   param\`etres extrins\`eques de la cam\'era;
           centre optique $\mathcal{C}_k$ ; matrice de rotation $\mathcal{R}_k$ ;
   \item   $\pi_0$ projection canonique $\pi_0(x,y,z)=\frac{(x,y)}{z}$
   \item   param\`etres intrins\`eques de la cam\'era (souvent ind\'ependant de $k$);
           distance focale $F_k$ ; point principal $Pp_k$; distortion $Dist_k$
     
\end{itemize}

    

    %    -   -   -   -   -   -

\subsubsection{G\'eom\'etrie {\tt   <eGeomImageModule>}}

\label{DU:Meca:Geo:Mod}

 Cette  valeur permet  d'ajouter dans MICMAC n'importe quelle g\'eom\'etrie
d\'ecrivant un syst\`eme imageur caract\'erisable par une fonction de
projection $\RR^3 \rightarrow \RR^2$ et sa fonction "r\'eciproque"
$\RR^3 \rightarrow \RR^2$. 

    Dans ce mode l'utilisateur indique \`a MICMAC le chemin d'acc\`es 
\`a une librairie partag\'ee qui sera charg\'ee dynmamiquement. C'est par cette
libraire que des objets d\'erivant d'une classe abstraite
 {\tt ModuleOrientation} seront cr\'ees. La classe {\tt ModuleOrientation}
d\'efinit dans son interface deux m\'ethode virtuelle permettant de
d\'efinir $\pi$ et $\pi^{-1}$.

   Il exite d\'ej\`a un module adapt\'e au format grille mis en place
par IGN-espace.


    %    -   -   -   -   -   -

\subsubsection{G\'eom\'etries  {\tt <eGeomImage\_Epip>, <eGeomImageDHD\_Px> } \dots }

\label{DuMeca:Geom:Epip}

Ces g\'eom\'etries  permettent de faire de la mise en correspondance
"image \`a image", c'est \`a dire en ignorant la caract\'erisation
g\'eom\'etrique (localisation en 3D) rigoureuse des capteurs imageurs.

Dans ce cas l'espace "terrain" est l'espace de la premi\`ere image.
La parallaxe est toujours bi-dimensionnelle et exprim\'ee en pixel.
Les variantes correspondent  \`a diff\'erents pr\'edicteurs de la
fonction de correspondance \`a priori. La parallaxe \'etant exprim\'ee
en diff\'erentiel par rapport \`a ces pr\'edicteur, cela permet
d'un part de r\'eduire les intervalles de recherche et d'autre part
de corriger des distorsion trop fortes (pour  que les "vignettes" 
de corr\'elation soient r\'eellement superposables).

En notant $D$ des distorsion, $H$ des homographies, 
les fonctions de projection sont les suivantes :


\begin{itemize}
   \item     {\tt   <eGeomImage\_Epip>     }  $\PiI(x,y,u,v) = (x+u,y+v)$ ,
             cette g\'eom\'etrie est donc adapt\'ee \`a la mise en 
             correspondance des images mise en \'epipolaire;
             la parallaxe "transverse" permet, le cas \'ech\'eant,
             de mod\'eliser des imperfection
             dans la g\'eom\'etrie \'epipolaire;

   \item {\tt   <eGeomImage\_Hom\_Px>  }
              $\PiI(x,y,u,v) = H(x,y) +(u,v)$,
              correpondance \emph{a priori} mod\'elis\'ee par une homographie,
              utile  quand on n'a pas mieux (par exemple en  initialisation
              de la "calibration par points de liaisons");

   \item     {\tt   <eGeomImageDHD\_Px>    }
              $\PiI(x,y,u,v) = D_1(H(D^{-1}_2(x,y))) +(u,v)$,
              correpondance \emph{a priori} mod\'elis\'ee par une homographie
              une fois les distorsion corrig\'ees; c'est la mod\'elisation
              utilis\'e pour la
              superposition des canaux de la cam\'era num\'erique;


   \item {\tt   <eGeomImageDH\_Px\_HD> } 
              $\PiI(x,y,u,v) = D_1(H_1 (H^{-1}_2(D^{-1}_2(x,y)) +(u,v)))$;
              g\'eom\'etrie pas encore support\'ee; l'int\'er\^et serait
              de pouvoir mod\'eliser la g\'eom\'etrie \'epipolaire \`a la
              vol\'ee uniquement avec les briques de base  $D$ et $H$;

\end{itemize}


\subsubsection{Lecture des  param\`etres de g\'eom\'etrie}

\label{Lec:Param:Geom}

Pour aller lire les param\`etres de g\'eom\'etrie ,
MicMac fonctionne ainsi :

\begin{itemize}
   \item  {\tt eGeomImageOri} un fichier {\tt ori} par image doit \^etre
          lu,   son nom est calcul\'e \`a partir du tag {\tt <PatNameGeom>}
          (voir~\ref{Assoc:Image:Geom});

   \item  {\tt eGeomImageModule} un fichier par image doit \^etre lu,
          son nom est calcul\'e \`a partir du tag {\tt <PatNameGeom>}
          (voir~\ref{Assoc:Image:Geom}), ce nom de fichier sera pass\'e 
          en argument \`a la fonction contenu dans la librairie   dynamique;

   \item quand une distorsion est n\'ecessaire ({\tt eGeomImageDHD\_Px}),
         celle ci doit \^etre disponible sous forme d'un fichier
         {\tt xml} au format d\'ecrit en~\ref{Format:Grille},  le nom est 
         calcul\'e \`a partir du tag {\tt <PatNameGeom>}; la valeur 
         {\tt GridDistId} est conventionnelle pour signifier une grille
         identit\'e;

   \item quand une homographie est n\'ecessaire 
         ({\tt eGeomImageDHD\_Px} et {\tt eGeomImage\_Hom\_Px}) 
         celle ci est calcul\'ee \`a partir de  points
         homologues fournis par l'utilisateur;
         ces points homologues doivent \^etre contenus dans un fichier
         dont le nom est calcul\'e \`a partir du tag {\tt <NomsHomomologues>}
         et qui doit \^etre au format d\'ecrit dans~\ref{Format:Homol};
         pour passer de $N$ points homologues \`a  une homographie, on
         utilise les conventions suivantes :

       \begin{itemize}
            \item $N=0$ : identit\'e;
            \item $N=1$ : translation;
            \item $N=2$ : similitude;
            \item $N=3$ : affinit\'e;
            \item $N=4$ : homographie exacte;
            \item $N>4$ : homographie ajust\'ee par moinde carr\'es;
       \end{itemize}
         
\end{itemize}


    % - - - - - - - - - - - - - - - - - - - - - - - - - - - - - - - -

\subsection{G\'eom\'etries de restitution (terrain)}



\subsubsection{Description g\'en\'erale}
Les g\'eom\'etrie de restitution actuellement connues par MICMAC sont :

\begin{itemize}
   \item  {\tt <eGeomMNTCarto>  } {\tt  <eGeomMNTEuclid>} :
          g\'eom\'etries terrain classique :
          le relief est restitu\'es en coordonn\'ees cartographiques ou
          dans le rep\`ere euclidien local;


   \item {\tt    <eGeomMNTFaisceauIm1ZTerrain\_Px1D> }
         {\tt    <eGeomMNTFaisceauIm1ZTerrain\_Px2D> } :
         g\'eom\'etries o\`u la premi\`ere image est ma\^itresse; (les
         "verticales" sont les rayons perspectifs de la premi\`ere image);

   \item {\tt   <eGeomMNTFaisceauIm1PrCh\_Px1D> }
         {\tt   <eGeomMNTFaisceauIm1PrCh\_Px2D> } :
         g\'eom\'etries o\`u la premi\`ere image est ma\^itresse avec une
         dynamique en $\frac{1}{z}$;

   \item {\tt  <eGeomPxBiDim>} valeur qui doit \^etre associ\'ee aux 
          g\'eom\'etries images (d\'ecrites en~\ref{DuMeca:Geom:Epip});

   \item{\tt  <eNoGeomMNT> }  valeur sp\'eciale, homologue de
        {\tt   <eNoGeomIm> }, d\'ecrite en~\ref{DuMeca:Geom:IntrinGene}

\end{itemize}


    %    -   -   -   -   -   -

\subsubsection{G\'eom\'etries terrain}

La distinction  entre {\tt <eGeomMNTCarto>  } et {\tt  <eGeomMNTEuclid>} 
n'est utile aujourd'hui que lorsque la g\'eom\'etrie intrins\`eque est conique
fournie au format {\tt ori}.

    %    -   -   -   -   -   -

\subsubsection{G\'eom\'etries {\tt <eGeomMNTFaisceauIm1ZTerrain\_Px1D>}}

Ces g\'eom\'etries de restitution sont accessibles  lorsque la
g\'eom\'etrie intrins\`eque correspond \`a une mod\'elisation $3d$
du capteur ({\tt   <eGeomImageOri> } et {\tt   <eGeomImageModule>}).

Soit $\PiI_1$, la fonction de projection intrins\'eque de la premi\`ere
image, on pose alors  pour {\tt    <eGeomMNTFaisceauIm1ZTerrain\_Px1D> } :


\begin{equation}
   \pi_k(x,y,z) = \PiI_k (\PiI^{-1}_1(x,y,z),z)
\label{Eq:Geom:Fais}
\end{equation}

Une cons\'equence imm\'ediate est que  :

\begin{equation}
   \pi_1(x,y,z) = (x,y)
\end{equation}

Ces formule ont une interpr\'etation intuitive simple :

\begin{itemize}
   \item l'espace de restitution est l'espace des coordonn\'ees de la premi\`ere
         image;
    \item $(\PiI^{-1}_1(x,y,z),z)$ est le point d'altitude $z$ situ\'e sur le rayon 
          perspectif issu de la premi\`ere image au point $x,y$;
    \item ce point est reprojet\'e sur le k\EME image par $\PiI_k$;
\end{itemize}

\subsubsection{G\'eom\'etries {\tt <eGeomMNTFaisceauIm1ZTerrain\_Px2D>}}

Le mode {\tt    <eGeomMNTFaisceauIm1ZTerrain\_Px2D> } est identique au pr\'ec\'edent, 
avec en plus une parallaxe transverse permettant de corriger d'\'eventuelles
impr\'ecisions dans la mod\'elisation du capteur.  Soit $\pi^{1D}_k$ la 
projection associ\'ee \`a {\tt    <eGeomMNTFaisceauIm1ZTerrain\_Px1D> }, on a :

\begin{equation}
   \pi_k(x,y,z,t) = \pi^{1D}_k(x,y,z) + t*(\bar u_k,\bar v_k)
\end{equation}

O\`u $(\bar u_k,\bar v_k)$ est une direction de parallaxe transverse d\'efinie comme
la direction orthogonale \`a la projection dans l'image $k$ du rayon perspectif
issu du centre du "chantier":

\begin{equation}
    (u_k,v_k) = \frac{\partial \pi^{1D}_k(x,y,z)}{\partial z}(x_0,y_0,z_0)
\end{equation}

\begin{equation}
    (\bar u_k,\bar v_k) = \frac{(v_k,-u_k)}{\sqrt{ u^2_k+v^2_k}}
\end{equation}


Pour $k=1$ on conserve la d\'efinition pr\'ec\'edente:

\begin{equation}
   \pi_1(x,y,z) = (x,y)
\end{equation}

    %    -   -   -   -   -   -

\subsubsection{G\'eom\'etries {\tt   <eGeomMNTFaisceauIm1PrCh\_Px?D> } }

Il s'agit de variantes des deux g\'eom\'etries pr\'ec\'edentes 
sp\'ecialis\'ees pour le cas particulier o\`u la g\'eom\'etrie intrins\`eque
est conique et ou il y a \'eventuellement de grandes variations relatives
de la profondeur de champs (sc\`enes terrestres par exemple).  
 Les diff\'erences sont les suivantes :


\begin{itemize}
   \item tout se passse comme si avant d'utiliser l'\'equation~\ref{Eq:Geom:Fais} 
         on se mettait dans un rep\`ere, d'origine le centre optique et 
         o\`u l'axe optique et la verticales
         sont confondus (si bien que $z$ est une profondeur de champ);

    \item la dynamique est en $\frac1z$ ,afin que les variations de  $z$ soient 
          proportionnelles \`a des  parallaxes;
\end{itemize}


Soit $(x,y,z)$ un point de l'espace, $(x,y)$ est un point de l'image $1$,
soit $(\tilde{x},\tilde{y},1)$  la direction du rayon perspectif dans le
rep\`ere image; 
avec les notations de l'\'equation~\ref{Eq:Geom:Ori},
on pose :

\begin{equation}
   \pi_k(x,y,z) = 
    \PiI_k( \mathcal{C}_1 + \mathcal{R}^t_1 \frac{(\tilde{x},\tilde{y},1)}{z})
\end{equation}

Ces g\'eom\'etries sont surtout utilis\'ees
pour le calcul de points homologues denses.


    % - - - - - - - - - - - - - - - - - - - - - - - - - - - - - - - -

\subsection{Caract\'eristiques li\'ees aux g\'eom\'etries}

\subsubsection{Combinaisons de g\'eom\'etries, dimension de parallaxe}

Comme indiqu\'e en~\ref{Geom:Intr:Rest}, MicMac pr\'evoit une approche
"matricielle" pour la description de la g\'eom\'etrie  o\`u la g\'eom\'etrie
finale est obtenue par combinaison d'une g\'eometrie intrins\`eque et d'une
g\'eom\'etrie de restitution.  En pratique, toute les combinaisons 
"intrins\`eque/restitution" ne conduisent pas forc\'ement \`a une g\'eom\'etrie
coh\'erente et ne sont donc pas autoris\'ees. Le tableau~\ref{TAB:COMB:GEOM}
synth\'etise les combinaisons de g\'eometries autoris\'ees et les dimensions
de parallaxes associ\'ees.


\begin{figure}
\vspace{4cm}
\begin{tabular} { c  c  c  c c c c c  c} 
              & \hspace{1mm} \begin{rotate}{45} {\tt eGeomMNTCarto} \end{rotate}   
              & \hspace{1mm} \begin{rotate}{45} {\tt eGeomMNTEuclid} \end{rotate} 
              & \hspace{1mm} \begin{rotate}{45} {\tt eGeomMNTFaisceauIm1ZTerrain\_Px1D} \end{rotate} 
              & \hspace{1mm} \begin{rotate}{45} {\tt eGeomMNTFaisceauIm1ZTerrain\_Px2D} \end{rotate} 
              & \hspace{1mm} \begin{rotate}{45} {\tt eGeomMNTFaisceauIm1PrCh\_Px1D} \end{rotate} 
              & \hspace{1mm} \begin{rotate}{45} {\tt eGeomMNTFaisceauIm1PrCh\_Px2D} \end{rotate} 
              & \hspace{1mm} \begin{rotate}{45} {\tt eGeomPxBiDim} \end{rotate} 
              & \hspace{1mm} \begin{rotate}{45} {\tt eNoGeomMNT} \end{rotate} \\  \hline
          {\tt eGeomImageOri} & 1 & 1 & 1 & 2  & 1 & 2 & X & X \\  \hline
          {\tt eGeomImageModule} & X & 1 & 1 & 2  & X & X & X & X \\  \hline
          {\tt eGeomImageDHD\_Px} & X & X & X & X  & X & X & 2 & X \\  \hline
          {\tt eGeomImage\_Hom\_Px} & X & X & X & X  & X & X & 2 & X \\  \hline
          {\tt eGeomImageDH\_Px\_HD} & X & X & X & X  & X & X & 2 & X \\  \hline
          {\tt eGeomImage\_Epip} & X & X & X & X  & X & X & 2 & X \\  \hline
          {\tt eNoGeomIm} & X & X & X & X  & X & X & X & 0 \\  \hline
\end{tabular}
\caption {Combinaisons autoris\'ees entre g\'eom\'etrie intrins\`eque et
g\'eom\'etries de restitution. Dimensions de parallaxe associ\'ees. X : combinaison interdite.}
\label{TAB:COMB:GEOM}
\end{figure}

On voit que pour les g\'eom\'etries purement images 
(celle d\'ecrites en~\ref{DuMeca:Geom:Epip})  la s\'eparation
en deux g\'eom\'etries n'a pas de sens, c'est pour cela que ces g\'eom\'etries
sont conventionnellement associ\'ees \`a {\tt eGeomPxBiDim}.

    % - - - - - - - - - - - - - - - - - - - - - - - - - - - - - - - -

\subsubsection{Unit\'es de parallaxes}


Une cons\'equence des g\'eom\'etries vari\'ees g\'er\'ees par MicMac
est que les parallaxes ont des signification tr\`es diff\'erentes
suivant les cas. L'unit\'e en laquelle est exprim\'ee la parallaxe 
varie donc en fonction de la g\'eom\'etrie de restitution.
Le tableau~\ref{TAB:DIM:PX} synth\'etise ces variations.



\begin{figure}
\vspace{4cm}
\begin{tabular} { c  c  c c } \\  \hline
 {\bf G\'eom\'etrie}  &  {\bf Unit\'e Px1} & {\bf Unit\'e Px2} & {\bf Unit\'e terrain}\\  \hline
{\tt eGeomMNTCarto} &  M\`etre & x  & M\`etre \\  \hline
{\tt eGeomMNTEuclid} & M\`etre & x  & M\`etre \\  \hline
{\tt eGeomMNTFaisceauIm1ZTerrain\_Px1D} &M\`etre & x  & Pixel\\  \hline
{\tt eGeomMNTFaisceauIm1ZTerrain\_Px2D} &M\`etre & Pixel & Pixel\\  \hline
{\tt eGeomMNTFaisceauIm1PrCh\_Px1D} & M\`etre$^{-1}$ & x & Pixel\\  \hline
{\tt eGeomMNTFaisceauIm1PrCh\_Px2D} & M\`etre$^{-1}$ & Pixel & Pixel\\  \hline
{\tt eGeomPxBiDim} & Pixel&Pixel  & Pixel\\  \hline
{\tt eNoGeomMNT} & x& x  & x\\  \hline
\end{tabular}
\caption {Unit\'es des parallaxe et des coordonn\'ees terrain
 en fonction des g\'eom\'etries de restitution}
\label{TAB:DIM:PX}
\end{figure}


%---------------------------------------------


\newpage
\section{Patrons de s\'election et de transformations de cha\^ines}

\subsection{Utilisation avec un nom}
\label{DU:Regex}

A plusieurs endroits du fichier {\tt <ParamMICMAC.xml>}, MicMac
utilise un m\'ecanisme de  patrons de cha\^ines de caract\`eres.
Cette fonctionnalit\'e  peut  \^etre utilis\'ee soit pour
sp\'ecifier un ensemble de cha\^ines, soit pour sp\'ecifier
un m\'ecanisme d'association automatique entre deux
familles de cha\^ines; ces deux utilisations 
peuvent \^etre conjointes.

Un exemple extrait d'un fichier MicMac :

\begin{verbatim}
     <Images >
          <Im1> ess238.10_229.ng_16b.tif </Im1>
          <Im2> ess238.10_228.ng_16b.tif </Im2>
          <ImPat> .*\.tif </ImPat>
     </Images>
     <NomsGeometrieImage>
          <PatternSel >(.*)\.tif</PatternSel>
          <PatNameGeom > $1.ori </PatNameGeom>
     </NomsGeometrieImage>
\end{verbatim}

Le premier patron, {\tt <ImPat> } est une utilisation en 
s\'election, il permet de rajouter dans la liste des images
du chantier tous les fichiers dont le nom porte l'extension
{\tt ".tif"}.

Le deuxi\`eme  patron, {\tt <PatternSel >} est 
une utilisation en  association qui, coupl\'ee avec 
{\tt <PatNameGeom>}, permet d'indiquer
\`a MicMac que le nom du fichier de g\'eom\'etrie associ\'e
au fichier image s'obtient en rempla\c{c}ant l'extension
{\tt ".tif"} par {\tt ".ori"} ; cela fonctionne notamment parce que {\tt \$k},
dans le pattern de subsitution, signifie "doit \^etre remplac\'e 
par la k\EME expression parenth\`es\'ee du pattern de s\'election".


Ce m\'ecanisme est tout \`a fait standard. Les patrons sont utilis\'es 
selon ce que l'on appelle les expressions r\'eguli\`eres modernes
dont la description d\'etaill\'ee peut \^etre, par exemple,
trouv\'ee en recherchant {\tt regex} dans  les diff\'erentes
pages du manuel {\tt unix}.

    % - - - - - - - - - - - - -

\subsection{Utilisation avec deux noms}

\label{DU:Regex:CatNames}
\emph{Les fonctionnalit\'es d\'ecrites ici sont surtout utiles lorsque
MicMac est rappel\'e par un programme \emph{ma\^itre} sur un grand nombre
de couple d'images.}

Il existe des utilisations MicMac o\`u l'on souhaite que
des noms calcul\'es d\'ependent nom pas seulement du nom
d'une seule image mais de plusieurs (g\'en\'eralement
les images $1$ et $2$). La solution que propose alors
MicMac pour r\'epondre \`a ces besoins est de fournir
une expression r\'eguli\`ere qui sera appari\'ee sur la
concat\'enation des noms {\tt <Im1>} et {\tt <Im2>}.
Pour \^etre s\^ur que l'appariement soit non ambigu
quelque soient {\tt <Im1>} et {\tt <Im2>} on rajoute un
s\'eparateur  entre les noms, ce s\'eparateur peut \^etre
fourni par l'utilisateur, par d\'efaut il vaut "{\tt @}".


\begin{verbatim}
  <NomsGeometrieImage>
            ...
            <PatternNameIm1Im2>
                    DSC_Gray_(.*)\.tif@DSC_Gray_(.*)\.tif
            </PatternNameIm1Im2>
            <PatNameGeom>   OriRelStep1_$1_For_$2.ori </PatNameGeom>
            ...
  </NomsGeometrieImage>

  <CalcNomChantier>
              <PatternSelChantier >
                    DSC_Gray_(.*)\.tif###DSC_Gray_(.*)\.tif
              </PatternSelChantier>
              <PatNameChantier >
                  ChantierHelico2_$1-$2_
              </PatNameChantier>
              <SeparateurChantier > ###  </SeparateurChantier>
  </CalcNomChantier>

\end{verbatim}

L'extrait de fichier {\tt xml} ci-dessous illustre ce 
m\'ecanisme. Appel\'e dans un contexte
o\`u, par exemple, {\tt Im1=DSC\_Gray\_5255.tif} et  
{\tt Im2=DSC\_Gray\_5256.tif}, on aura :

\begin{itemize}
    \item pour {\tt <PatternNameIm1Im2>}, c'est 
         {\tt DSC\_Gray\_5255.tif@DSC\_Gray\_5256.tif}  qui
         va \^etre appari\'e sur {\tt\verb|DSC_Gray_(.*)\.tif@DSC_Gray_(.*)\.tif|},
         {\tt PatNameGeom} g\'en\`erera alors le nom 
         {\tt OriRelStep1\_5255\_For\_5256.ori}; pour ce tag, le s\'eparateur
         est d'office {\tt @};

    \item pour le calcul du nom de chantier, le s\'eparateur est fourni par
          l'utilisateur; ici {\tt <PatNameChantier>} g\'en\`erera
          {\tt ChantierHelico2\_5255-5256\_};
\end{itemize}



%---------------------------------------------
%---------------------------------------------
%---------------------------------------------

\section{Librairies Dynamiques}

\label{Lib:Dyn}

\subsection{Fonctionnement G\'en\'eral}

Afin de pouvoir sp\'ecialiser MicMac, de mani\`ere "fine",
en rajoutant des morceaux de codes, sans avoir \`a les recompiler
 dans l'environnement MicMac, il est offert un 
m\'ecanisme de chargement de librairies dynamiques pour certaines
caract\'eristiques de MicMac.

La mani\`ere d'\'ecrire une librairie dynamique est d\'ecrite dans
la partie \emph{Documentation programmeur}, ici on se limite \`a
la description du m\'ecanisme d'insertion d'un librairie dynamique,
suppos\'ee \'ecrite correctement, via le fichier de param\`etres effectifs.

Pour sp\'ecifier un module charg\'e dynamiquement, il suffit de
de sp\'ecifier deux noms :

\begin{itemize}
   \item un nom de fichier, qui est le nom absolu de la librairie
         partag\'ee qui doit \^etre charg\'ee;

   \item un nom de symbole qui servira de point d'entr\'ee \`a MicMac
         pour aller cr\'eer les objets correspondants une fois
         la librairie charg\'ee (\'etant entendu
         que la m\^eme librairie partag\'ee peut contenir plusieurs 
         services);
\end{itemize}

Actuellement deux fonctionnalit\'es sont accessibles via des librairies
dynamiques :

\begin{itemize}
   \item la g\'eom\'etrie des images ;
   \item la gestion de la pyramide d'image;
\end{itemize}


\subsection{Utilisation pour la g\'eom\'etrie}

A titre d'exemple pour la g\'eom\'etrie :

\begin{itemize}
   \item le tag {\tt <ModuleGeomImage>}, \`a l'int\'erieur de la 
         section {\tt <Section\_PriseDeVue>}, permet de sp\'ecifier
         une librairie dynamique d\'ecrivant la g\'eom\'etrie des
         capteurs;

   \item \`a l'int\'erieur, le tag {\tt <NomModule>} permet de sp\'ecifier
         le nom du fichier contenant la librairie  et le tag  {\tt <NomGeometrie>}
         permet de sp\'ecifier un point d'entr\'ee dans la librairie;

   \item ces valeurs seront recherch\'ees lorsque  la g\'eom\'etrie image
         vaut  {\tt <eGeomImageModule>}.
\end{itemize}

Gr\'egoire Maillet a \'ecrit une librairie dynamique permettant de
prendre en compte le format grille d'IGN-espace; les lignes
ci-dessous sont extraites d'un fichier utilisant ce module
pour faire fonctionner MicMac avec des images au format grille.

\begin{verbatim}
<Section_PriseDeVue >
     <GeomImages> eGeomImageModule </GeomImages>
     <ModuleGeomImage>
         <NomModule>./applis/MICMAC/ModuleGrille/.libs/libmodulegrille.so</NomModule>
         <NomGeometrie> Grille </NomGeometrie>
     </ModuleGeomImage>
     <NomsGeometrieImage>
          <PatternSel >(.*)\.tif</PatternSel>
          <PatNameGeom > $1.GRI </PatNameGeom>
     </NomsGeometrieImage>
       ...
</Section_PriseDeVue>
\end{verbatim}

\subsection{Utilisation pour les pyramides}

Un module pour utiliser le format JPEG-2000 comme format 
de pyramide d'images a \'et\'e \'ecrit. Voir Gr\'egoire Maillet
et/ou Gilles Martinoty pour ce module.

%---------------------------------------------
%---------------------------------------------
%---------------------------------------------

\section{Types r\'eutilis\'es}

Cette section d\'ecrit des types d'arbres qui, apr\`es 
avoir \'et\'e d\'efinis, sont utilis\'es
plusieurs fois  selon le m\'ecanisme d\'ecrit en~\ref{Def:Type:Arbre}.

\subsection{Le type {\tt FileOriMnt}}
\label{Def:FOM}

\begin{verbatim}
   <FileOriMnt Nb="1" Class="true" ToReference="true">
      <NameFileMnt Nb="1" Type="std::string">       </NameFileMnt>
      <NameFileMasque Nb="?" Type="std::string">    </NameFileMasque>
      <NombrePixels   Nb="1" Type="Pt2di">          </NombrePixels>
      <OriginePlani  Nb="1" Type="Pt2dr">           </OriginePlani>
      <ResolutionPlani Nb="1" Type="Pt2dr">        </ResolutionPlani>
      <OrigineAlti  Nb="1" Type="double">           </OrigineAlti>
      <ResolutionAlti Nb="1" Type="double">         </ResolutionAlti>
      <NumZoneLambert  Nb="?" Type="int">           </NumZoneLambert>
      <Geometrie   Nb="1" Type="eModeGeomMNT">      </Geometrie>
      <OrigineTgtLoc Nb="?"  Type="Pt2dr">          </OrigineTgtLoc>
   </FileOriMnt>
\end{verbatim}

Le type {\tt FileOriMnt} permet de d\'ecrire un mod\`ele num\'erique de terrain 
sous la forme d'un fichier image et des m\'eta-donn\'ees associ\'ees.
Il s'agit essentiellement d'un \emph{xml-isation} de l'ancien format
{\tt ori} pour les fichiers MNT.
Il contient:

\begin{itemize}
   \item un nom de fichier image {\tt NameFileMnt} ainsi qu'un nom
         optionnel, {\tt NameFileMasque}, de fichier masque (g\'en\'eralement
	 sur $1$ bit);

   \item soit $I,J$ un pixel du fichier {\tt NameFileMnt}, il est valide
         si $NameFileMasque(I,J) = true$ (toujours valide 
	 si {\tt NameFileMasque} n'est pas sp\'ecifi\'e);
    \item soit $K=NameFileMnt(I,J)$:
    \item   soit $P = OriginePlani+ResolutionPlani*(I,J)$;
    \item   soit $Z = OrigineAlti+ResolutionAlti*K$;
    \item alors le point $(P,Z)$ est un point du MNT dans le syst\`eme de
    coordonn\'ees sp\'ecifi\'e par {\tt Geometrie} ainsi que les param\`etres
    optionnels {\tt NumZoneLambert} et {\tt OrigineTgtLoc}.
\end{itemize}



\subsection{Le type {\tt SpecFitrageImage}}

%---------------------------------------------
%---------------------------------------------
%---------------------------------------------

\section{Gestion des erreurs}

%---------------------------------------------
\subsection{Bugs}


%---------------------------------------------
\subsection{Erreurs mal signal\'ees}

%---------------------------------------------
\subsection{Erreurs catalogu\'ees}
\label{Err:Catal}




