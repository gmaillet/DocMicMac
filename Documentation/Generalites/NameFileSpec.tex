\chapter{Names Convention Organization}

This chapter describes how the user can specify the names convention
and organization and how the different programs communicate
and share the naming. To be honest, the mechanism is certainly more
complex than it could and should be; the idea, when building these
mechanisms, was to be able to integrate any name convention without having
to rename files. This objective is not completely satisfied, but the
complexity is here. It has to be assumed \dots



\label{Chap:NFS}

%=============================================================
%=============================================================
%=============================================================

\section{General Organization}

\subsection{Requirements}

We suppose, in all this document, that all the initial images necessary
for the procces are located in the same directory. Although it may
not be required for all the steps, I cannot guarantee that this can be avoided,
and I am quite sure that it is a good idea to do it this way.

We will call this directory the project directory and
we will refer to it as {\tt ProjDIR}.


\subsection{The \UNCLEAR{struct} {\tt ChantierDescripteur}}

The XML \UNCLEAR{struct} describing the name convention is named {\tt ChantierDescripteur}.
Its formal specification can be found in the file {\tt ParamChantierPhotogram.xml}.
The main parts of a {\tt ChantierDescripteur}  are:

\begin{itemize}
   \item  {\tt KeyedSetsOfNames}, it allows to describe sets of strings, and to give them
          a key identifier that will be used to reference them;

   \item  {\tt KeyedNamesAssociations}, it allows to describe string mapping,  and
          to reference them by key; when these mappings are invertible, the user can
          specify the \UNCLEAR{invert} function %inverted
           (which is often required);

   \item  {\tt KeyedSetsORels}, it allows to describe relations between  strings;
          these relations are always given between existing sets of strings
          (created by a {\tt KeyedSetsOfNames}); of course, to these relations are
          given key identifier used to reference them.
\end{itemize}

{\tt ChantierDescripteur} contains also many other structures, some are 
obsolete, others are not so often used and will be introduced at the end of the chapter.


\subsection{The Files Containing {\tt ChantierDescripteur}}

The programs using {\tt ChantierDescripteur}\footnote{almost all now} expect to
find it in 3 possible locations:


\begin{itemize}

   \item {\tt micmac/include/XML\_GEN/DefautChantierDescripteur.xml}, this file contains the
         predefined conventions; the conventions defined in this file are known \UNCLEAR{from} all the
         tools; the objective is that, in a short future these predefined and "standard"
         conventions should be sufficient in $95\%$  of projects; it is a good idea
         to use these conventions when they exist; \emph{user should never modify this file};
    

   \item {\tt ProjDIR/MicMac-LocalChantierDescripteur.xml}, this file contains conventions
         specific to the project; the name of this file
         must be exactly {\tt MicMac-LocalChantierDescripteur.xml}, this is how the
         tools recognize this special file; the existence of this file is not mandatory
         and simple projects using standard conventions will  omit it;

   \item {\tt DicoLoc}, for some tools (including {\tt MicMac} and {\tt Apero}) it is possible
         to define conventions directly in the paramaters file, this may be convenient if
         a convention that will be used only once is required.

\end{itemize}

\subsection{Overriding and Priority} 

Is was seen before that conventions are given keys (or names), overriding
of key is possible but must obey certain rules:

\begin{itemize}
   \item it is not possible to define twice a key in the same file; for example
         if you try to define two sets, with the same name {\tt Key-My-Set} in 
         {\tt ProjDIR/MicMac-LocalChantierDescripteur.xml}  you will get an error;

   \item it is possible to override a key between different files; the priority rule
         is naturally to privilegiate the more local definition; that is: 

\begin{itemize}
             \item definitions of {\tt DicoLoc} override definitions of 
                    {\tt MicMac-LocalChantierDescripteur.xml};
             \item definitions of {\tt MicMac-LocalChantierDescripteur.xml}
                   override definitions of \newline {\tt DefautChantierDescripteur.xml}.
\end{itemize}

\end{itemize}

It is sometimes useful to override in {\tt MicMac-LocalChantierDescripteur.xml} the
definition of a default key given in {\tt DefautChantierDescripteur.xml} because this is 
the easiest way for parameterizing a tool. For example:

\begin{itemize}
    \item the tool {\tt bin/MpDcraw}, for demosaicing image, expects, in certain 
          condition, to get for each image a "chromatic aberration calibration file";

    \item  for 	a given image, {\tt bin/MpDcraw} computes the name of chromatic calibration file
           using the rule associated to the key {\tt Key-Assoc-Calib-Coul};

    \item  {\tt Key-Assoc-Calib-Coul} is defined in  {\tt DefautChantierDescripteur.xml} 
           with a rule that is convenient for my camera but that may be inapropriate
           for the user, so it may be overrided in \newline {\tt MicMac-LocalChantierDescripteur.xml}.
\end{itemize}


%=============================================================
%=============================================================
%=============================================================

\section{Regular Expression and Substitution}

\subsection{Regular Expression}

Regular expressions are a very powerful tool for concise description of string subsets.
For example the expression:

\begin{verbatim}
                                ^.*_(..?)x[0-9]{3}.*
\end{verbatim}

Means a string that contains a {\tt \_}, followed by 1 or 2 characters,
followed by {\tt x}, followed by $3$ digits and ending by anything.

It is also a standard, in fact \UNCLEAR{as curent} in \UNCLEAR{informatics}, %computer science
 it is a standard with a
lot of variants \dots The regular expressions used are the so called "posix-regular expressions".
This standard has two advantages:

\begin{itemize}
   \item there exists standard library that made the implementation easy for me;

    \item the user can easily find a complete description of regex. On Unix, one
         may type {\tt man regex}. Of course, this is not very didactic and you may
          prefer to search on the web, you will get a lot of interesting pages
          by simply typing "regular expression" on your favorite browser.  Be
          sure to refer to the posix page, and not the other variants
          who may be different (pearl regular expression is a current one).
         
\end{itemize}


\subsection{Substitution}

%=============================================================
%=============================================================
%=============================================================

\section{Describing String Sets}

%=============================================================
%=============================================================
%=============================================================

\section{Describing String Mapping}

\label{Ref:Key:Map}

%=============================================================
%=============================================================
%=============================================================

\section{Describing String Relations}

%=============================================================
%=============================================================

\section{Filters and In-File Definition}





