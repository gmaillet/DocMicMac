\chapter{New "generation" of tools}

This chapter describes some new tools, probably their documentation will be reorganized once they are completely stabilized.


%-------------------------------------------------------------------
%-------------------------------------------------------------------

\section{Fully automatic dense matching}

\subsection{Generalities}

The {\tt C3CD} command is the command that compute automatically a point cloud from a set of oriented images.

\begin{verbatim}
mm3d C3DC -help
Valid types for enum value:
   Ground
   Statue
   TestIGN
   QuickMac
*****************************
*  Help for Elise Arg main  *
*****************************
Mandatory unnamed args :
  * string :: {Type in enumerated values}
  * string :: {Full Name (Dir+Pattern)}
  * string :: {Orientation}
Named args :
  * [Name=Masq3D] string :: {3D masq for point selection}
  * [Name=Out] string :: {final result (Def=C3DC.ply)}
  * [Name=SzNorm] INT :: {Sz of param for normal evaluation (<=0 if none, Def=2 mean 5x5) }
  * [Name=PlyCoul] bool :: {Colour in ply ? Def = true}
  * [Name=Tuning] bool :: {Will disappear soon ...}
\end{verbatim}


The syntax is :

\begin{itemize}
  \item type of matching in enumerated values,

  \item set of images to use

  \item orientation

  \item if {\tt Masq3D} is specified, indicates a 3D masq as created with {\tt SaisieMasqQT};
  \item if {\tt SzNorm} is specified, indicates the window size parameters for normal extraction in ply file
        (usefull for meshing);
  \item if {\tt PlyCoul} is specified, indicates that coloring of points is required.
\end{itemize}


\subsection{Quickmac option}

The {\tt QuickMac} uses the {\tt MMInitialModel} as matcher, which is quite fast on CPU.
As example we use a dataset of $41$ images of a statue, they are presented on figure~\ref{FIG:Angel:Flog}.


\begin{figure}[H]
\begin{center}
\includegraphics[width=120mm]{FIGS/Ange/Panel.jpg}
\includegraphics[width=120mm]{FIGS/Ange/SMALL_MG_1044.JPG}
\end{center}
\caption{The Angel statue used for the {\tt C3DC QuickMac} command}
\label{FIG:Angel:Flog}
\end{figure}


Here is a possible command using a dataset of $41$ images of a statue :


\begin{verbatim}
mm3d C3DC QuickMac _MG_10.*JPG Ori-All2/ Masq3=AperiCloud_All2_selectionInfo.xml
\end{verbatim}


The result are presented on figure~\ref{FIG:Angel:Result}. Computation time was $12$ min with a $8$ processor machine.




\begin{figure}[H]
\begin{center}
\includegraphics[width=100mm]{FIGS/Ange/snapshot00.jpg}
\includegraphics[width=100mm]{FIGS/Ange/snapshot01.jpg}
\includegraphics[width=100mm]{FIGS/Ange/snapshot201.jpg}
\end{center}
\caption{Result of {\tt C3DC QuickMac} command: point cloud, coloured point cloud, meshed poind cloud}
\label{FIG:Angel:Result}
\end{figure}

%-------------------------------------------------------------------
%-------------------------------------------------------------------

\section{Post-processing tools - mesh generation and texturing}

\subsection{Mesh generation}

{\tt TiPunch} command creates a mesh from a point cloud. The point cloud has to be in .ply format and has to store normal direction for each point.
This commands performs two steps:
\begin{itemize}
\item mesh generation
\item mesh filtering
\end{itemize}
Mesh generation is built as a call to PoissonRecon binary from Misha Khazdan (for more information on M. Khazdan's code and research: \href{http://www.cs.jhu.edu/~misha/Code/PoissonRecon/}{http://www.cs.jhu.edu/~misha/Code/PoissonRecon/} )
It has mainly one important parameter: the depth of reconstruction. PoissonRecon solves the Poisson equation with a discretization of space into a voxel grid. The depth $d$ parameter defines the size of the voxel grid, as grid is $2^d$ x $2^d$ x $2^d$ voxels.
As a result, a higher depth will lead to a higher level of detail in the final mesh.\\*

As PoissonRecon can sometimes generate wrong mesh parts, mesh filtering is necessary to deletes parts of the mesh which are too far from point cloud.
Mesh filtering has one parameter: the distance between the input point cloud and the output mesh. For each triangle of the mesh, its distance to the point cloud is tested.
Mesh filtering makes the assumption that point cloud ply has been generated using C3DC command. But one can also use Nuage2Ply (with Normale option) and MergePly to generate compatible point cloud. In this case, you can desactivate mesh filtering, with option Filter=0.

\begin{verbatim}
mm3d TiPunch -help
*****************************
*  Help for Elise Arg main  *
*****************************
Mandatory unnamed args :
  * string :: {Ply file}
Named args :
  * [Name=Pattern] string :: {Full Name (Dir+Pat)}
  * [Name=Ori] string :: {Orientation path}
  * [Name=Out] string :: {Mesh name (def=plyName+ _mesh.ply)}
  * [Name=Bin] bool :: {Write binary ply (def=true)}
  * [Name=Depth] INT :: {Maximum reconstruction depth for PoissonRecon (def=8)}
  * [Name=Rm] bool :: {Remove intermediary Poisson mesh (def=false)}
  * [Name=Dist] REAL :: {Threshold on distance between mesh and point cloud (def=1)}
  * [Name=Filter] bool :: {Filter mesh with distance (def=false)}
  * [Name=Mode] string :: {C3DC mode (def=Statue)}
\end{verbatim}

Syntax is:

\begin{itemize}
  \item ply file, with normal direction for each point
  \item needed if Filter=true, set of images to use to filter mesh (we use depth images computed by C3DC)
  \item needed if Filter=true, orientation folder
  \item Out, output mesh filename
  \item Bin, output mesh ply format (ascii or binary, true means binary)
  \item Depth, Maximum reconstruction depth for PoissonRecon
  \item Rm, remove output of PoissonRecon (mainly if Filter=true)
  \item Dist, distance threshold for mesh filtering
  \item Filter, do we filter mesh
  \item needed if Filter=true, mode of C3DC (needed for PIMs- directory)
\end{itemize}

\subsection{Texturing the mesh}

{\tt Tequila} computes a UV texture image from a ply file, a set of images and their orientations. Ply file has to be a mesh, and can be the result of {\tt TiPunch} (but not the direct result of {\tt C3DC}).\\*

{\tt Tequila} performs six steps:

\begin{itemize}
    \item load data
    \item compute zbuffers
    \item choose which image is best for each triangle
    \item filter mesh according to visibility
    \item write UV texture
    \item write ply file with uv texture coordinates\\*
\end{itemize}

Choosing which image is best for each triangle can be done for the moment with two different criterions:
\begin{itemize}
\item   best angle between triangle normal and image viewing direction (parameter Crit=Angle, by default)
\item   best stretching of triangle projection in image (parameter=Stretch)\\*
\end{itemize}

For the angle criterion, expressed in degrees, a threshold is set to avoid using images that view a triangle with a low incidence (parameter Angle).
It means that if the angle between triangle normal and image viewing direction is lower than $Angle$, the image will not be used for texturing.\\*

{\tt Tequila} has also two modes, which refer to texture computing strategies: $basic$ and $pack$. In the $basic$ mode, all images from the set are stored in the uv texture, and if necessary are downscaled. Each image is masked with the zbuffer, to store a minimum of significant information.
In the $pack$ mode, each image is divided in small regions, and only useful regions are packed into the uv texture, in an optimal way. This mode leads to smaller images, and gives better texture quality.

\begin{verbatim}
mm3d Tequila -help
*****************************
*  Help for Elise Arg main  *
*****************************
Mandatory unnamed args :
  * string :: {Full Name (Dir+Pat)}
  * string :: {Orientation path}
  * string :: {Ply file}
Named args :
  * [Name=Out] string :: {Textured mesh name (def=plyName+ _textured.ply)}
  * [Name=Bin] bool :: {Write binary ply (def=true)}
  * [Name=Texture] string :: {Texture name (def=plyName + _UVtexture.jpg)}
  * [Name=Sz] INT :: {Texture max size (def=4096)}
  * [Name=Scale] INT :: {Z-buffer downscale factor (def=2)}
  * [Name=QUAL] INT :: {jpeg compression quality (def=70)}
  * [Name=Angle] REAL :: {Threshold angle, in degree, between triangle normal and image viewing direction (def=90)}
  * [Name=Mode] string :: {Mode (def = Pack)}
  * [Name=Crit] string :: {Texture choosing criterion (def = Angle)}
\end{verbatim}

Relevant parameters are:
\begin{itemize}
\item Angle, threshold for minimum angle between normal and viewing direction (if Crit=Angle)
\item Mode, choose between Basic and Pack (see upper)
\item Crit, choose between Angle and Stretch (see upper)
\item Scale, which allow to speed up computation (higher downscale factor leads to faster computation).
\item Sz, which will force texture size, to conform with graphic card capacity (see $GL_MAX_TEXTURE_SIZE$)
\item QUAL, the jpeg compression quality, which allows to compact UV texture image.
\end{itemize}

\begin{figure}[H]
\begin{center}
\includegraphics[width=110mm]{FIGS/Ange/Tequila.jpg}
\end{center}
\caption{The Angel statue mesh textured with {\tt Tequila} command}
\label{FIG:Angel:Tequila}
\end{figure}
%-------------------------------------------------------------------
%-------------------------------------------------------------------

\section{Parallelizing Apero}

For now works only with linear orientation.

\subsection{Parallelizing Apero}

The new tool {\tt Liquor} (for LInear QUick ORientation) accelerate the computation of orientation. The acceleration comes
from two aspects:

\begin{itemize}
   \item  it uses a hierarchical building of orientation, which make the computation in $N Log N$ instead of $N^2$
   \item  at the low level of the pyramid, it parallelizes the computation of subset on the several processors.
\end{itemize}

The syntax :

\begin{verbatim}
mm3d Liquor -help
*****************************
*  Help for Elise Arg main  *
*****************************
Mandatory unnamed args :
  * string :: {Full name (Dir+Pat)}
  * string :: {Calibration Dir}
Named args :
  * [Name=SzInit] INT :: {Sz of initial interval (Def=50)}
  * [Name=OverLap] REAL :: {Prop overlap (Def=0.1) }
\end{verbatim}

An example of use with the data set of figure~\ref{FIG:Liquor:DataMap} :

\begin{verbatim}
mm3d Liquor CAM2_0.* Ori-Calib/
\end{verbatim}


With these $150$ images the computation time is $20 min$ instead of $1h10min$ with traditional {\tt Tapas}.

\begin{figure}[H]
\begin{center}
\includegraphics[width=100mm]{FIGS/Ange/LineAcq.jpg}
\end{center}
\caption{Linear acquisition used for {\tt Liquor} command}
\label{FIG:Liquor:DataMap}
\end{figure}


