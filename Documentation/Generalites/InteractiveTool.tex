\chapter{Interactive tools}

\section{Generalities}

%Generate with make input.

%=======================================
\section{Entering mask with {\tt SaisieMasq} or {\tt SaisieMasqQT} }

\subsection{SaisieMasq}

SaisieMasq is a very simple tool to edit mask images.
It creates a binary mask image from a polygonal selection in the displayed image.\\*

Typing {\tt SaisieMasq -help}, one gets:

\begin{verbatim}
*****************************
*  Help for Elise Arg main  *
*****************************
Unamed args :
  * string :: {Name of input image}
Named args :
  * [Name=SzW] Pt2di
  * [Name=Post] string
  * [Name=Name] string :: {Name of result, default toto->toto_Masq.tif}
  * [Name=Gama] REAL
  * [Name=Attr] string
\end{verbatim}

Meaning of args is:

\begin{itemize}
   \item First arg: pattern specifying the images to load (can be 1 or more images) - regular expression are supported;
   \item optional {\tt SzW}, def = [$900,700$], size of display window;
   \item optional {\tt Post}, def = "Masq" , postfix to add to output filename;
   \item optional {\tt Name}, name of result;
   \item optional {\tt Gama}, def = $1.0$ , gama applied to images, it can help with dark images, or wide dynamics;
   \item optional {\tt Attr}, text to add to postfix;
\end{itemize}


Processing is as follow:
\begin{itemize}
\item Click: add a point to polygon
\item Shift click: close polygon and apply selection
\item Ctrl + right click: delete last point
\item Shift + right click + Coul : switch between add mode and remove mode
\item Shift + right click + Exit : save mask image and {\tt Xml} file and quit
\end{itemize}

\subsection{SaisieMasqQT}

SaisieMasqQT is the same tool as SaisieMasq, available on all platforms (Linux, Windows, MacOS).
SaisieMasqQT should be run with command: mm3d SaisieMasqQT + arguments.\\*

Example: mm3d SaisieMasqQT IMG.tif SzW=[1200,800] Name=PLAN Gama=1.5\\*

SaisieMasqQT has the same arguments as SaisieMasq. Some light differences with SaisieMasq processing workflow should be noticed:
you need to draw a polygon first, and then apply an action (add to mask, remove from mask, etc.).
You can get a complete list of possibles actions typing F1 when application is launched.\\*

Main actions are:
\begin{itemize}
\item Left click: add a point to polygon
\item Right click: close polygon
\item Space: Add to mask
\item Suppr: Remove from mask
\item Right click (close to a point): delete point
\item Echap: delete polygon
\item Shift + click: insert point
\item Ctrl+S : save mask image and {\tt Xml} file
\item Ctrl+Q : quit
\end{itemize}

NB:
\begin{itemize}
\item 1: SaisieMasqQT can run without any argument (open files from File menu, or drag \& drop them)
\item 2: visual interface for argument edition available with command: mm3d vSaisieMasqQT
\item 3: SaisieMasqQT can display ply files and camera orientation files
\item 4: Typing {\tt mm3d SaisieMasqQT -help}, one gets help message.
\end{itemize}

%========================================

\section{Entering points}

 %  -  -  -  -  -  -  -  -  -  -  -  -  -
\subsection{Generalities}


To move into an image, various solutions are proposed in the interface:
\begin{itemize}
\item Click on wheel + move = drag
\item Shift + wheel + vertical move = quick zoom
\item Shift + wheel + horizontal move = slow zoom
\item Wheel roll = zoom
\end{itemize}

\vspace{\baselineskip}
To input points, some menus can be displayed with these shortcuts:
\begin{itemize}
\item Right-click: geometry menu
\item Shift + left-click: info menu
\item Shift + right-click: undo menu
\item Ctrl + right-click: zoom menu
\end{itemize}

\subsubsection{Geometry menu}

This menu can be shown with a right-click:

\begin{figure}[H]
\begin{center}
\includegraphics[width=95pt]{FIGS/Saisie/geometry.png}
\end{center}
\label{FIG:button1}
\end{figure}

The corresponding actions are:
\begin{itemize}
\item ;-) validate closest point;
\item (/) invalidate closest point;
\item ? : set point status to dubious
\item skull: don't use closest point
\item HL: highlight point
\item empty box: escape menu (do nothing)
\end{itemize}


\subsubsection{Info menu}

This menu can be shown with Shift + left-click:
\begin{figure}[H]
\begin{center}
\includegraphics[width=154pt]{FIGS/Saisie/info.png}
\end{center}
\label{FIG:info}
\end{figure}

The corresponding actions are:
\begin{itemize}
\item Pts: select or add a name for this point
\item Min3:
\item Min5:
\item Max3:
\item Max5:
\item skull: delete the point in all images (needs a confirmation)
\item empty box: escape menu (do nothing)
\end{itemize}

\subsubsection{Undo menu}

This menu can be shown with Shift + right-clic:

\begin{figure}[H]
\begin{center}
\includegraphics[width=95pt]{FIGS/Saisie/contextual.png}
\end{center}
\label{FIG:contextual}
\end{figure}

The corresponding actions are:

\begin{itemize}
\item Exit: quit the interface, saving {\tt Xml} files
\item Undo: undo last action
\item Redo: redo last action in history
\item Ref: display or not refuted points
\item NoD/Ret: display or not the points names
\item empty box: escape menu (do nothing)
\end{itemize}

\subsubsection{Zoom menu}

This menu can be shown with Ctrl + right-click:

\begin{figure}[H]
\begin{center}
\includegraphics[width=52pt]{FIGS/Saisie/zoom.png}
\end{center}
\label{FIG:zoom}
\end{figure}

The three corresponding actions are:

\begin{itemize}
\item {\tt All W}: full zoom in all windows, and show images where points have not been measured yet;
\item {\tt This W}: zoom only in the window where the menu has been displayed;
\item {\tt This Point}: zoom on the nearest point in all windows where the point is visible
\end{itemize}


 %  -  -  -  -  -  -  -  -  -  -  -  -  -
\subsection{For initial GCP  with {\tt SaisieAppuisInit}}
\label{SaisieAppuisInit}
 %  -  -  -  -  -  -  -  -  -  -  -  -  -

This section describes {\tt SaisieAppuisInit} the graphic interface to input 2D and 3D coordinates of ground control points.

For example with the Saint-Michel de Cuxa data set \ref{Cuxa:DataSet}:

\begin{verbatim}
SaisieAppuisInit  "Abbey-IMG_(021[12]|023[3456]).jpg"  All-Rel  NamePointInit.txt  MesureInit.xml
\end{verbatim}

When running this command, the interface shows data set's first images, where one can point GCPs:

\begin{figure}[H]
\begin{center}
\includegraphics[width=150mm]{FIGS/Saisie/interface.jpg}
\end{center}
\caption{SaisieAppuis interface for ground control point selection}
\label{FIG:SaisieAppuis:interface}
\end{figure}

The general process for inputting ground control points is:
\begin{itemize}
\item Input a point in an image (Left-click)
\item Select its name,
\item Input the same point in the other images: move the yellow point and validate it with (right-clic + ;-) )
\item Iterate on each point you want to add (at each iteration, it can be useful after having pointed the point in one image to zoom on this point in all the images,
this can be done by (Ctrl + right-click + {\tt This Point})
\end{itemize}

When exiting the interface, two {\tt Xml} files are stored, with respectively 2D and 3D coordinates of input points.
Note that if for some reason some points are missing, you can re-run the same command, and continue the input job.
Points that have already been stored will be displayed, and the same process can be followed.\\

SaisieAppuisInit is available on Linux and MacOS.
An equivalent tool is available on Windows, Linux and MacOS and is called with command: \begin{verbatim} mm3d SaisieAppuisInitQT + arguments \end{verbatim}
It runs with the same arguments as SaisieAppuisInit. For example :
\begin{verbatim}
mm3d SaisieAppuisInitQT  "IMG_(023[3456]).jpg" All NamePoint.txt  Mesure.xml
\end{verbatim}

Same equivalent tools exist for SaisieAppuisPredic and SaisieBasc (ie. mm3d SaisieAppuisPredicQT and mm3d SaisieBascQT).


\subsection{For fast predictive entering GCP with {\tt SaisieAppuisPredic}}

When enough points have been selected, interface can give a prediction for each new input:

\begin{figure}[H]
\begin{center}
\includegraphics[width=120mm]{FIGS/Saisie/prediction.jpg}
\end{center}
\caption{Prediction help for adding new point}
\label{FIG:SaisieAppuis:prediction}
\end{figure}

 %  -  -  -  -  -  -  -  -  -  -  -  -  -
\subsection{For bascule with {\tt SaisieBasc}}
\label{SaisieBasc}

{\tt SaisieBasc} is a graphic interface to measure objects
to be able to perform transformations such as data scaling, rigid transformation (rotation, translation).\\*

One can size a point to set the origin of the new frame.
One can size two lines:
\begin{itemize}
\item one to set horizontal (with two points: Line1, Line2)
\item one to set scale (with two points: Ech1, Ech2)
\end{itemize}


%=======================================

\section{Visualize Tie-points with {\tt SEL}}

An old and ugly tool, but it can help. To visualize tie points computed with
{\tt Tapioca} :

\begin{verbatim}
SEL  ./ Face2-IMGP5331.JPG Face2-IMGP5333.JPG KH=NB
\end{verbatim}

For creating a few set of tie points and save in XML format :

\begin{verbatim}
SEL  ./ Face2-IMGP5331.JPG Face2-IMGP5333.JPG KH=S
\end{verbatim}



