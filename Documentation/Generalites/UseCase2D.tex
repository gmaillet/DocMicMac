\chapter{Use cases for $2D$ Matching}

The chapter covers examples of using MicMac when the matching problem is
a $2$ dimensionnal one. This can occur in the following situation:


\begin{itemize}
   \item the problem is intrinsically $2$ dimensionnal, for example in
         movement detection (see~\ref{Gulya}); this can be done with a simplified tool;

   \item the problem should be $1$ dimensionnal, but the orientation parameters
         are unknown (see~\ref{Mars}) or, at least, "very" unaccurate; at the time being,
         this requires a parametrization of {\tt MICMAC} with {\tt XML} file;

   \item the problem should be $1$ dimensionnal, the orientation parameters have been computed,
         but for some reason, there are doubts on their accuracy and the user want to check
         this accuracy (see~\ref{CheckOri});  this can be done with a simplified tool;
\end{itemize}


%-------------------------------------------------------------------
%-------------------------------------------------------------------
%-------------------------------------------------------------------

\section{Checking orientation}
\label{CheckOri}

\begin{figure}
\begin{center}
\includegraphics[width=45mm]{FIGS/TestOri/DraixIl.JPG}
\includegraphics[width=45mm]{FIGS/TestOri/DraixPx1.jpg}
\includegraphics[width=45mm]{FIGS/TestOri/DraixPx2.jpg}
\end{center}
\caption{Image, deptht map and transverse paralaxe with draix data set
(images P4090163.JPG and P4090134.JPG)
}
\label{FIG:Draix:PxTr}
\end{figure}



\begin{figure}
\begin{center}
\includegraphics[width=45mm]{FIGS/TestOri/CuxIm.jpg}
\includegraphics[width=45mm]{FIGS/TestOri/CuxPx1.jpg}
\includegraphics[width=45mm]{FIGS/TestOri/CuxPx2.jpg}
\end{center}
\caption{Image, deptht map and transverse paralaxe with MiniCuxa data set
(images Abbey-IMG\_0208.jpg and Abbey-IMG\_0209.jpg), the correlation between two paralax
is lightly visible
}
\label{FIG:Cuxa:PxTr}
\end{figure}

\begin{figure}
\begin{center}
\includegraphics[width=60mm]{FIGS/TestOri/FullCuxPx1.jpg}
\includegraphics[width=60mm]{FIGS/TestOri/FullCuxPx2.jpg}
\end{center}
\caption{deptht map and transverse paralaxe with full resolution Cuxa images,
correlation between both is clearly visible, amplitude is $\pm 1$ pixel.
}
\label{FIG:CuxaFull:PxTr}
\end{figure}


\begin{figure}
\begin{center}
\includegraphics[width=60mm]{FIGS/TestOri/Mun1.jpg}
\includegraphics[width=60mm]{FIGS/TestOri/Mun2.jpg}
\end{center}
\caption{deptht map and transverse paralaxe with $10$ cm image of Munich, acquired with a DMC,
except in "noisy part" like the river, amplitued of transverse parals is $\pm\, 0.1$ pixel
}
\label{FIG:Mubich:PxTr}
\end{figure}




In image geometry {\tt MicMac} has "special" modes where the matching can be done
taking into account a possible unaccuracy of the orientation. Although, it can be used
to match badly oriented images, this is generally not a good idea (it's a better idea to
understand what was wrong in orientation or acquisition and to correct it !!). However,
when the user has doubts on orientation parameters, these tool can  be convenient to check
these orientations. In these mode :

\begin{itemize}
   \item the matching is done in image geometry : there is a master image, and the $X,Y$
         are the pixel of this master image;
   \item there is \emph{only} one secondary image;
   \item for each pixel of the master image, two value are computed, one represents the depths
         and the other represents the "transverse paralax" : it is the displacement in the direction
         orthogonal to the epipolar;
\end{itemize}

These mode can be fairly complex to use directly in {\tt XML} mode, so it's generaly sufficient
to use the simplied tool {\tt MMTestOrient}. The  arguments should be quite obvious from
inline help :

\begin{verbatim}
mm3d MMTestOrient -help
*****************************
*  Help for Elise Arg main  *
*****************************
Unnamed args : 
  * string :: {First Image}
  * string :: {Second Images}
  * string :: {Orientation}
Named args : 
  * [Name=Dir] string :: {Directory, Def=./}
  * [Name=Zoom0] INT :: {Zoom init, pow of 2  in [128,8], Def depend of size}
  * [Name=ZoomF] INT :: {Zoom init,  pow of 2  in [4,1], Def=2}
\end{verbatim}

The result of transverse paralaxes in stored in images {\tt Px2\dots}, the number of the last
and most accurate image depends of the other parameters, so you have to check what is 
present on the directory {\tt GeoI-Px}. 

How can these image be used ? Basically, the idea is that with a "perfect" orientation the 
transverse paralax should be zero on all the image. In real life, this is more complicated, because this
paralax can be noisy ($2d$ general matching problem can be fairly ambiguous). So what is important is not
only the amplitude of the transverse paralax but also it spatial analysis : is there systematism in
this paralax ? Does it present low frequency movement ? Is this transverse paralax correlated to the
depth map ? \dots 
It is not so easy to make an automatic quantitative analyse 
of the results and the firt purpose of this tool is to help human expertise in a qualitative analysis of
the result.  The {\tt MMTestOrient} is illustrated on three examples (in each case with {\tt ZoomF=1}) :

\begin{itemize}
   \item on figure~\ref{FIG:Draix:PxTr} , with image from the Draix data set; in this case
          the transverse paralax is a bit noisy but does not show  obvious systematism;

   \item on figure~\ref{FIG:Cuxa:PxTr} , the amplitude of transverse paralax do not seem
         very high, but here it is computed on reduced images, and conversely one can guess
         some systematism and a correlation between the depth and the transverse paralax;

   \item on figure~\ref{FIG:CuxaFull:PxTr} ,  the full resolution image of Cuxa  have been 
         used (they are not in the data-set); in this case, the tool show clearly a high systematism in
         the transverse paralax  , if we except the noisy part like the tree ~\footnote{for tree the
         transverse paralaxe can be created by wind} the amplitude in almost $\pm\,1$pixel
         between highest and lowest value; furthermore the high correlation bewteen two paralax
         maybe originated by a calibration problem , probably due to focal length;

   \item figure~\ref{FIG:Mubich:PxTr} present an almost perfect orientation; with these $14144,15552$
         coming from a $DMC$ camera the amplitude of transverse paralax is $\pm\, 0.1$ pixel on most
         of the image; the only part of the image where the amplitude is significatively higher is
         the river, but as can be seen on the depth image, the matching is very noisy here and the 
         result has meaning in such part;

\end{itemize}


%-------------------------------------------------------------------
%-------------------------------------------------------------------
%-------------------------------------------------------------------


\section{The Mars data-set}

\label{Mars}



\begin{figure}
\begin{center}
\includegraphics[width=35mm]{FIGS/Mars/SmaIm1.jpg}
\includegraphics[width=35mm]{FIGS/Mars/SmIm2.jpg}
\includegraphics[width=35mm]{FIGS/Mars/Px1.jpg}
\includegraphics[width=35mm]{FIGS/Mars/Px2.jpg}

\end{center}
\caption{Mars data-set : the two  images, the $X$ parallax, in gray-level, and the $Y$-parallax in
hue colour}
\label{FIG:OK:Mars}
\end{figure}



\subsection{Description of the data set}

The data can be found in the directory {\tt Mars/} of directory {\tt ExempleDoc/}.
It consists of two stereo images acquired by Cassini (??) probe of the planet Mars. In this case
we do not have the physicall model of the sensor, but we know that:

\begin{itemize}
   \item the satellite is a  pushbroom-satellite;
   \item it flights in the $x$ direction.
\end{itemize}

\subsection{Comment on the paramaters}

\subsubsection{Geometry}

The tags controlling geometry are:

\begin{itemize}

   \item   {\tt <GeomImages> eGeomImage\_Hom\_Px </GeomImages>} indicates the geometry of the acquisition,
          here it means that there is a principal homography $H$, let $P_1=x_1,y_1$ and  $P_2=x_2,y_2$ be two
          homologous points, MicMac will compute $U(P_1)$ and $V(P_1)$ such that

\begin{equation}
    P_2 = H(P_1) + (U(P_1),V(P_1))
\end{equation}

   \item  the homography $H$ is computed by MicMac from a set of homologous point;

   \item  {\tt  <FCND\_CalcHomFromI1I2> NKS-Assoc-CplIm2Hom@-Man@xml  </FCND\_CalcHomFromI1I2>}  indicates
          where {\tt MicMac} must look for the tie points (see directory {\tt Homol-Man/});


   \item  {\tt <GeomMNT> eGeomPxBiDim  </GeomMNT>} indicates the geometry of restitution,
          the value {\tt eGeomPxBiDim} indicates that what is computed is the pixel offset, in fact this value
          is mandatory when using {\tt eGeomImage\_Hom\_Px}


\end{itemize}

\subsubsection{Matching}

In this case, the two parallax directions have completely different meanings:

\begin{itemize}
   \item the parallax $1$ represents mainly the relief, it is expected to contain high frequencies;
   \item the parallax $2$ represents mainly the error of the geometric model, it is expected to have
         low amplitude and low frequencies;
\end{itemize}

This asymmetry in the \emph{a priori} knowledge of parallax is specified at different parts of the file :

\begin{itemize}
   \item {\tt <Px1IncCalc>}   and {\tt <Px2IncCalc>},  representing the global uncertainty on each parallax;
   \item {\tt <Px1Regul>}   and {\tt <Px2Regul>},  representing the \emph{a priori} knowledge of the regularity of each
         parallax;
   \item {\tt <Px1PenteMax>}   and {\tt <Px2PenteMax>},  representinng the \emph{a priori} knowledge of the
         \UNCLEAR{steep} of each parallax;
   \item {\tt <Px1Pas>}   and {\tt <Px2Pas>},  representing the discretization step (as Px2 is low frequency and
         low amplitude, we can compute it with higher precision);
   \item {\tt <Px1DilatAlti>}   and {\tt <Px2DilatAlti>}, to gain some time, we decide not to re-estimate Px2 at the last step;
\end{itemize}

\subsubsection{Results}

Figure~\ref{FIG:OK:Mars} presents the two images and the results of the computed parallax.
As expected:

\begin{itemize}
   \item The Px1 contains mainly high frequency information on the relief;
   \item The Px2 contains mainly low frequency information on  the geometry of the sensor.
\end{itemize}


%-------------------------------------------------------------------
%-------------------------------------------------------------------



\section{The Gulya Earthquake Data-Set}

\label{Gulya}

\subsection{Introduction}

Since september $2011$ CNES\footnote{Centre National d'Etudes Spatiales, the French spatial agency}
has been funding a development for using MicMac for  earthquake quantification. This development
was made as a collaboration between CNES, CEA, IPGP and IGN/ENSG. The main developer of this part
is Ana-Maria Rosu.

Although there exist other tools for doing this, the objective was:

\begin{itemize}
   \item have a totally free tool, that scientists can use in open source mode;
   \item have a more parametrizable tool;
\end{itemize}

Although the study is not finished, the tool  is now operational. The program has been
tested on $3$ real data set and several synthetic data set, and compared to several existing solutions 
working in frequency domain. From a purely subjective evaluation, these tests show that the
results with MicMac are generally equivalent in quality to frequency approach and, on one of the
real data-set, the results of MicMac where "better"\footnote{i.e. subjectively easier to intepret
for scientist} . One of the drawback of the dense approach of
MicMac is the computation time : $15$ minutes, with a $8$ core computer, with the $1600*3600$
images of the Gulya data-set.


%-------------------------------------------------------------------

\subsection{Description of the data set}

The data can be found in the directory {\tt {SeismGuyla/}} 
 of directory {\tt ExempleDoc/}. % micmac_data/ExempleDoc/
It consists of  two \emph{Spot 5} ortho photos of the same scene taken in $2002$ and
$2008$. Between these two dates, an earthquake occured and image matching can be used to
localize the rupture and quantify the movement.

We want to use MicMac to measure very small displacements (arround $\frac{1}{10}$ pixel) in
a context where the images are quite different. Figure~\ref{FIG:OK:Guylia} presents the two
ortho images.

\begin{figure}
\begin{center}
\includegraphics[width=35mm]{FIGS/SeismGuylia/250802_ortho.jpg}
\includegraphics[width=35mm]{FIGS/SeismGuylia/260608_ortho.jpg}
\includegraphics[width=35mm]{FIGS/SeismGuylia/Px1.jpg}
\includegraphics[width=35mm]{FIGS/SeismGuylia/Correl.jpg} %pourquoi la hauteur de l'image de correl est plus petite que le reste?

\end{center}
\caption{Guliya data set : the two ortho images, the $X$-parallax computed, and the correlation
coefficient computed}
\label{FIG:OK:Guylia}
\end{figure}


%-------------------------------------------------------------------
\subsection{Simplified interface}

A simplified interface has been written. At the time being, it gives acces to few parameters, but 
it will evolve.

\begin{verbatim}
$ mm3d MM2DPosSism
*****************************
*  Help for Elise Arg main  *
*****************************
Unnamed args : 
  * string :: {Image 1}
  * string :: {Image 2}
Named args : 
  * [Name=Masq] string :: {Masq of focus zone (def=none)}
  * [Name=Teta] REAL :: {Direction of seism if any (in radian)}
  * [Name=Exe] bool :: {Execute command , def=true (tuning purpose)}
\end{verbatim}

An example of use :

\begin{verbatim}
    mm3d MM2DPosSism 250802_ortho.tif 260608_ortho.tif Teta=1.5
\end{verbatim}


%-------------------------------------------------------------------
\subsection{Comment on the parameters}

This section describes the "classical" interface using the {\tt XML} parameters.

\subsubsection{Interpolation}

Aiming at measuring very small displacements, we use a sinus cardinal interpolation :

\begin{itemize}
   \item {\tt <ModeInterpolation> eInterpolSinCard </ModeInterpolation>}

   \item  {\tt <SzSinCard>  5.0 </SzSinCard>} specifies the size of the kernel;

   \item  {\tt  <SzAppodSinCard>  5.0 </SzAppodSinCard>} controls the shape of the appodization
          window (the general shape is a Tukey window, when SzAppodSinCard=SzSinCard, it turns to be
          a Hamming window);
\end{itemize}


\subsubsection{Image term}

By default in MicMac, the image term is $1-Cor$ where $Cor$ is the normalized cross correlation
coefficient. In such data-sets, where there is a very important change locally, this can not be
suitable because when there are changes of the nature (snow \dots)  the correlation has no
signification and it is better to consider that there is no information. %pourquoi?
Three parameters are used here to control the meaning of the correlation:


\begin{itemize}
   \item {\tt  <CorrelMin> }=$C^{min}$ ,
         so  that correlation bellow $<C^{min}$ has no influence;
   \item {\tt  <GammaCorrel> }=$\gamma$, with $\gamma$ higher, higher is the influence of the correlation
         close to $1$;
   \item {\tt  <DynamiqueCorrel>=eCoeffGamma } to activate the previous one \dots
\end{itemize}

The following equations indicate how these parameters define the conversion from correlation to cost:

\begin{equation}
    C_1=Max(Cor,C^{min}) ,
\end{equation}

\begin{equation}
   C_2 = \frac{C_1 -C^{min}}{1-C^{min}}
\end{equation}

\begin{equation}
   C_3 = {C_2} ^\gamma
\end{equation}

\begin{equation}
   Cost  = (1-C_3) * (1-C^{min});
\end{equation}

On figure~\ref{FIG:OK:Guylia}, the image on the left presents the correlation coefficients. The yellow value corresponds to the threshold value (here $<0.5$).


\subsubsection{Non isotropic regularization}

It can happen that we have an \emph{a priori} knowledge for favouring some direction of regularization.
This can be done using in conjunction the following parameters of {\tt EtapeProgDyn}:

\begin{itemize}
   \item  {\tt  <NbDir>}=$N$  fixes the number of direction that will be explored;
   \item  {\tt  <Teta0>}=$\theta_0$  fixes the angle of the favoured direction;
   \item  the directions that will be explored are $\alpha_k=\theta_0 + k\frac{\pi}{N} , k\in[0,N-1]$
   \item if the parameters  {\tt <Px1MultRegul>}=$V_1$ or {\tt <Px2MultRegul> } are used, then the value
         of regularization in the direction  $\alpha_k$ is  mutiplied by $V_1[k]$ ($V_1$ is a vector);
\end{itemize}

In this example, we regularise more the direction close to $\frac{\pi}{2}$, with a weight
$\frac{1}{1+\frac{10}{N}*K}$.




