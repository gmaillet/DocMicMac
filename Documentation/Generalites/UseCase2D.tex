\chapter{Use cases for $2D$ Matching}

The chapter cover example of using MicMac when the matching problem is
a $2$ dimensionnal problem. This can occurs in the following situation :


\begin{itemize}
   \item the problem is intrinsiquely $2$ dimensionnal, for example in
         movement detection;

   \item the problem shoulde be $1$ dimensionnal, but the orientation parameters
         are unknown or, at least, unaccurate;
\end{itemize}

%-------------------------------------------------------------------
%-------------------------------------------------------------------
%-------------------------------------------------------------------


\section{The Mars data-set}



\begin{figure}
\begin{center}
\includegraphics[width=35mm]{FIGS/Mars/SmaIm1.jpg}
\includegraphics[width=35mm]{FIGS/Mars/SmIm2.jpg}
\includegraphics[width=35mm]{FIGS/Mars/Px1.jpg}
\includegraphics[width=35mm]{FIGS/Mars/Px2.jpg}

\end{center}
\caption{Mars data-set : the two  images, the $X$ paralaxe, in gray-level, and the $Y$-paralax in
hue colour}
\label{FIG:OK:Mars}
\end{figure}

\subsection{Description of the data set}

The data can be found in directory {\tt Mars/} of directory {\tt ExempleDoc/}.
It consist of two stereo images acquired by Cassini (??) sond. In this case
we do not have the physicall model of the sensor, but we know that :

\begin{itemize}
   \item the satellite is a  pushbroom-satellite;
   \item it flights in the $x$ direction.
\end{itemize}

\subsection{Comment on the paramaters}

\subsubsection{Geometry}

The tags controlling geometry are :

\begin{itemize}

   \item   {\tt <GeomImages> eGeomImage\_Hom\_Px </GeomImages>} indicate the geometry of the acquisition,
          here it means that there is a principal homography $H$, let $P_1=x_1,y_1$ and  $P_2=x_2,y_2$ be two
          homologous points, MicMac will compute $U(P_1)$ and $V(P_1)$ such that

\begin{equation}
    P_2 = H(P_1) + (U(P_1),V(P_1))
\end{equation}

   \item  the homography $H$ is computed by MicMac from a set of homologous point;

   \item  {\tt  <FCND\_CalcHomFromI1I2> NKS-Assoc-CplIm2Hom@-Man@xml  </FCND\_CalcHomFromI1I2>}  indicates
          where {\tt MicMac} must look for the tie points (see directory {\tt Homol-Man/});


   \item  {\tt <GeomMNT> eGeomPxBiDim  </GeomMNT>} indicateseGeomImage the geoemtry of restitution,
          the value {\tt eGeomPxBiDim} indicate that what is copmuted is pixel offset, in fact this value
          is mandatory when using {\tt eGeomImage\_Hom\_Px}


\end{itemize}

\subsubsection{Matching}

In this case, the two paralax direction have completely different meanings :

\begin{itemize}
   \item the paralax $1$ represente mainly the relief, it is expected to contain high frequencies;
   \item the paralax $2$ represente mainly the error of the geometric model, it is expected to have
         low amplitude and low frequencies;
\end{itemize}

This dissymetry in \emph{a priori} knowledge on paralax is specified at different part of the file :

\begin{itemize}
   \item {\tt <Px1IncCalc>}   and {\tt <Px2IncCalc>},  representinng the global incertitude on each paralax;
   \item {\tt <Px1Regul>}   and {\tt <Px2Regul>},  representinng the \emph{a priori} knowledge on regularity of each
         paralax;
   \item {\tt <Px1PenteMax>}   and {\tt <Px2PenteMax>},  representinng the \emph{a priori} knowledge on the
         steep of each paralax;
   \item {\tt <Px1Pas>}   and {\tt <Px2Pas>},  representing the discretization step (as Px2 is low frequency and
         low amplitude, we can compute it with higher precision);
   \item {\tt <Px1DilatAlti>}   and {\tt <Px2DilatAlti>}, to gain som time, we decide not re-estimate Px2 at last step;
\end{itemize}

\subsubsection{Results}

Figure~\ref{FIG:OK:Mars} present the two images and the results of computed paralax.
As expected :

\begin{itemize}
   \item The Px1 contains mainly high frequency information on the relief;
   \item The Px2 contains mainly low frequency information on  geometry of the sensor.
\end{itemize}


%-------------------------------------------------------------------
%-------------------------------------------------------------------



\section{The Gulyia Earthquake Data-Set}


\subsection{Introduction}

Since september $2011$ CNES\footnote{Centre National d'Etudes Spatiales, the french spatial agencny}
has been funding a devlopment for using MicMac for  Earthquake quantification. This development
was made as collaboration between CNES, CEA, IPGP and IGN/ENSG. The main devlopper of this part
is Ana-Maria Rosu.

Although there exist other tools for doing this, the objective was :

\begin{itemize}
   \item have a totally free tool, that scientific can use in open source mode;
   \item have a more parametrizable tool;
\end{itemize}

Although the study is not finished, the tool  is now operationnal.

%-------------------------------------------------------------------

\subsection{Description of the data set}

The data can be found in the directory {\tt \textcolor{blue}{SeismGuyla/}} % pourquoi Guyla et non pas Guliya
 of directory {\tt ExempleDoc/}. % micmac_data/ExempleDoc/
It consists of  two \emph{Spot 5} ortho photos of the same scene taken in $2002$ and
$2008$. Between these two dates, an earthquake occured and image matching can be used to
localize the \textcolor{blue}{rupture} and quantify the movement.

We want to use MicMac to measure very \textcolor{blue}{small} displacements (arround $\frac{1}{10}$ pixel) in
a context where the images are quite different. Figure~\ref{FIG:OK:Guylia} presents the two
ortho images.

\begin{figure}
\begin{center}
\includegraphics[width=35mm]{FIGS/SeismGuylia/250802_ortho.jpg}
\includegraphics[width=35mm]{FIGS/SeismGuylia/260608_ortho.jpg}
\includegraphics[width=35mm]{FIGS/SeismGuylia/Px1.jpg}
\includegraphics[width=35mm]{FIGS/SeismGuylia/Correl.jpg} %pourquoi la hauteur de l'image de correl est plus petite que le reste?

\end{center}
\caption{Guliya data set : the two ortho images, the $X$-parallax computed, and the correlation
coefficient computed}
\label{FIG:OK:Guylia}
\end{figure}


%-------------------------------------------------------------------
\subsection{Simplified interface}

A simplified interface has been written. At the time being, it gives acces to few parameters, but 
it will evolves.

\begin{verbatim}
$ mm3d MM2DPosSism
*****************************
*  Help for Elise Arg main  *
*****************************
Unnamed args : 
  * string :: {Image 1}
  * string :: {Image 2}
Named args : 
  * [Name=Masq] string :: {Masq of focus zone (def=none)}
  * [Name=Teta] REAL :: {Direction of seism if any (in radian)}
  * [Name=Exe] bool :: {Execute command , def=true (tuning purpose)}
\end{verbatim}

An example of use :

\begin{verbatim}
    mm3d MM2DPosSism 250802_ortho.tif 260608_ortho.tif Teta=1.5
\end{verbatim}


%-------------------------------------------------------------------
\subsection{Comment on the parameters}

This section describe the "classical" interface using the {\tt XML} parameters.

\subsubsection{Interpolation}

Aiming at measuring very \textcolor{blue}{small} displacements, we use a sinus cardinal interpolation :

\begin{itemize}
   \item {\tt <ModeInterpolation> eInterpolSinCard </ModeInterpolation>}

   \item  {\tt <SzSinCard>  5.0 </SzSinCard>} specifies the size of the kernel;

   \item  {\tt  <SzAppodSinCard>  5.0 </SzAppodSinCard>} controls the shape of the appodization
          window (the general shape is a Tukey window, when SzAppodSinCard=SzSinCard, it turns to be
          a Hamming window);
\end{itemize}


\subsubsection{Image term}

By default in MicMac, the image term is $1-Cor$ where $Cor$ is the normalized cross correlation
coefficient. In such data-set, where there is a very important change locally, this can not be
suitable because when there is changes of the nature (snow \dots)  the correlation has no
signification and it is better to consider that there is no information. %pourquoi?
Three parameters are used here to control the meaning of the correlation:


\begin{itemize}
   \item {\tt  <CorrelMin> }=$C^{min}$ ,
         so  that correlation bellow $<C^{min}$ has no influence;
   \item {\tt  <GammaCorrel> }=$\gamma$, with $\gamma$ higher, higher is the influence of the correlation
         close to $1$;
   \item {\tt  <DynamiqueCorrel>=eCoeffGamma } to activate the previous one \dots
\end{itemize}

Following equations indicate how these parameters define the conversion from correlation to cost :

\begin{equation}
    C_1=Max(Cor,C^{min}) ,
\end{equation}

\begin{equation}
   C_2 = \frac{C_1 -C^{min}}{1-C^{min}}
\end{equation}

\begin{equation}
   C_3 = {C_2} ^\gamma
\end{equation}

\begin{equation}
   Cost  = (1-C_3) * (1-C^{min});
\end{equation}

On figure~\ref{FIG:OK:Guylia}, the image on the left presents the correlation coefficients. The yellow value corresponds to the threshold value (here $<0.5$).


\subsubsection{Non isotropic regularization}

It can happen that we have an \emph{a priori} knowledge for favouring some direction of regularization.
This can be done using in conjunction the following parameters of {\tt EtapeProgDyn}:

\begin{itemize}
   \item  {\tt  <NbDir>}=$N$  fixes the number of direction that will be explored;
   \item  {\tt  <Teta0>}=$\theta_0$  \textcolor{blue}{fixes the angle of the favoured direction};
   \item  the direction that will explored are $\alpha_k=\theta_0 + k\frac{\pi}{N} , k\in[0,N-1]$
   \item if the parameters  {\tt <Px1MultRegul>}=$V_1$ or {\tt <Px2MultRegul> } are used, then the value
         of regularization in the direction  $\alpha_k$ is  mutiplied by $V_1[k]$ ($V_1$ is a vector);
\end{itemize}

In this example, we regularise more the direction close to $\frac{\pi}{2}$, with a weight
$\frac{1}{1+\frac{10}{N}*K}$.




