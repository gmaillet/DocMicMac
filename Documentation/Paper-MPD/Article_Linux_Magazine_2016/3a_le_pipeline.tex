\section{Le pipeline automatique standard}\label{sec_pipeline_auto}
Le traitement d'un chantier photogrammétrique suit un enchaînement d'étapes entièrement automatique dans la majorité des cas. La suite {\tt MicMac} offre du point de vue de son organisation, présentée en \ref{subsec_organisation}, la possibilité de réaliser un script de commandes qui va réaliser le traitement. Ci-dessous sont présentées les principales étapes de l'acquisition des images à la réalisation des livrables photogrammétriques.

\subsection{Protocole d'acquisition}
L'étape préalable au traitement est celle de l'acquisition d'images de qualité. En général, l'appareil photo est paramétré par l'utilisateur en mode manuel afin de disposer de paramètres constants durant l’acquisition. La mise au point est fixée afin de garder une valeur de focale identique pour toutes les images. Pour des prises de vue terrestres, l'utilisation d'un pied photo offre la possibilité de réaliser des acquisitions même dans des conditions de luminosité défavorable et ainsi conserver une radiométrie constante sur l'ensemble des images. Tandis que pour des images acquises avec un vecteur en déplacement (drone par exemple), la priorité sera donnée au temps d'exposition (en fonction de la vitesse du vecteur) afin d'avoir des images le plus nette possible.\newline
\\
La prise de vue photogrammétrique impose une recouvrement des images dans les axes longitudinal et latéral. En général, les images présentent un taux de recouvrement respectivement de l'ordre de 75-35 $\%$.

\subsection{Extraction des points de liaisons}
MPD
...
\subsection{Orientation du bloc d'images}
MPD