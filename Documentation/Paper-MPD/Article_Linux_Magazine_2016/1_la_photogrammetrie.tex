\section{La photogrammétrie} 
\subsection{Une brève histoire de la photogrammétrie}
MPD

La photogrammétrie est la science permettant de réaliser des mesures géométriques à partir d'images stéréoscopiques. L'origine de cette technique remonte à 1849-1850, lorsque Aimé Laussedat\footnote{1819-1907} utilisa les perspectives de photographies pour réaliser des plans métriques et nomma sa méthode  "Métrophotographie". La photogrammétrie a considérablement évolué au fil du temps, passant d'un procédé nécessitant un matériel lourd et accessible uniquement à un public d'experts à une méthode accessible à tous. Au cours du 20\up{ème} siècle, la photogrammétrie employait une instrumentation mécanique lourde et onéreuse afin de réaliser des restitutions du relief en se basant sur des clichés argentiques associés à des appareils photo dits "métriques". À partir des années 1990, les images numériques ont remplacées les films argentiques et les appareils de restitution ont suivi cette évolution passant à l'aire de l'analytique où les tâches mécaniques réalisées par un opérateur qualifié sont désormais réalisées par un ordinateur.

\subsection{Les produits de la photogrammétrie}
MPD
La photogrammétrie offre un certain nombre de produits dont les plus communs sont : La chaîne de traitement qui permet à partir d'images stéréoscopiques de réaliser les produits suivants est détaillée à la section \ref{sec_pipeline_auto}.
\subsection{Avantages et limites de la photogrammétrie}