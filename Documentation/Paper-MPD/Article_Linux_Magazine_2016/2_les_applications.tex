\section{Les domaines d'application}
ER\\
%
Photogrammetry is a cheap and instantenous methodology of performing 2D and 3D documentation of the surrounding world. Due to that, as well as thanks to the rapid developpements in automated data processing, photogrammetry has found its applications in various fields of science and engineering.\par
%
In architecture and archeology, it is commonly used in architectural analyses of monuments and sites or within the conservation and restoration activities. The main purpose of photogrammetry is to deliver high precision representations of objects in terms of their geometry, photometry (visual appearance) and possibly the material type. Among the the 2D products, one can distinguish engineering drawings of ground plans, horizontal and vertical sections of architectural objects, or the so called orthophomaps, i.e. textured maps of planar (facade building) or other parametric surfaces (cylinders, dishes, cones). The typical 3D products are 3D vector models, textured polygonal models or point clouds.\par 
%
In the BTP sector, and the industry in general, photogrammetry is 


- 

Les applications
champs d'appli : archi, archéo, BTP construction , carto, agriculture, géologie, sismo …
les catpeurs appareils reflex, compact, vidéo, caméra photogrammétrique
les vecteurs : drone, terrestre avions, satellite ….