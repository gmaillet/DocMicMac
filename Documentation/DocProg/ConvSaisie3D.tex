
%\selectlanguage{english}

\section{Conventions for 3D selection tool}

SaisieMasqQT allows to open ply files and to do some manual segmentation with a polygonal selection tool.
User can mainly perform 2 actions:
\begin{itemize}
\item move camera around point cloud (rotate and/or translate)
\item draw a polygon and select/deselect points inside polygon
\end{itemize}

SaisieMasqQT can store an xml file (myPly_selectionInfo.xml) with polygonal selection information (camera position, and viewport coordinates of polygon vertex).

For each pair of camera pose and polygonal selection, xml file contains a tag <Item> with:
\begin{itemize}
\item camera pose matrix
\item openGL viewport size
\item polygon vertex viewport coordinates
\item selection mode (add, remove, invert, etc.)
\end{itemize} 

Two camera pose matrix are stored using openGL conventions:

\begin{itemize}
\item modelView matrix (16 parameters)
\itme projection matrix (16 parameters)
\end{itemize}

For more information on these matrix: 

http://www.opengl-tutorial.org/beginners-tutorials/tutorial-3-matrices/#The_Model__View_and_Projection_matrices


How to transform polygon viewport coordinates (as stored) into 3D world coordinates:

HistoryManager->load()
QVector <selectInfos> vInfos = HistoryManager->getSelectInfos();
for (int aK=0; aK< vInfos.size();++aK)
{
	MatrixManager->importMatrices(vInfos[aK]);
	for (int bK=0;bK < vInfos[aK].poly.size();++bK)
	{
		QPointF pt = vInfos[aK].poly[bK];
		MatrixManager->InverseProjection(pt, 100, pt3d);
	}
}

