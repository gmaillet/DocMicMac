\section{Conventions for 3D selection tool}
\label{Conv:selection}

{\tt SaisieMasqQT} allows to open ply files and to do some manual segmentation with a polygonal selection tool.
User can mainly perform 2 actions:
\begin{itemize}
\item move camera around point cloud (rotate and/or translate)
\item draw a polygon and select/deselect points inside polygon\\*
\end{itemize}

{\tt SaisieMasqQT} can store an xml file (selectionInfo.xml) with polygonal selection information (camera position, and viewport coordinates of polygon vertex).\\*

For each pair of camera pose and polygonal selection, xml file contains a tag {\tt <Item> } with:
\begin{itemize}
\item camera pose matrix
\item openGL viewport size (4 parameters)
\item a list of polygon vertex viewport coordinates
\item selection mode (add, remove, invert, etc.)\\*
\end{itemize}

Two camera pose matrix are stored using openGL conventions:

\begin{itemize}
\item model-view matrix (16 parameters)
\item projection matrix (16 parameters)\\*
\end{itemize}

For more information on these matrix: \url{http://www.glprogramming.com/red/chapter03.html}\\*

How to transform polygon viewport coordinates (as stored) into world coordinates:

\begin{itemize}
\item In  : point in viewport coordinates P(x,y)
\item Out : point in world coordinates P'(x',y',z')\\*
\end{itemize}

First, we map x and y from viewport coordinates to [-1,1]

\begin{eqnarray*}
        & x = 2 * ( x - viewport[0] ) / viewport[2] - 1\\
    & y = 2 * ( y - viewport[1] ) / viewport[3] - 1
\end{eqnarray*}

We compute global projection matrix from camera to world coordinates:

\begin{equation*}
        M  = ModelViewMatrix * ProjectionMatrix
\end{equation*}

Let's define a point $P_h$ in homogeneous coordinates in the image plane:

\begin{eqnarray*}
        & Ph[0] = x\\
        & Ph[1] = y\\
        & Ph[2] = 0\\
        & Ph[3] = 1
\end{eqnarray*}

Projection point $P_r$ will be:

\begin{equation*}
        P_r = P_h*M
\end{equation*}

And final point in world coordinates is:

\begin{eqnarray*}
        & x' = P_{r}[0]/P_{r}[3] \\
        & y' = P_{r}[1]/P_{r}[3] \\
        & z' = P_{r}[2]/P_{r}[3]
\end{eqnarray*}


These operations are performed in function {\tt getInverseProjection} from {\tt saisieQT/MatrixManager.cpp }

a function to convert a polygon stored in {\tt selectInfo.xml} (filename) into point cloud local frame looks like:

\begin{verbatim}

#include MatrixManager.h

HistoryManager *HM = new HistoryManager();
MatrixManager  *MM = new MatrixManager();

HM->load(filename);
QVector <selectInfos> vInfos = HM->getSelectInfos();

for (int aK=0; aK< vInfos.size();++aK)
{
    selectInfos &Infos = vInfos[aK];
    MM->importMatrices(Infos);

    for (int bK=0;bK < Infos.poly.size();++bK)
    {
        QPointF pt = Infos.poly[bK];
        Pt3dr pt3d;
        MM->getInverseProjection(pt3d, pt, 0.f);

        std::cout << pt3d.x  << " " << pt3d.y << " " << pt3d.z << std::endl;
    }
}

\end{verbatim}



