\documentclass{article}
\usepackage[utf8]{inputenc}
\usepackage[english]{babel}

\usepackage{minted}

\usepackage{tikz}
\usetikzlibrary{arrows,positioning}


\newcommand\file[1]{
\textit{\textcolor{blue}{#1}}
}
\newcommand\commande[1]{
\textbf{#1}
}

\begin{document}

\section{Introduction}
Ce document a \'et\'e r\'ealis\'e durant la formation MICMAC d\'eveloppeur r\'ealis\'e �  l'ENSG du 08 au 12 d\'ecembre.
Durant ces 5 jours de formation,  plusieurs exercices ont \'et\'e r\'ealis\'es.
Le premier concerne la conversion de calibration de cam\'era d'un format dans un autre en utilisant des points d'appuis fictifs.
Le second concerne l'ajout d'\'equations d'observations concernant une paire de cam\'era rigidement li\'ee et synchronis\'ee.

Avant d'expliquer les diff\'erents exercices, voici deux rappels succincts concernant la mise �  jour des sources depuis mercurial ainsi que la compilation des sources.

\subsection{Rappel Mercurial}
\begin{itemize}
\item \commande{hg push} permet de mettre �  jour le d\'epôt local (dossier micmac) en r\'ecup\'erant les sources du d\'epôt distant (ign.fr)
\item \commande{hg update} permet de mettre �  jour les sources depuis le d\'epôt local  (dossier micmac)
\end{itemize}
\subsection{Rappel Compilation}
\begin{itemize}
\item \commande{make -j X} compile les sources et cr\'eer les binaires
\item \commande{make install} copie les binaires dans le dossier .../micmac/bin
\end{itemize}

\section{Jour 1 :  Conversion de calibration}
Le but de cet exercice est de cr\'eer un programme permettant de transformer un format de camera en un autre format.
Notre programme va:
\begin{itemize}
\item Lire le fichier cam\'era
\item G\'en\'erer une s\'erie de points en g\'eom\'etrie image $(c,l)$
\item Calculer des points terrain $(X,Y,Z)$ �  partir des coordonn\'ees images $(c,l)$
\item G\'en\'er\'e les fichiers de mesures et de points d'appuis n\'ecessaires �  Apero (qui effectuera le calcul de la nouvelle calibration �  partir des donn\'ees g\'en\'er\'ees)
\end{itemize}

\subsection{Objets utiles}

\subsubsection{Parseurs d'arguments}
MicMac possède son propre parseur d'argument qui permet de sp\'ecifier les paramètres obligatoires et facultatifs.
Ce parseur permet de v\'erifier le bon type de donn\'ees ainsi que le bon nombre d'argument.

\begin{minted}{cpp}
int main(int argc, char** argv)
        double a;
        std::string b;
        int c;
        bool d;
ElInitArgMain
(
//nb d'arguments
   argc,
//chaines de caracteres contenants tous les arguments
   argv,
//arg obligatoires
   LArgMain() << EAMC(a, "Param a",eSAM_IsExistFile) << EAMC(b,"Param b",eSAM_IsExistFile),
//arg facultatifs
   LArgMain() << EAM(c,"C",true,"Param C") << EAM(d,"D",true,"Param D")
);
   return EXIT_SUCCESS;
}
 \end{minted}

\subsubsection{Cam\'eras}
La lecture d'un fichier XML de cam\'era se fait par la fonction \emph{Std\_Cal\_From\_File} qui retourne un pointeur sur un objet de type cam\'era.
\begin{minted}{cpp}
CamStenope * aCamera =  Std_Cal_From_File("camera.xml");
\end{minted}


Fonction permettant de passer d'un point image (coordonn\'ees exprim\'ees en $(c,l)$) et une distance �  un point 3D ($(X,Y,Z)$)
\begin{minted}{cpp}
Pt2dr PtIm (123.4,567.8);
double distance=12.4;
Pt3dr aPGround = aCamera->ImEtProf2Terrain(PtIm,distance);
\end{minted}

\subsection{G\'en\'eration des fichiers}
\subsubsection{Fichiers XML}
MicMac utilise plusieurs fichiers xml pour fonctionner.
Il existe des classes cpp permettant de les \'ecrire automatiquement.
Pour chaque balise XML, il existe une classe permettant de la lire.
Par exemple pour la balise \emph{DicoAppuisFlottant}, il existe une classe \emph{cDicoAppuisFlottant}.
Les balises filles sont des donn\'ees membres de la classe issue du nom de la balise mère.
Voici le code cpp ainsi que le fichier xml g\'en\'er\'e pour le fichier appui et le fichier mesure.

\subsubsection{Fichier appui}

Code cpp :

\begin{minted}{cpp}
cDicoAppuisFlottant aDAF;

cOneAppuisDAF anApA;
Pt3dr aPGroundA(979.094 6533.994 1.055);
anApA.Pt() = aPGroundA;
anApA.Incertitude() = Pt3dr(1.,1.,1.);
anApA.NamePt() = "Megev1055";

cOneAppuisDAF anApB;
Pt3dr aPGroundB( 981.103 6553.402 1.102);
anApB.Pt() = aPGroundB;
anApB.Incertitude() = Pt3dr(1.,1.,1.);
anApB.NamePt() = "Carroz1002";

aDAF.OneAppuisDAF().push_back(anApA);
aDAF.OneAppuisDAF().push_back(anApB);

MakeFileXML(aDAF, "Mes3D.xml");

\end{minted}

Fichier XML g\'en\'er\'e :


\begin{minted}{xml}
<?xml version="1.0" ?>
<DicoAppuisFlottant>
     <OneAppuisDAF>
          <Pt> 979.094 6533.994 1.055 </Pt>
          <NamePt> Megev1055 </NamePt>
          <Incertitude>1 1 1</Incertitude>
     </OneAppuisDAF>
     <OneAppuisDAF>
          <Pt>  981.103 6553.402 1.102 </Pt>
          <NamePt> Carroz1002 </NamePt>
          <Incertitude>1 1 1</Incertitude>
     </OneAppuisDAF>
</DicoAppuisFlottant>
\end{minted}

\subsubsection{Fichier mesure}

Code cpp :

\begin{minted}{cpp}
cSetOfMesureAppuisFlottants XMLMesuresImages;


cMesureAppuiFlottant1Im XMLImageA;
XMLImageA.NameIm()="T0-_MG_0001.JPG";

cOneMesureAF1I XMLMesureA;
XMLMesureA.NamePt()="Megev1055";
XMLMesureA.PtIm()=Pt2dr(189.32188537332104,2365.73434175065222);
XMLImageA.OneMesureAF1I().push_back(XMLMesureA);

cOneMesureAF1I XMLMesureB;
XMLMesureB.NamePt()="Carroz1002";
XMLMesureB.PtIm()=Pt2dr(11395.38751581008842,862.673280912829);
XMLImageA.OneMesureAF1I().push_back(XMLMesureB);


cMesureAppuiFlottant1Im XMLImageB;
XMLImageB.NameIm()="T0-_MG_0002.JPG";

cOneMesureAF1I XMLMesureC;
XMLMesureC.NamePt()="Megev1055";
XMLMesureC.PtIm()=Pt2dr(1703.75004653847122,2356.52137674056758);
XMLImageB.OneMesureAF1I().push_back(XMLMesureA);

cOneMesureAF1I XMLMesureD;
XMLMesureD.NamePt()="Carroz1002";
XMLMesureD.PtIm()=Pt2dr(1872.70867622042897,849.874687754920728);
XMLImageB.OneMesureAF1I().push_back(XMLMesureB);

MakeFileXML(XMLImage, "Mesure2D.xml");

\end{minted}


Fichier XML g\'en\'er\'e :

\begin{minted}{xml}
<?xml version="1.0" ?>
<SetOfMesureAppuisFlottants>
     <MesureAppuiFlottant1Im>
          <NameIm>T0-_MG_0001.JPG</NameIm>
          <OneMesureAF1I>
               <NamePt>Megev1055</NamePt>
               <PtIm>1189.32188537332104 2365.73434175065222</PtIm>
          </OneMesureAF1I>
          <OneMesureAF1I>
               <NamePt>Carroz1002</NamePt>
               <PtIm>1395.38751581008842 862.673280912829</PtIm>
          </OneMesureAF1I>
     </MesureAppuiFlottant1Im>
     <MesureAppuiFlottant1Im>
          <NameIm>T0-_MG_0002.JPG</NameIm>
          <OneMesureAF1I>
               <NamePt>Megev1055</NamePt>
               <PtIm>1703.75004653847122 2356.52137674056758</PtIm>
          </OneMesureAF1I>
          <OneMesureAF1I>
               <NamePt>Carroz1002</NamePt>
                <PtIm>1872.70867622042897 849.874687754920728</PtIm>
          </OneMesureAF1I>
     </MesureAppuiFlottant1Im>
<SetOfMesureAppuisFlottants>
\end{minted}

\subsubsection{Utilisation de programmes MicMac}
On peut \'egalement faire appel �  un programme de la suite MicMac �  partir d'un autre.
Il suffit d'utiliser la fonction \emph{MM3dBinFile} avec le nom du programme �  utiliser ainsi que les arguments n\'ecessaires au programme puis de faire appel �  la commande  \emph{System}
\begin{minted}{cpp}
std::string aCom =    MM3dBinFile("Apero")
                    + XML_MM_File("Apero-ConvCal.xml")
                    + " DirectoryChantier=" +DirOfFile(aCalibIn)
                    + " +AeroIn=ConvCalib"
                    + " +CalibIn="  + aCalibOut
                    + " +CalibOut=" + aNameCalibOut
                    + " +FocFree="  + ToString(FocFree)
                    + " +PPFree="   + ToString(PPFree)
                    + " +CDFree="   + ToString(CDFree)
                    + " +DRMax=" + ToString(aDRMax)
                    + " +DegGen=" + ToString(aDegGen)
                   ;


System(aCom);
\end{minted}

\subsubsection{Int\'egration dans la chaîne MicMac}
Pour integrer son programme dans la chaîne MicMac et l'appeler avec la syntaxe \emph{mm3d MonProgramme}, il faut ajouter dans le fichier \file{src/CBinaires/mm3d.cpp} :
\begin{itemize}
\item \emph{extern int MonProgramme (int argc, char **argv)} : d\'eclaration de la fonction
\item \emph{aRes.push\_back(cMMCom("MonProgramme",MonProgramme,"Explication de mon programme"));} : int\'egration dans la suite mm3d avec le nom du programme "MonProgramme" ainsi que son explication "Explication de mon programme"
\end{itemize}

\section{Jour 2 et 3 : Impl\'ementation d'un bloc rigide de 2 cam\'eras}

\subsection{Donn\'ees utilis\'ees}
La cam\'era utilis\'e pour ce jeux de donn\'ees genere des images au format MPO.
Ce format permet de stocker dans un même fichier les deux images prises par la cam\'era.
Un outil de MicMac permet de s\'eparer le fichier MPO en 2 fichiers images.
Voici la commande
\emph{mm3d SplitMPO ".MPO" pour r\'ecup\'erer les paires d'images.}

\subsection{Description des diff\'erents processus}
%Utilisation du logiciel :

% \begin{tikzpicture}[node distance=1cm, auto]
% \tikzset{
%     mynode/.style={rectangle,rounded corners,draw=black, top color=white, bottom color=lightgray,very thick, inner sep=1em, minimum size=3em, text centered},
%     myarrow/.style={->, >=latex', shorten >=1pt, thick},
%     mylabel/.style={text width=7em, text centered}
% }
% \node[mynode] (xml) {Fichier XML de paramètres};
% \node[mynode,below=of xml] (donnees) {\begin{tabular}{c}Donnees Annexes \\ (images,pts homologues...)\end{tabular}};
% \node[mynode,right=of xml] (apero) {Apero};
%
% \draw[myarrow] (xml.east) --  (apero.west);
% \draw[myarrow] (donnees.east) -- ++(0.5,0)  |-  (apero.west);
% \end{tikzpicture}

Plusieurs \'etapes sont n\'ecessaires pour ajouter des observations dans le système de compensation de MicMac :
\begin{enumerate}
\item Écriture math\'ematique des \'equations dans le système de calcul formel distribu\'e par MicMac
\item G\'en\'eration des fichiers cpp contenant les classes avec entre autre la diff\'erentielle et le calcul de r\'esidus
\item Ajout dans le système de compilation MicMac
\item Modification du fichier de paramètres
\item Cr\'eation des fichiers cpp pour la lecture du fichier de paramètres
\item Interaction entre tout ça
\end{enumerate}


La plupart du code d\'ecrit ci-après provient du fichier

\noindent \file{micmac/src/uti\_phgrm/Apero/cImplemBlockCam.cpp}

\subsubsection{Écriture math\'ematique des \'equations}

On appelle $i$ l'instant ou des images sont prises en même temps.
On appelle $a$ et $b$ les deux cam\'eras de la paire st\'er\'eo.
Considerer un block rigide de 2 cam\'eras $a$ et $b$ signifie  que la transformation $T$ comportant rotation $R$ et translation $T$ entre 2 instants est constant.

\begin{equation}
T_{a,b,i}=T_{a,b,i+1}
\end{equation}

MicMac possède son sytème de calculs formels.
Cela permet de calculer automatiquement la d\'eriv\'ee des observations par rapport aux paramètres mais \'egalement d'optimiser le code g\'en\'er\'e (minimisation du nombre de calculs \'el\'ementaires).

Quelques \'el\'ements se trouvent dans les fichiers \file{/micmac/include/general/phgr\_formel.h} et dans \file{/micmac/src/uti\_phgrm/Apero/Conventions.txt}


Voici les 2 fonctions concern\'ees par le calcul formel

\begin{minted}{cpp}

void GenerateCode()
{
          cSetEqFormelles *   mSet;
          cRotationFormelle * mRotRT0;
          cRotationFormelle * mRotLT0;
          cRotationFormelle * mRotRT1;
          cRotationFormelle * mRotLT1;
          cIncListInterv      mLInterv;
          cElCompiledFonc*    mFoncEqResidu;

// permet de donner un nom aux inconnues
// sert a la localisation dans le systeme d'equation

   mRotRT0->IncInterv().SetName("OriR0");
   mRotLT0->IncInterv().SetName("OriL0");
   mRotRT1->IncInterv().SetName("OriR1");
   mRotLT1->IncInterv().SetName("OriL1");

   mLInterv.AddInterv(mRotRT0->IncInterv());
   mLInterv.AddInterv(mRotLT0->IncInterv());
   mLInterv.AddInterv(mRotRT1->IncInterv());
   mLInterv.AddInterv(mRotLT1->IncInterv());

// equation d'observation
    Pt3d<Fonc_Num>     aTrT0;
    ElMatrix<Fonc_Num> aMatT0(3,3);
    CalcParamEqRel(aTrT0,aMatT0,*mRotRT0,*mRotLT0,0,1);

    Pt3d<Fonc_Num>     aTrT1;
    ElMatrix<Fonc_Num> aMatT1(3,3);
    CalcParamEqRel(aTrT1,aMatT1,*mRotRT1,*mRotLT1,2,3);

    Pt3d<Fonc_Num> aResTr = aTrT1-aTrT0;
    ElMatrix<Fonc_Num> aResMat = aMatT1-aMatT0;

// residus
    std::vector<Fonc_Num> aVF;
    aVF.push_back(aResTr.x);
    aVF.push_back(aResTr.y);
    aVF.push_back(aResTr.z);
    for (int aKi=0 ; aKi<3 ; aKi++)
        for (int aKj=0 ; aKj<3 ; aKj++)
           aVF.push_back(aResMat(aKi,aKj));

// generation des classes cpp a partir du code precedent
    cElCompileFN::DoEverything
    (
        DIRECTORY_GENCODE_FORMEL,  // Directory ou est localise le code genere
        "cCodeBlockCam",  // donne les noms de fichier .cpp et .h ainsi que les nom de classe
        aVF,  // expressions formelles utilisees pour les observations
        mLInterv  // intervalle de reference
    );
}


void GenerateCodeBlockCam()
{
   cSetEqFormelles aSet;

   ElRotation3D aRot(Pt3dr(0,0,0),0,0,0);
   cRotationFormelle * aRotRT0 = aSet.NewRotation (cNameSpaceEqF::eRotLibre,aRot);
   cRotationFormelle * aRotLT0 = aSet.NewRotation (cNameSpaceEqF::eRotLibre,aRot);
   cRotationFormelle * aRotRT1 = aSet.NewRotation (cNameSpaceEqF::eRotLibre,aRot);
   cRotationFormelle * aRotLT1 = aSet.NewRotation (cNameSpaceEqF::eRotLibre,aRot);


  cEqObsBlockCam * aEOBC = aSet.NewEqBlockCal (*aRotRT0,*aRotLT0,*aRotRT1,*aRotLT1,true);
  DoNothingButRemoveWarningUnused(aEOBC);
}

\end{minted}


\subsubsection{G\'en\'eration du code CPP}
Un fois la fonction permettant l'\'equation formelle \'ecrite, il faut appeler l'utilitaire qui permet de g\'en\'erer la classe effectuant le calcul.
Il s'agit de \emph{GenCode} fichier : \file{src/uti\_files/CPP\_GenCode.cpp} et d'appeler la fonction \emph{GenerateCodeBlockCam()}.

Deux fichiers sont alors cr\'e\'es \file{cCodeBlockCam.h} et  \file{cCodeBlockCam.cpp} dans le dossier \file{/micmac/CodeGenere/photogram/} .
Ces fichiers contiennent plusieurs fonctions dont :

\begin{itemize}
\item ComputeVal() : calculs les r\'esidus
\item ComputeValDeriv() : calcul la d\'eriv\'ee par rapport aux inconnues
\end{itemize}

\subsubsection{Ajout du code dans MicMac}

Ensuite, il faut ajouter ces fichiers dans le code de MicMac.
Pour cela il faut ajouter
\begin{itemize}
\item dans \emph{src/photogram/phgr\_or\_code\_gen00.cpp}
\end{itemize}
\begin{minted}{cpp}
#include "../../CodeGenere/photogram/cCodeBlockCam.h"
AddEntry("cCodeBlockCam",cCodeBlockCam::Alloc);
\end{minted}

\begin{itemize}
\item dans \emph{src/photogram/phgr\_or\_code\_gen6.cpp}
\end{itemize}
\begin{minted}{cpp}
#include "../../CodeGenere/photogram/cCodeBlockCam.cpp"
\end{minted}


\footnote{MicMac possède un m\'ecanisme de rangement des inconnues afin de minimiser le nombre de coefficients non nul de la matrice �  inverser (gain de place et gain de temps de calcul). Ce m\'ecanisme n'est pas pr\'esent\'e dans ce document mais est pr\'esent dans les sources. Il s'agit de toutes les fonctions et objets contenant AMD.}



\subsubsection{Modification du fichier de paramètres}
Lorsqu'on veut \'ecrire de nouvelles observations, il faut pouvoir indiquer au programme via les fichiers xmls traditionnels, les diff\'erents paramètres qui sont n\'ecessaires.
Il s'agit du fichier
\file{include/XML\_GEN/ParamApero.xml}

Cette partie a \'et\'e ajout\'e afin de tenir compte des pond\'erations sur un bloc rigide.
La pond\'eration sur la translation et la rotation a \'et\'e diff\'erenci\'ee.

\begin{minted}{xml}
<RigidBlockWeighting Nb="1" Class="true" ToReference="true">
    <PondOnTr Nb="1" Type="double"> </PondOnTr>
   <PondOnRot Nb="1" Type="double"> </PondOnRot>
</RigidBlockWeighting>
\end{minted}

\begin{minted}{xml}
<UseForBundle Nb="?">
    <GlobalBundle Nb="1" Type="bool">      </GlobalBundle>
    <RelTimeBundle Nb="1" Type="bool">     </RelTimeBundle>
</UseForBundle>
\end{minted}

Lien vers la pond\'eration d\'eclar\'e pr\'ec\'edemment.
\emph{GlobalBundle} correspond �  un bloc rigide pour toute une acuqisition tandis que \emph{RelTimePond} correspond �  un bloc rigide de proche en proche (on tolère une variation de la rigidit\'e pour toute une acuqisition)


\begin{minted}{xml}
<GlobalPond Nb="?" RefType="RigidBlockWeighting"> </GlobalPond>
<RelTimePond Nb="?" RefType="RigidBlockWeighting"> </RelTimePond>
\end{minted}



\subsubsection{Int\'egration complète dans MicMac}

Dans la fonction \emph{void cAppliApero::AddObservations} du fichier \file{micmac/src/uti\_phgrm/Apero/cAA\_Compensation.cpp} on doit ajouter :
\begin{minted}{cpp}
          AddObservationsRigidBlockCam(anSO.ObsBlockCamRig(),IsLastIter,aSO);
\end{minted}

Avec dans le même fichier son impl\'ementation :
\begin{minted}{cpp}
void cAppliApero::AddObservationsRigidBlockCam
     (
         const std::list<cObsBlockCamRig> & anOBCR,
         bool IsLastIter,
         cStatObs & aSO
     )
{
    for
    (
       std::list<cObsBlockCamRig>::const_iterator itO=anOBCR.begin();
       itO !=anOBCR.end();
       itO++
    )
    {
                //cette fonction renvoie a l'implementation dans
                // cImplemBlockCam.cpp
         AddObservationsRigidBlockCam(*itO,IsLastIter,aSO);
    }

}
\end{minted}

La classe cImplemBlockCam du fichier \file{micmac/src/uti\_phgrm/Apero/cImplemBlockCam.cpp} effectue tout le travail.
Le constructeur permet d'initialiser toutes les donn\'ees (lecture de la transformation initiale, tri des cam\'eras...).
La fonction \emph{DoCompensation} pose les \'equations d'observations dans la matrice normale

\end{document}




EXO-PAIRE-RIGID
 Ori-UnScaledLR = orientation avec tapas

 Fraser model
  X = (1+b1)x +...
  Y = (1+b2)y +...



Apero Apero-EstimBlock.xml
Estimation de la solution initiale du block

R i camera par rappor au terrain
R' camera A vers B
R'i -1 = Rj-1 Rj


mm3d GenCode => generation des fichiers

Ajouter le code generer dans les fichiers
/home/ecg/Bureau/micmac/src/photogram/phgr_or_code_gen00.cpp
/home/ecg/Bureau/micmac/src/photogram/phgr_or_code_gen6.cpp


Exercice :
toutes les poses sont exprim\'es
on rajoute juste un observation de rigidit\'e :
        cas 1 une observation pour tous les couples
        cas 2 un observation pour chaque couple avec lien avec la rigidit\'e precedente.

modification du XML de APERO.xml afin de prendre en compte les nouvelles inconnues

                    <UseForBundle Nb="?" >
                                                <GlobalBundle      Nb="1" Type="bool"> </GlobalBundle>
                                                <RelTemporalBundle Nb="1" Type="bool"> </RelTemporalBundle>
                                        </UseForBundle>


Compilation du fichier XML en CPP
 make -f Makefile-XML2CPP  all

Compilation Global...
Modification de implment block cam pour prendre en compte la lecture du fichier xml


Ajout d'\'equations


\end{document}
